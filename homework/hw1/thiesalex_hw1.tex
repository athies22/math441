%!TEX output_directory = temp
\documentclass[letterpaper, 12pt]{amsart}
	%%%%%%%%%%%%%%%%%%%%%%%%%%%%%%%%%%%%%%%%%%%%%%%%%%%%%%%%%%%%%%%%%%%%%%%%%%%%%%
	%%%%%%%%%%%%%%%%%%%%%%%%%%%% boilerplate packages %%%%%%%%%%%%%%%%%%%%%%%%%%%%
	\usepackage{amsmath,amssymb,amsthm}
	\usepackage[mathscr]{euscript}
	\usepackage{enumerate}
	\usepackage{graphicx}
	\usepackage{mathrsfs}
	\usepackage{color}
	\usepackage{hyperref}
	\usepackage{verbatim}
	\usepackage{stmaryrd}
	\usepackage[margin=1.25in]{geometry}

	%%%%%%%%%%%%%%%%%%%%%%%%%%%%%%%%%%%%%%%%%%%%%%%%%%%%%%%%%%%%%%%%%%%%%%%%%%%%%%
	%%%%%%%%%%%%%%%%%%%%%%%%%%%%% rename the abstract %%%%%%%%%%%%%%%%%%%%%%%%%%%%
	% \renewcommand{\abstractname}{Introduction}

	%%%%%%%%%%%%%%%%%%%%%%%%%%%%%%%%%%%%%%%%%%%%%%%%%%%%%%%%%%%%%%%%%%%%%%%%%%%%%%
	%%%%%%%%%%%%%%%%%%%%%%%%%%%%%%%%%%%%% sets %%%%%%%%%%%%%%%%%%%%%%%%%%%%%%%%%%%
		%% sets 
		\DeclareMathOperator{\N}{\mathbb{N}}
		\DeclareMathOperator{\Z}{\mathbb{Z}}
		\DeclareMathOperator{\Zp}{\mathbb{Z}^{+}}
		\DeclareMathOperator{\Q}{\mathbb{Q}}
		\DeclareMathOperator{\Qp}{\mathbb{Q}^{+}}
		\DeclareMathOperator{\Qc}{\mathbb{Q}^{c}}
		\DeclareMathOperator{\R}{\mathbb{R}}
		\DeclareMathOperator{\F}{\mathbb{F}}
		\DeclareMathOperator{\Rp}{\mathbb{R}^{+}}
		\DeclareMathOperator{\C}{\mathbb{C}}
		\DeclareMathOperator{\Cnon}{\mathbb{C}^{\times}}
		%% powerset of a set
		\DeclareMathOperator{\pset}{\mathcal{P}}
		%% set of continuous functions in a certain variable
		\DeclareMathOperator{\cont}{\mathscr{C}}
		%% set of functions in a certain variable
		\DeclareMathOperator{\func}{\mathscr{F}}
		
	%%%%%%%%%%%%%%%%%%%%%%%%%%%%%%%%%%%%%%%%%%%%%%%%%%%%%%%%%%%%%%%%%%%%%%%%%%%%%%
	%%%%%%%%%%%%%%%%%%%%%%%%%%%%%%%% linear algebra %%%%%%%%%%%%%%%%%%%%%%%%%%%%%%
		%% linear span
		\DeclareMathOperator{\Ell}{\mathscr{L}}
		%% bold vectors with arrows, and bold matrices
		\newcommand{\bmat}[1]{{\mathbf{#1}}}
		\newcommand{\bvec}[1]{{\vec{\mathbf{#1}}}}
		%% independent vectors/matrices
		\DeclareMathOperator{\ind}{\perp\!\!\!\perp}
		%% order
		\DeclareMathOperator{\ord}{\text{ord}}

	%%%%%%%%%%%%%%%%%%%%%%%%%%%%%%%%%%%%%%%%%%%%%%%%%%%%%%%%%%%%%%%%%%%%%%%%%%%%%%
	%%%%%%%%%%%%%%%%%%%%%%%%%%% probability & statistics %%%%%%%%%%%%%%%%%%%%%%%%%
		%% probability, expectation, variance, etc.
		\renewcommand{\Pr}{\mathbb{P}}
		\DeclareMathOperator{\E}{\mathbb{E}}
		\DeclareMathOperator{\var}{\rm Var}
		\DeclareMathOperator{\sd}{\rm SD}
		\DeclareMathOperator{\cov}{\rm Cov}
		\DeclareMathOperator{\SE}{\rm SE}
		\DeclareMathOperator{\ssreg}{{\rm SS}_{{\rm Reg}}}
		\DeclareMathOperator{\ssr}{{\rm SS}_{{\rm Res}}}
		\DeclareMathOperator{\sst}{{\rm SS}_{{\rm Tot}}}

	%%%%%%%%%%%%%%%%%%%%%%%%%%%%%%%%%%%%%%%%%%%%%%%%%%%%%%%%%%%%%%%%%%%%%%%%%%%%%%
	%%%%%%%%%%%%%%%%%%%%%%%%%%%%%%%% congruences %%%%%%%%%%%%%%%%%%%%%%%%%%%%%%%%%
		\renewcommand{\mod}[1]{\ (\mathrm{mod}\ #1)}

	%%%%%%%%%%%%%%%%%%%%%%%%%%%%%%%%%%%%%%%%%%%%%%%%%%%%%%%%%%%%%%%%%%%%%%%%%%%%%%
	%%%%%%%%%%%%%%%%%%%%%%%%%%%%%% bracket notation %%%%%%%%%%%%%%%%%%%%%%%%%%%%%%
		% I first used this for principal ideals, that is why the abbreviation is pid
		\newcommand{\pid}[1]{\langle #1 \rangle}

	%%%%%%%%%%%%%%%%%%%%%%%%%%%%%%%%%%%%%%%%%%%%%%%%%%%%%%%%%%%%%%%%%%%%%%%%%%%%%%
	%%%%%%%%%%%%%%%%%%%%%%%%%%%%%%% fatdot notation %%%%%%%%%%%%%%%%%%%%%%%%%%%%%%
		\makeatletter
			\newcommand*\fatdot{\mathpalette\fatdot@{.5}}
			\newcommand*\fatdot@[2]{\mathbin{\vcenter{\hbox{\scalebox{#2}{$\m@th#1\bullet$}}}}}
		\makeatother

	%%%%%%%%%%%%%%%%%%%%%%%%%%%%%%%%%%%%%%%%%%%%%%%%%%%%%%%%%%%%%%%%%%%%%%%%%%%%%%
	%%%%%%%%%%%%%%%%%%%%%%%%%%%%%% use pretty letters %%%%%%%%%%%%%%%%%%%%%%%%%%%%
		\DeclareMathOperator{\ep}{\varepsilon}
		\DeclareMathOperator{\ph}{\varphi}

	%%%%%%%%%%%%%%%%%%%%%%%%%%%%%%%%%%%%%%%%%%%%%%%%%%%%%%%%%%%%%%%%%%%%%%%%%%%%%%
	%%%%%%%%%%%%%%%%%%%%%%%%%%% stolen from Jeske/Dugger %%%%%%%%%%%%%%%%%%%%%%%%%
	% Some theorem-like environments, all numbered together starting at 1
	% in each section.

	% The default \theoremstyle is bold headings and italic body text.
	\newtheorem{thm}{Theorem}[section]
	\newtheorem{defn}[thm]{Definition}
	\newtheorem{prop}[thm]{Proposition}
	\newtheorem{claim}[thm]{Claim}
	\newtheorem{cor}[thm]{Corollary}
	\newtheorem{lemma}[thm]{Lemma}

	\theoremstyle{definition}  % Bold headings and Roman body text.
	\newtheorem{example}[thm]{Example}
	\newtheorem{examples}[thm]{Examples}
	\newtheorem{exercise}[thm]{Exercise}
	\newtheorem{note}[thm]{Note}
	\newtheorem{remark}[thm]{Remark}
	\newtheorem{remarks}[thm]{Remarks}
	\newtheorem{discussion}[thm]{Discussion}

	\newcommand{\dfn}{\textbf} % Make defined words bold.
	\newcommand{\mdfn}[1]{\dfn{\mathversion{bold}#1}} % Even make math symbols bold

	% Various commands that are useful.  Please add your own.

	\DeclareMathOperator{\Arg}{Arg}
	\DeclareMathOperator{\re}{Re}
	\DeclareMathOperator{\im}{Im}
	\DeclareMathOperator{\Log}{Log}
	\DeclareMathOperator{\Span}{Span}

	\newcommand{\iso}{\cong}						% isometric/congruent
	\newcommand{\ra}{\rightarrow}                   % right arrow
	\newcommand{\Ra}{\Rightarrow}                   % right implies
	\newcommand{\lra}{\longrightarrow}              % long right arrow
	\newcommand{\la}{\leftarrow}                    % left arrow
	\newcommand{\La}{\Leftarrow}                    % left implies
	\newcommand{\lla}{\longleftarrow}               % long left arrow
	\newcommand{\llra}[1]{\stackrel{#1}{\lra}}      % labeled long right arrow
	\newcommand{\we}{\llra{\sim}}                   % weak equivalence
	\newcommand{\cof}{\rightarrowtail}              % cofibration
	\newcommand{\fib}{\twoheadrightarrow}           % fibration
	\newcommand{\inc}{\hookrightarrow}              % inclusion
	\newcommand{\dbra}{\rightrightarrows}           % double arrow for equalizer diagrams
	\newcommand{\eqra}{\llra{\sim}}                 % equivalence/isomorphism

	% \newcommand{\blank}{\underbar{\ \ }}          % An underscore, as in (__)xV
	\newcommand{\blank}{-}                          % A hyphen, as in (-)xV
	\newcommand{\Id}{Id}                            % The identity functor
	\newcommand{\und}{\underline}
	\newcommand{\norm}[1]{\mid \!\!#1 \!\!\mid}             %\norm{x} gives |x|

	% These commands are for the period and comma in the lower right entry of
	% a diagram.  They put the punctuation 2 pts to the right, but make
	% TeX (and hence the diagram package) unaware of the extra width
	% of that entry.
	\newcommand{\period}    {{\makebox[0pt][l]{\hspace{2pt} .}}}
	\newcommand{\comma}     {{\makebox[0pt][l]{\hspace{2pt} ,}}}
	\newcommand{\semicolon} {{\makebox[0pt][l]{\hspace{2pt} ;}}}

	\newcommand{\Cech}{\v{C}ech}
	\newcommand{\scat}{\Delta}
	\newcommand{\assign}{\ra}
	\newcommand{\copr}{\,\amalg\,}
	\newcommand{\ovcat}{\downarrow}
	\newcommand{\pder}[2]{{\frac{\partial #1}{\partial #2}}}
	\newcommand{\del}{\nabla}
	\newcommand{\vectr}[1]{{\mbox{\boldmath $#1$}}}
	\newcommand{\uvectr}[1]{\vectr{\hat #1}}
	\newcommand{\ihat}{\uvectr \imath}
	\newcommand{\jhat}{\uvectr \jmath}
	\newcommand{\khat}{\uvectr k}
	\newcommand{\rhat}{\uvectr r}
	\newcommand{\thhat}{\uvectr \theta}
	\newcommand{\zhat}{\uvectr z}
	\newcommand{\rhohat}{\uvectr \rho}
	\newcommand{\phihat}{\uvectr \phi}
	\newcommand{\grad}{\vectr{\vec\nabla}}
	% \newcommand{\R}{\mathbb{R}}
	\newcommand{\vv}[1]{\vectr{v_{#1}}}
	\newcommand{\crad}{0.1}
	\newcommand{\lline}[1]{\overleftrightarrow{#1}}
	\DeclareMathOperator{\area}{area}
	\DeclareMathOperator{\vol}{vol}
	\newcommand{\ray}[1]{\overset{\rightarrow}{#1}}
	\newcommand{\sr}[2]{???}
	\newcommand{\iihat}{i}
	\newcommand{\jjhat}{j}
	\newcommand{\kkhat}{k}

		
\begin{document}
	\title{Homework 1  -- Math 441 \\ \today}
	\author{Alex Thies \\ \href{mailto:athies@uoregon.edu}{\lowercase{athies$@$uoregon.edu}}}

	\maketitle

	\section*{Section 1.A}
		\subsection*{Problem 1}
		Suppose $a$ and $b$ are real numbers, not both $0$. 
		Find real numbers $c$ and $d$ such that $$1/(a + bi) = c + di.$$

		\begin{proof}[Solution]
		Let $a,b$ be as above, utilizing the complex conjugate we compute the following:
			\begin{align*}
			\frac{1}{a + ib} &= \left( \frac{1}{a + ib} \right) \frac{a - ib}{a - ib}, \\
			&= \frac{a - ib}{a^{2} - i^{2}b^{2}}, \\
			&= \frac{a - ib}{a^{2} + b^{2}}, \\
			&= \frac{a}{a^{2} + b^{2}} + i\left( \frac{-b}{a^{2} + b^{2}} \right).
			\end{align*}
		Notice that we have $c = a/(a^{2} + b^{2})$ and $d = -b/(a^{2} + b^{2})$.
		Further, since $\R$ is a field we know that $c,d \in \R$.
		Thus, we have found real numbers $c$ and $d$ with the properties stated above.
		\end{proof}
		% subsection problem_1 (end)
		\newpage

		\subsection*{Problem 2}
		Show that $$\frac{-1 + i\sqrt{3}}{2}$$ is a cube root of 1.

		\begin{proof}[Solution]
		Let $z \in \C$ such that $z = (-1 + i\sqrt{3})/2$
		We compute the following:
			\begin{align*}
			z^{3} &= \left( \frac{-1 + i\sqrt{3}}{2} \right)^{3}, \\
			&= \left( \frac{-1 + i\sqrt{3}}{2} \right) \left( \frac{-1 + i\sqrt{3}}{2} \right) \left( \frac{-1 + i\sqrt{3}}{2} \right), \\
			&= \left( \frac{1 - 2i\sqrt{3} + 3i^{2}}{4} \right) \left( \frac{-1 + i\sqrt{3}}{2} \right), \\
			&= \left( \frac{(1 - 3) - i(2\sqrt{3})}{4} \right) \left( \frac{-1 + i\sqrt{3}}{2} \right), \\
			&= \left( \frac{ -2 - i(2\sqrt{3})}{4} \right) \left( \frac{-1 + i\sqrt{3}}{2} \right), \\
			&= \left( \frac{ -1 - i\sqrt{3}}{2} \right) \left( \frac{-1 + i\sqrt{3}}{2} \right), \\
			&= \frac{1 - i\sqrt{3} + i\sqrt{3} - 3i^{2}}{4} = \frac{4}{4} = 1.
			\end{align*}
		Since $z^{3} = 1$ we have that $z = \sqrt[3]{1}$, as desired.			
		\end{proof}
		% subsection problem_2 (end)

		\subsection*{Problem 3}
		Find two distinct square roots of $i$.

		\begin{proof}[Solution]
		We will show that two distinct square roots of $i$ are $\pm (1/\sqrt{2})(1 + i)$.

		Recall that the imaginary unit $i = \sqrt{-1}$ is defined as a number whose square is $-1$.
		From this definition, it follows that $i$ could have several square roots, since that there may exist several numbers that square to $-1$; this is particularly notable given that the previous exercise shows that there are several numbers that cube to $1$.
		
		Customarily, we write elements of $\C$ as ordered pairs, and we think of them as points in the complex plane.
		Given this framework, we can ask things about complex numbers -- points in the plane -- like what is its distance from the origin (denote this $r$), and what angle (denote this $\theta$) does it make with the positive real axis?
		Recall further that with these tools for thinking about complex numbers, we can express elements of $\C$ in the form $re^{i \theta}$ or $r(\cos{\theta} + i\sin{\theta})$ where $r$ and $\theta$ are defined as above.
		Since the complex number $i$ is equivalent to the point $(0,1)$ in the complex plane, we know that its distance from the origin is $r = 1$, and that its angle with the positive real axis is $\theta = \pi/2$.
		Hence, let's write $i = e^{i(\pi/2)}$.
		Taking the square root of $i$ in this form yields $\sqrt{i} = \pm e^{i(\pi/4)}$.
		In order to arrive at our final answer, we will transition from the exponential notation ($re^{i \theta}$) to the trigonometric notation ($r(\cos{\theta} + i\sin{\theta})$) and see that:
			\begin{align*}
			\pm\sqrt{i} &= \pm\sqrt{e^{i(\pi/2)}}, \\
			&= \pm \left( e^{i(\pi/2)} \right)^{1/2}, \\
			&= \pm e^{i(\pi/4)}, \\
			&= \pm \left( \cos{(\pi/4)} + i\sin{(\pi/4)} \right), \\
			&= \pm \left( \frac{1}{\sqrt{2}} + i \frac{1}{\sqrt{2}} \right), \\
			&= \frac{\pm 1}{\sqrt{2}} \left( 1 + i \right), \\
			&= \frac{-1 - i}{\sqrt{2}}, \ \frac{1 + i}{\sqrt{2}}.
			\end{align*}
		Thus, we have two distinct square roots of $i$, $z_{1} = \frac{1}{\sqrt{2}}(-1 - i)$, and $z_{2} = \frac{1}{\sqrt{2}}(1 + i)$.
		It is easy to check that these numbers do indeed square to $i$.
		\end{proof}
		% subsection problem_3 (end)

		\subsection*{Problem 4}
		Show that $\alpha + \beta = \beta + \alpha$ for all $\alpha, \beta \in \C$.

		\begin{proof}[Solution]
		Let $\alpha,\beta$ be as above, i.e., $\alpha = (a,b)$ and $\beta = (c,d)$ where $\alpha = a + bi$ and $\beta = c + di$.
		Recall that addition over $\C$ is defined by adding the components of ordered pairs, which are themselves elements of $\R$.
		Since $\R$ is a field, the operations of addition and multiplication over $\R$ are associative and commutative; we utilize these properties in the following computation:
			\begin{align*}
			\alpha + \beta &= (a,b) + (c,d), \\
			&= (a + c, b + d), \\
			&= (c + a, d + b), \\
			&= (c,d) + (a,b), \\
			&= \beta + \alpha.
			\end{align*}
		Hence $\alpha + \beta = \beta + \alpha$ for all $\alpha, \beta \in \C$, as desired.
		\end{proof}
		% subsection problem_4 (end)

		\subsection*{Problem 5}
		Show that $(\alpha + \beta) + \lambda = \alpha + (\beta + \lambda)$ for all $\alpha, \beta, \lambda \in \C$.

		\begin{proof}[Solution]
		Let $\alpha, \beta, \lambda$ be as above, utilizing the same arguments as in Problem 4, we compute the following:
			\begin{align*}
			(\alpha + \beta) + \lambda &= [(a,b) + (c,d)] + (x,y), \\
			&= (a + c, b + d) + (x,y), \\
			&= (a + c + x, b + d + y), \\
			&= (a,b) + (c + x, d + y), \\
			&= (a,b) + [(c,d) + (x,y)], \\
			&= \alpha + (\beta + \lambda).
			\end{align*}
		Hence $(\alpha + \beta) + \lambda = \alpha + (\beta + \lambda)$ for all $\alpha, \beta, \lambda \in \C$, as desired.
		\end{proof}
		% subsection problem_5 (end)

		\subsection*{Problem 6}
		Show that $(\alpha \beta)\lambda = \alpha(\beta \lambda)$ for all $\alpha, \beta, \lambda \in \C$.

		\begin{proof}[Solution]
		Let $\alpha,\beta,\lambda$ be as above, let's write them as $\alpha = (a,b)$, $\beta = (c,d)$, and $\lambda = (x,y)$.
		Recall our definition of multiplication over $\C$, that is, $\alpha \beta = (a,b)(c,d) = (ac - bd, ad + bc)$.
		Utilizing the same reasoning that undergirds Problems 4 and 5, we compute the following:
			\begin{align*}
			(\alpha \beta) \lambda &= [(a,b)(c,d)](x,y), \\
			&= [(ac - bd, \ ad + bc)](x,y), \\
			&= ((ac - bd)x - (ad + bc)y, \ (ac - bd)y + (ad + bc)x), \\
			&= (acx - bdx - ady - bcy, \ acy - bdy + adx + bcx), \\
			&= (acx - ady - bdx - bcy, \ bcx - bdy + adx + acy), \\
			&= (a(cx - dy) - b(dx + cy), \ b(cx - dy) + a (dx + cy)), \\
			&= (a,b)[(cx - dy, \ dx + cy)], \\
			&= (a,b)[(c,d)(x,y)], \\
			&= \alpha (\beta \lambda).
			\end{align*}
		Thus, multiplication over the complex numbers is associative, as desired.
			
		\end{proof}
		% subsection problem_6 (end)

		\subsection*{Problem 7}
		Show that for every $\alpha \in \C$ there exists a unique $\beta \in \C$ such that $\alpha + \beta = 0$.

		\begin{proof}[Solution]
		Let $\alpha \in \C$, write $\alpha = a + bi$, or $(a,b)$ where $a,b \in \R$ and $i^{2} = -1$.
		Since $a,b \in \R$, and $\R$ is a field, there exist unique additive inverses $-a$ and $-b$ for $a$ and $b$, respectively.
		It follows that $\beta = -a + (-b)i = (-a,-b) = - \alpha$ is the unique additive inverse for $\alpha$.
		\end{proof}
		% subsection problem_7 (end)

		\subsection*{Problem 8}
		Show that for every $\alpha \in \C$ with $\alpha \neq 0$, there exists a unique $\beta \in \C$ such that $\alpha \beta = 1$.

		\begin{proof}[Solution]
		Let $\alpha$ be as above, write $\alpha = a + bi$, or $(a,b)$ where $a,b \in \R$ and $i^{2} = -1$.
		We will now compute the real and imaginary components of $\beta$, and then show that $\beta$ must be unique.
			\begin{align*}
			\alpha \beta &= 1, \\
			(a + bi)(c + di) &= 1, \\
			c + di &= \frac{1}{a + bi}, \\
			&= \frac{a - bi}{a^{2} + b^{2}}, \\
			&= \frac{a}{a^{2} + b^{2}} - i\frac{b}{a^{2} + b^{2}}.
			\end{align*}
		Thus, $\beta = (1/a^{2} + b^{2})(a,-b)$ is certainly \textit{a} multiplicative inverse for $\alpha$, but is it unique?
		Suppose that there exists another multiplicative inverse for $\alpha$, call it $\beta'$.
		Since $\beta$ is a multiplicative inverse for $\alpha$, we know that $\alpha \beta = 1$, and for the same reasoning we know $\alpha \beta' = 1$, hence $\alpha \beta = \alpha \beta'$, and because $\alpha \neq 0$, we have that $\beta = \beta'$.
		Thus, $\beta$ is the unique multiplicative inverse for $\alpha$.	
		\end{proof}
		% subsection problem_8 (end)

		\subsection*{Problem 9}
		Show that $\lambda(\alpha + \beta) = \lambda \alpha + \lambda \beta$ for all $\alpha, \beta, \lambda \in \C$.

		\begin{proof}[Solution]
		Let $\alpha,\beta,\lambda$ be as above, and as we have done several times already recall that each of these can be expressed as $a + bi$, or as an ordered pair $(a,b)$.
		We compute the following:
			\begin{align*}
			\lambda(\alpha + \beta) &= (x + iy)[(a + bi) + (c + di)], \\
			&= (x + iy)[(a+c)+ i(b+d)], \\
			&= x(a + c) + xi(b + d) + yi(a + c) - y(b + d), \\
			&= x(a + c) - y(b + d) + i[x(b + d) + y(a + c)], \\
			&= xa + xc - yb - yd + i(xb + xd + ya + yc), \\
			&= xa - yb + xc - yd + i(xb + ya) + i(xd + yc), \\
			&= xa - yb + i(xb + ya) + xc - yd + i(xd + yc), \\
			&= (ax + ayi + bxi + i^{2}by) + (cx + cyi + dxi + i^{2}dy), \\
			&= [(a + bi)(x + yi)] + [(c + di)(x + yi)], \\
			&= \lambda \alpha + \lambda \beta.
			\end{align*}
			
		\end{proof}
		Hence, addition and multiplication have the distributive property over the complex numbers.
		% subsection problem_9 (end)

		\subsection*{Problem 10}
		Find $x \in \R^{4}$ such that $$(4, -3, 1, 7) + 2x = (5, 9, -6, 8).$$

		\begin{proof}[Solution]
		I prefer to write column vectors, so that's what we'll use in the following computation:
		\begin{align*}
		\begin{pmatrix} 4 \\ -3 \\ 1 \\ 7 \end{pmatrix} + 2 \begin{pmatrix} x_{1} \\ x_{2} \\ x_{3} \\ x_{4} \end{pmatrix} &= \begin{pmatrix} 5 \\ 9 \\ -6 \\ 8 \end{pmatrix}, \\
		2 \begin{pmatrix} x_{1} \\ x_{2} \\ x_{3} \\ x_{4} \end{pmatrix} &= \begin{pmatrix} 5 \\ 9 \\ -6 \\ 8 \end{pmatrix} - \begin{pmatrix} 4 \\ -3 \\ 1 \\ 7 \end{pmatrix}, \\
		\begin{pmatrix} x_{1} \\ x_{2} \\ x_{3} \\ x_{4} \end{pmatrix} &= \frac{1}{2}\begin{pmatrix} 1 \\ 12 \\ -7 \\ 1 \end{pmatrix}.
		\end{align*}
		Thus, the desired vector is $\vec{x} = (0.5, 6, -3.5, 0.5)$.
		\end{proof}
		% subsection problem_10 (end)

		\subsection*{Problem 11}
		Explain why there does not exist $\lambda \in \C$ such that $$\lambda(2 - 3i, 5 + 4i, -6 + 7i) = (12 - 5i, 7 + 22i, -32 - 9i).$$

		\begin{proof}[Solution]
		There does not exist a number $\lambda$ for which $2 \lambda = 12$ and $5\lambda = 7$ (the first two real components of the given complex vector), thus the vectors on either side of the equals sign are linearly independent.
		\end{proof}
		% subsection problem_11 (end)

		\subsection*{Problem 12}
		Show that $(x + y) + z = x + (y + z)$ for all $x,y,z \in \F^{n}$.
		% subsection problem_12 (end)

		\subsection*{Problem 13}
		Show that $(ab)x = a(bx)$ for all $x \in \F^{n}$ and $a,b \in \F$.	
		% subsection problem_13 (end)

		\subsection*{Problem 14}
		Show that $1x = x$ for all $x \in \F^{n}$.
		% subsection problem_14 (end)

		\subsection*{Problem 15}
		Show that $\lambda(x + y) = \lambda x + \lambda y$ for all $\lambda \in \F$ and all $x,y \in \F^{n}$.	
		% subsection problem_15 (end)

		\subsection*{Problem 16}
		Show that $(a + b)x = ax + bx$ for all $a,b \in \F$ and all $x \in \F^{n}$.
		% subsection problem_16 (end)
	% section section_1_a (end)
\end{document}