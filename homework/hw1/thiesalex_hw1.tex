%!TEX output_directory = temp
\documentclass[letterpaper, 12pt]{amsart}
	%%%%%%%%%%%%%%%%%%%%%%%%%%%%%%%%%%%%%%%%%%%%%%%%%%%%%%%%%%%%%%%%%%%%%%%%%%%%%%
	%%%%%%%%%%%%%%%%%%%%%%%%%%%% boilerplate packages %%%%%%%%%%%%%%%%%%%%%%%%%%%%
		\usepackage[margin=2in]{geometry}
		\usepackage{amsmath,amssymb,amsthm}
		\usepackage{marvosym}
		\usepackage[mathscr]{euscript}
		\usepackage{enumerate}
		\usepackage{graphicx}
		\usepackage{mathrsfs}
		\usepackage{color}
		\usepackage{hyperref}
		\usepackage{verbatim}
		\usepackage{stmaryrd}

	%%%%%%%%%%%%%%%%%%%%%%%%%%%%%%%%%%%%%%%%%%%%%%%%%%%%%%%%%%%%%%%%%%%%%%%%%%%%%%
	%%%%%%%%%%%%%%%%%%%%%%%%%%%%% rename the abstract %%%%%%%%%%%%%%%%%%%%%%%%%%%%
		\renewcommand{\abstractname}{Assignment}

	%%%%%%%%%%%%%%%%%%%%%%%%%%%%%%%%%%%%%%%%%%%%%%%%%%%%%%%%%%%%%%%%%%%%%%%%%%%%%%
	%%%%%%%%%%%%%%%%%%%%%%%%%%%%%%%%%%%%% sets %%%%%%%%%%%%%%%%%%%%%%%%%%%%%%%%%%%
		\DeclareMathOperator{\N}{\mathbb{N}}				% natural numbers
		\DeclareMathOperator{\Z}{\mathbb{Z}}				% integers
		\DeclareMathOperator{\Zp}{\mathbb{Z}^{+}}			% positive integers
		\DeclareMathOperator{\Q}{\mathbb{Q}}				% rationals
		\DeclareMathOperator{\Qc}{\mathbb{Q}^{c}}			% irrationals
		\DeclareMathOperator{\R}{\mathbb{R}}				% reals
		\DeclareMathOperator{\F}{\mathbb{F}}				% a field
		\DeclareMathOperator{\C}{\mathbb{C}}				% complex numbers
		\DeclareMathOperator{\Cnon}{\mathbb{C}^{\times}}	% nonzero complex numbers
		\DeclareMathOperator{\Pcal}{\mathcal{P}}			% powerset, or set of polynomials
		\DeclareMathOperator{\Ell}{\mathscr{L}}				% set of linear maps, or linear operator

	%%%%%%%%%%%%%%%%%%%%%%%%%%%%%%%%%%%%%%%%%%%%%%%%%%%%%%%%%%%%%%%%%%%%%%%%%%%%%%
	%%%%%%%%%%%%%%%%%%%%%%%%%%%%%% use pretty letters %%%%%%%%%%%%%%%%%%%%%%%%%%%%
		\DeclareMathOperator{\ep}{\varepsilon}				% epsilons
		\DeclareMathOperator{\ph}{\varphi}					% phis

	%%%%%%%%%%%%%%%%%%%%%%%%%%%%%%%%%%%%%%%%%%%%%%%%%%%%%%%%%%%%%%%%%%%%%%%%%%%%%%
	%%%%%%%%%%%%%%%%%%%%%%%%%%%%%%%%%%% algebra %%%%%%%%%%%%%%%%%%%%%%%%%%%%%%%%%%
		\renewcommand{\null}{\text{null }}					% null space
		\DeclareMathOperator{\range}{\text{range }}			% range
		\newcommand{\bmat}[1]{{\mathbf{#1}}}				% bold matrix
		\newcommand{\bvec}[1]{{\vec{\mathbf{#1}}}}			% bold vector
		\DeclareMathOperator{\ind}{\perp\!\!\!\perp}		% perpendicular, orthogonal
		\DeclareMathOperator{\ord}{\text{ord}}				% order of a structure
		\DeclareMathOperator{\Log}{Log}						% logarithm
		\DeclareMathOperator{\Span}{Span}					% span
		\newcommand{\pid}[1]{\langle #1 \rangle}			% bracket notation, used for 
															% ideals or inner products
		\newcommand{\norm}[1]{\mid \!\!#1 \!\!\mid}			%\norm{x} gives |x|

		% fatdot notation
		\makeatletter
			\newcommand*\fatdot{\mathpalette\fatdot@{.5}}
			\newcommand*\fatdot@[2]{\mathbin{\vcenter{\hbox{\scalebox{#2}{$\m@th#1\bullet$}}}}}
		\makeatother

	%%%%%%%%%%%%%%%%%%%%%%%%%%%%%%%%%%%%%%%%%%%%%%%%%%%%%%%%%%%%%%%%%%%%%%%%%%%%%%
	%%%%%%%%%%%%%%%%%%%%%%%%%%% probability & statistics %%%%%%%%%%%%%%%%%%%%%%%%%
		\renewcommand{\Pr}{\mathbb{P}}						% probability
		\DeclareMathOperator{\E}{\mathbb{E}}				% expectation
		\DeclareMathOperator{\var}{\rm Var}					% variance
		\DeclareMathOperator{\sd}{\rm SD}					% standard deviation
		\DeclareMathOperator{\cov}{\rm Cov}					% covariance
		\DeclareMathOperator{\SE}{\rm SE}					% standard error
		\DeclareMathOperator{\ssreg}{{\rm SS}_{{\rm Reg}}}	% sum of squared regression
		\DeclareMathOperator{\ssr}{{\rm SS}_{{\rm Res}}}	% sum of squared residuals
		\DeclareMathOperator{\sst}{{\rm SS}_{{\rm Tot}}}	% total sum of squares

	%%%%%%%%%%%%%%%%%%%%%%%%%%%%%%%%%%%%%%%%%%%%%%%%%%%%%%%%%%%%%%%%%%%%%%%%%%%%%%
	%%%%%%%%%%%%%%%%%%%%%%%%%%%%%%% number theory %%%%%%%%%%%%%%%%%%%%%%%%%%%%%%%%
		\renewcommand{\mod}[1]{\ (\mathrm{mod}\ #1)}		% congruences

	%%%%%%%%%%%%%%%%%%%%%%%%%%%%%%%%%%%%%%%%%%%%%%%%%%%%%%%%%%%%%%%%%%%%%%%%%%%%%%
	%%%%%%%%%%%%%%%%%%%%%%%%%%%% theorem environments %%%%%%%%%%%%%%%%%%%%%%%%%%%%
		% Some theorem-like environments, all numbered together starting at 1
		% in each section.

		\newtheorem{thm}{Theorem}[section]					% The default \theoremstyle is 
		\newtheorem{defn}[thm]{Definition}					% bold headings and italic body text.
		\newtheorem{prop}[thm]{Proposition}
		\newtheorem{claim}[thm]{Claim}
		\newtheorem{cor}[thm]{Corollary}
		\newtheorem{lemma}[thm]{Lemma}

		\theoremstyle{definition}  							% Bold headings and Roman body text.
		\newtheorem{example}[thm]{Example}
		\newtheorem{examples}[thm]{Examples}
		\newtheorem{exercise}[thm]{Exercise}
		\newtheorem{note}[thm]{Note}
		\newtheorem{remark}[thm]{Remark}
		\newtheorem{remarks}[thm]{Remarks}
		\newtheorem{discussion}[thm]{Discussion}

		\newcommand{\dfn}{\textbf} 							% Make defined words bold.
		\newcommand{\mdfn}[1]{\dfn{\mathversion{bold}#1}} 	% Even make math symbols bold

	%%%%%%%%%%%%%%%%%%%%%%%%%%%%%%%%%%%%%%%%%%%%%%%%%%%%%%%%%%%%%%%%%%%%%%%%%%%%%%
	%%%%%%%%%%%%%%%%%%%%%%%%%%%%%%% complex numbers %%%%%%%%%%%%%%%%%%%%%%%%%%%%%%
		\DeclareMathOperator{\Arg}{Arg}						% argument of z \in \C
		\DeclareMathOperator{\re}{Re}						% real component
		\DeclareMathOperator{\im}{Im}						% imaginary component

	%%%%%%%%%%%%%%%%%%%%%%%%%%%%%%%%%%%%%%%%%%%%%%%%%%%%%%%%%%%%%%%%%%%%%%%%%%%%%%
	%%%%%%%%%%%%%%%%%%%%%%%%%%%%%%% various symbols %%%%%%%%%%%%%%%%%%%%%%%%%%%%%%
		\newcommand{\iso}{\cong}						% isometric/congruent
		\newcommand{\ra}{\rightarrow}                   % right arrow
		\newcommand{\Ra}{\Rightarrow}                   % right implies
		\newcommand{\lra}{\longrightarrow}              % long right arrow
		\newcommand{\la}{\leftarrow}                    % left arrow
		\newcommand{\La}{\Leftarrow}                    % left implies
		\newcommand{\lla}{\longleftarrow}               % long left arrow
		\newcommand{\eqra}{\llra{\sim}}                 % equivalence/isomorphism
		\newcommand{\blank}{\underbar{\ \ }}          	% An underscore, as in (__)xV
		% \newcommand{\blank}{-}                          % A hyphen, as in (-)xV
		\newcommand{\Id}{Id}                            % The identity functor
		\newcommand{\und}{\underline}
		\newcommand{\del}{\nabla}						% gradient vector

		\raggedbottom		
\begin{document}
	\title{Homework 1  -- Math 441 \\ A\lowercase{pril 18, 2018}}
	\author{Alex Thies \\ \href{mailto:athies@uoregon.edu}{\lowercase{athies$@$uoregon.edu}}}

	\begin{abstract}
	The following exercises are assigned from \textit{Linear Algebra Done Right}, 3rd Edition, by Sheldon Axler. 
			\begin{tabular}{rl}
				& 1.A - 3, 8, 13; \\
				& 1.B - 2, 6; \\
				& 1.C - 3, 8, 10, 12, 21, 23.
			\end{tabular}
	\end{abstract}
	\maketitle

	\section*{Section 1.A}
		\subsection*{Problem 3}
		Find two distinct square roots of $i$.

		\begin{proof}[Solution]
		We will show that two distinct square roots of $i$ are $\pm (1/\sqrt{2})(1 + i)$.

		Recall that the imaginary unit $i = \sqrt{-1}$ is defined as a number whose square is $-1$.
		From this definition, it follows that $i$ could have several square roots, since that there may exist several numbers that square to $-1$; this is particularly notable given that the previous exercise shows that there are several numbers that cube to $1$.
		
		Customarily, we write elements of $\C$ as ordered pairs, and we think of them as points in the complex plane.
		Given this framework, we can ask things about complex numbers -- points in the plane -- like what is its distance from the origin (denote this $r$), and what angle (denote this $\theta$) does it make with the positive real axis?
		Recall further that with these tools for thinking about complex numbers, we can express elements of $\C$ in the form $re^{i \theta}$ or $r(\cos{\theta} + i\sin{\theta})$ where $r$ and $\theta$ are defined as above.
		Since the complex number $i$ is equivalent to the point $(0,1)$ in the complex plane, we know that its distance from the origin is $r = 1$, and that its angle with the positive real axis is $\theta = \pi/2$.
		Hence, let's write $i = e^{i(\pi/2)}$.
		Taking the square root of $i$ in this form yields $\sqrt{i} = \pm e^{i(\pi/4)}$.
		In order to arrive at our final answer, we will transition from the exponential notation ($re^{i \theta}$) to the trigonometric notation ($r(\cos{\theta} + i\sin{\theta})$) and see that:
			\begin{align*}
			\pm\sqrt{i} &= \pm\sqrt{e^{i(\pi/2)}}, \\
			&= \pm \left( e^{i(\pi/2)} \right)^{1/2}, \\
			&= \pm e^{i(\pi/4)}, \\
			&= \pm \left( \cos{(\pi/4)} + i\sin{(\pi/4)} \right), \\
			&= \pm \left( \frac{1}{\sqrt{2}} + i \frac{1}{\sqrt{2}} \right), \\
			&= \frac{\pm 1}{\sqrt{2}} \left( 1 + i \right), \\
			&= \frac{-1 - i}{\sqrt{2}}, \ \frac{1 + i}{\sqrt{2}}.
			\end{align*}
		Thus, we have two distinct square roots of $i$, $z_{1} = \frac{1}{\sqrt{2}}(-1 - i)$, and $z_{2} = \frac{1}{\sqrt{2}}(1 + i)$.
		It is easy to check that these numbers do indeed square to $i$.
		\end{proof}
		% subsection problem_3 (end)

		\subsection*{Problem 8}
		Show that for every $\alpha \in \C$ with $\alpha \neq 0$, there exists a unique $\beta \in \C$ such that $\alpha \beta = 1$.

		\begin{proof}[Solution]
		Let $\alpha$ be as above, write $\alpha = a + bi$, or $(a,b)$ where $a,b \in \R$ and $i^{2} = -1$.
		We will now compute the real and imaginary components of $\beta$, and then show that $\beta$ must be unique.
			\begin{align*}
			\alpha \beta &= 1, \\
			(a + bi)(c + di) &= 1, \\
			c + di &= \frac{1}{a + bi},
			\end{align*}
			\begin{align*}
			&= \frac{a - bi}{a^{2} + b^{2}}, \\
			&= \frac{a}{a^{2} + b^{2}} - i\frac{b}{a^{2} + b^{2}}.
			\end{align*}
		Thus, $\beta = (1/a^{2} + b^{2})(a,-b)$ is certainly \textit{a} multiplicative inverse for $\alpha$, but is it unique?
		Suppose that there exists another multiplicative inverse for $\alpha$, call it $\beta'$.
		Since $\beta$ is a multiplicative inverse for $\alpha$, we know that $\alpha \beta = 1$, and for the same reasoning we know $\alpha \beta' = 1$, hence $\alpha \beta = \alpha \beta'$, and because $\alpha \neq 0$, we have that $\beta = \beta'$.
		Thus, $\beta$ is the unique multiplicative inverse for $\alpha$.	
		\end{proof}
		% subsection problem_8 (end)

		\subsection*{Problem 13}
		Show that $(ab)x = a(bx)$ for all $x \in \F^{n}$ and $a,b \in \F$.

		\begin{proof}[Solution]
		Let $a,b,x$ be as above, we compute the following:
		\begin{align*}
		(ab)x &= (ab) \begin{pmatrix} x_{1} \\ x_{2} \\ \vdots \\ x_{n} \end{pmatrix}, \\
		&= \begin{pmatrix} (ab)x_{1} \\ (ab)x_{2} \\ \vdots \\ (ab)x_{n} \end{pmatrix}, \\
		&= \begin{pmatrix} a(bx_{1}) \\ a(bx_{2}) \\ \vdots \\ a(bx_{n}) \end{pmatrix}, \\		
		&= a\begin{pmatrix} bx_{1} \\ bx_{2} \\ \vdots \\ bx_{n} \end{pmatrix}, \\
		&= a (bx).
		\end{align*}
		Hence, $(ab)x = a(bx)$ for all $x \in \F^{n}$ and $a,b \in \F$.
		\end{proof}
		% subsection problem_13 (end)
	% section section_1_a (end)

	\section*{Section 1.B}
		\subsection*{Problem 2}
		Suppose $a \in \F$ $v \in V$, and $av = 0$.
		Prove that $a = 0$ or $v = 0$.

		\begin{proof}
		Let $a,v$ be as above, we proceed by cases.
		
		Case 1: Suppose $a \neq 0$, then we can divide $a$ wherever we see it.
		Then since $av = 0$, we can divide by $a$ and find that $v = 0$; and we're done.

		Case 2: Suppose $v \neq 0$, then it has a unique multiplicative inverse $v^{-1}$.
		Then since $av = 0$, we have $avv^{-1} = 0v^{-1} \iff a = 0$; and we're done again.

		So, since the product $av = 0$, we know that one of the factors from that product must also be 0.
		Behind the scenes, this is because $a$ and the components of $v$ are elements of a field, so none of the numbers with which we are working are zero divisors, otherwise the zero product property does not hold.
		\end{proof}
		% subsection problem_2 (end)

		\subsection*{Problem 6}
		Let $\infty$ and $-\infty$ denote two distinct objects, neither of which is in $\R$.
		Define an addition and scalar multiplication on $\R \cup \{\infty\} \cup \{-\infty\}$ as you could guess from the notation.
		Specifically, the sum and product of two real numbers is as usual, and for $t \in \R$ define \[ t \cdot \infty = \begin{cases} -\infty & \text{if $t < 0$,} \\ 0 & \text{if $t = 0$,} \\ \infty & \text{if $t > 0$,} \end{cases} \hspace{1cm} t \cdot (-\infty) = \begin{cases} \infty & \text{if $t < 0$,} \\ 0 & \text{if $t = 0$,} \\ -\infty & \text{if $t > 0$,} \end{cases} \]
		\[ t + \infty = \infty + t = \infty, \hspace{1.5cm} t + (-\infty) = (-\infty) + t = (-\infty), \]
		\[ \infty + \infty = \infty, \hspace{.75cm} (-\infty) + (-\infty) = -\infty \hspace{.75cm} \infty + (-\infty) = 0. \]
		Is $\R \cup \{\infty\} \cup \{-\infty\}$ a vector space over $\R$?
		Explain.

		\begin{proof}[Solution]
		Let $W = \R \cup \{ \infty \} \cup \{ -\infty \}$, we claim that $W$ is not a vector space over $\R$.
		The definition of a vector space requires that it have an additive identity, and further we have a theorem that tells us this additive identity is unique. 
		From the definition above we see that $\infty$ has two additive identities, namely 0 and itself, hence $W$ cannot be a vector space.
		\end{proof}
		% subsection problem_6 (end)
	% section section_1_b (end)

	\section*{Section 1.C}	
		\subsection*{Problem 3}
		Show that the set of differentiable real-valued functions $f$ on the interval $(-􏰈4, 4)$ such that $f'(-1) = 3f(2)$ is a subspace of $\R^{(-4,4)}$.

		\begin{proof}[Solution]
		Let $W$ be the above described set of functions, in order to show that $W$ is a subspace of $\R^{(-4,4)}$, it will suffice to show that $0 \in W$, and that $W$ is closed under addition and scalar multiplication.
		Let $f,g \in W$, so we can write $f'(-1) = 3f(2)$ and $g'(-1) = 3g(2)$.
		Therefore, we have $f'(-1) + g'(-1) = 3(f(2) + g(2))$, and by the linearity of the derivative, we also have $(f(-1) + g(-1))' = 3(f(2) + g(2))$.
		Hence, $(f + g)' \in W$ and $W$ is closed under addition; it remains to show that $W$ is closed under scalar multiplication.
		\end{proof}
		% subsection problem_3 (end)

		\subsection*{Problem 8}
		Give an example of a nonempty subset $U$ of $\R^{2}$ such that $U$ is closed under scalar multiplication, but $U$ is not a subspace of $\R^{2}$.

		\begin{proof}[Solution]
		Consider $U = \{ (x_{1},x_{2}) : x_{1},x_{2} \in \R \ \text{and} \ |x_{1}| = |x_{2}| \}$.
		Clearly $U \subset \R^{2}$ and $U$ is closed under scalar multiplication, but we claim that $U$ is not closed under addition, and thus not a subspace of $\R^{2}$.
		Consider $\vec{u} = (5,-5)$ and $\vec{v} = (7,7)$, both of which are elements of $U$.
		Notice that $\vec{u} + \vec{v} = (5,-5) + (7,7) = (12,2)$.
		But $|12| \neq |2|$, so $(\vec{u} + \vec{v}) \notin U$, thus $U$ is not closed under addition (as we claimed).
		Hence $U$ is a subset of $\R^{2}$ that is closed under multiplication, but not a subspace of $\R^{2}$, as we aimed to show.
		\end{proof}	
		% subsection problem_8 (end)

		\subsection*{Problem 10}
		Suppose $U_{1}$ and $U_{2}$ are subspaces of $V$. 
		Prove that the intersection $U_{1} \cap U_{2}$ is a subspace of $V$.

		\begin{proof}
		The intersection $U_{1} \cap U_{2}$ is defined to be the set of elements which are contained in $U_{1}$ and $U_{2}$, both of which we assume to be subspaces of $V$.
		It should be clear that since each element of the intersection $U_{1} \cap U_{2}$ is also an element of the subspaces $U_{1}$ and $U_{2}$ by assumption, that the intersection $U_{1} \cap U_{2}$ is also a subspace of $V$.
		\end{proof}
		% subsection problem_10 (end)

		\subsection*{Problem 12}
		Prove that the union of two subspaces of $V$ is a subspace of $V$ if and only if one of the subspaces is contained in the other.

		\begin{proof}
		\end{proof}		
		% subsection problem_12 (end)
	% section section_1_c (end)
\end{document}