%!TEX output_directory = temp
\documentclass[letterpaper, 12pt]{amsart}
	%%%%%%%%%%%%%%%%%%%%%%%%%%%%%%%%%%%%%%%%%%%%%%%%%%%%%%%%%%%%%%%%%%%%%%%%%%%%%%
	%%%%%%%%%%%%%%%%%%%%%%%%%%%% boilerplate packages %%%%%%%%%%%%%%%%%%%%%%%%%%%%
	\usepackage{amsmath,amssymb,amsthm}
	\usepackage[mathscr]{euscript}
	\usepackage{enumerate}
	\usepackage{graphicx}
	\usepackage{mathrsfs}
	\usepackage{color}
	\usepackage{hyperref}
	\usepackage{verbatim}
	\usepackage{stmaryrd}
	\usepackage[margin=1.25in]{geometry}

	%%%%%%%%%%%%%%%%%%%%%%%%%%%%%%%%%%%%%%%%%%%%%%%%%%%%%%%%%%%%%%%%%%%%%%%%%%%%%%
	%%%%%%%%%%%%%%%%%%%%%%%%%%%%% rename the abstract %%%%%%%%%%%%%%%%%%%%%%%%%%%%
	% \renewcommand{\abstractname}{Introduction}

	%%%%%%%%%%%%%%%%%%%%%%%%%%%%%%%%%%%%%%%%%%%%%%%%%%%%%%%%%%%%%%%%%%%%%%%%%%%%%%
	%%%%%%%%%%%%%%%%%%%%%%%%%%%%%%%%%%%%% sets %%%%%%%%%%%%%%%%%%%%%%%%%%%%%%%%%%%
		%% sets 
		\DeclareMathOperator{\N}{\mathbb{N}}
		\DeclareMathOperator{\Z}{\mathbb{Z}}
		\DeclareMathOperator{\Zp}{\mathbb{Z}^{+}}
		\DeclareMathOperator{\Q}{\mathbb{Q}}
		\DeclareMathOperator{\Qp}{\mathbb{Q}^{+}}
		\DeclareMathOperator{\Qc}{\mathbb{Q}^{c}}
		\DeclareMathOperator{\R}{\mathbb{R}}
		\DeclareMathOperator{\F}{\mathbb{F}}
		\DeclareMathOperator{\Rp}{\mathbb{R}^{+}}
		\DeclareMathOperator{\C}{\mathbb{C}}
		\DeclareMathOperator{\Cnon}{\mathbb{C}^{\times}}
		%% powerset of a set
		\DeclareMathOperator{\pset}{\mathcal{P}}
		%% set of continuous functions in a certain variable
		\DeclareMathOperator{\cont}{\mathscr{C}}
		%% set of functions in a certain variable
		\DeclareMathOperator{\func}{\mathscr{F}}
		
	%%%%%%%%%%%%%%%%%%%%%%%%%%%%%%%%%%%%%%%%%%%%%%%%%%%%%%%%%%%%%%%%%%%%%%%%%%%%%%
	%%%%%%%%%%%%%%%%%%%%%%%%%%%%%%%% linear algebra %%%%%%%%%%%%%%%%%%%%%%%%%%%%%%
		%% linear span
		\DeclareMathOperator{\Ell}{\mathscr{L}}
		%% bold vectors with arrows, and bold matrices
		\newcommand{\bmat}[1]{{\mathbf{#1}}}
		\newcommand{\bvec}[1]{{\vec{\mathbf{#1}}}}
		%% independent vectors/matrices
		\DeclareMathOperator{\ind}{\perp\!\!\!\perp}
		%% order
		\DeclareMathOperator{\ord}{\text{ord}}

	%%%%%%%%%%%%%%%%%%%%%%%%%%%%%%%%%%%%%%%%%%%%%%%%%%%%%%%%%%%%%%%%%%%%%%%%%%%%%%
	%%%%%%%%%%%%%%%%%%%%%%%%%%% probability & statistics %%%%%%%%%%%%%%%%%%%%%%%%%
		%% probability, expectation, variance, etc.
		\renewcommand{\Pr}{\mathbb{P}}
		\DeclareMathOperator{\E}{\mathbb{E}}
		\DeclareMathOperator{\var}{\rm Var}
		\DeclareMathOperator{\sd}{\rm SD}
		\DeclareMathOperator{\cov}{\rm Cov}
		\DeclareMathOperator{\SE}{\rm SE}
		\DeclareMathOperator{\ssreg}{{\rm SS}_{{\rm Reg}}}
		\DeclareMathOperator{\ssr}{{\rm SS}_{{\rm Res}}}
		\DeclareMathOperator{\sst}{{\rm SS}_{{\rm Tot}}}

	%%%%%%%%%%%%%%%%%%%%%%%%%%%%%%%%%%%%%%%%%%%%%%%%%%%%%%%%%%%%%%%%%%%%%%%%%%%%%%
	%%%%%%%%%%%%%%%%%%%%%%%%%%%%%%%% congruences %%%%%%%%%%%%%%%%%%%%%%%%%%%%%%%%%
		\renewcommand{\mod}[1]{\ (\mathrm{mod}\ #1)}

	%%%%%%%%%%%%%%%%%%%%%%%%%%%%%%%%%%%%%%%%%%%%%%%%%%%%%%%%%%%%%%%%%%%%%%%%%%%%%%
	%%%%%%%%%%%%%%%%%%%%%%%%%%%%%% bracket notation %%%%%%%%%%%%%%%%%%%%%%%%%%%%%%
		% I first used this for principal ideals, that is why the abbreviation is pid
		\newcommand{\pid}[1]{\langle #1 \rangle}

	%%%%%%%%%%%%%%%%%%%%%%%%%%%%%%%%%%%%%%%%%%%%%%%%%%%%%%%%%%%%%%%%%%%%%%%%%%%%%%
	%%%%%%%%%%%%%%%%%%%%%%%%%%%%%%% fatdot notation %%%%%%%%%%%%%%%%%%%%%%%%%%%%%%
		\makeatletter
			\newcommand*\fatdot{\mathpalette\fatdot@{.5}}
			\newcommand*\fatdot@[2]{\mathbin{\vcenter{\hbox{\scalebox{#2}{$\m@th#1\bullet$}}}}}
		\makeatother

	%%%%%%%%%%%%%%%%%%%%%%%%%%%%%%%%%%%%%%%%%%%%%%%%%%%%%%%%%%%%%%%%%%%%%%%%%%%%%%
	%%%%%%%%%%%%%%%%%%%%%%%%%%%%%% use pretty letters %%%%%%%%%%%%%%%%%%%%%%%%%%%%
		\DeclareMathOperator{\ep}{\varepsilon}
		\DeclareMathOperator{\ph}{\varphi}

	%%%%%%%%%%%%%%%%%%%%%%%%%%%%%%%%%%%%%%%%%%%%%%%%%%%%%%%%%%%%%%%%%%%%%%%%%%%%%%
	%%%%%%%%%%%%%%%%%%%%%%%%%%% stolen from Jeske/Dugger %%%%%%%%%%%%%%%%%%%%%%%%%
	% Some theorem-like environments, all numbered together starting at 1
	% in each section.

	% The default \theoremstyle is bold headings and italic body text.
	\newtheorem{thm}{Theorem}[section]
	\newtheorem{defn}[thm]{Definition}
	\newtheorem{prop}[thm]{Proposition}
	\newtheorem{claim}[thm]{Claim}
	\newtheorem{cor}[thm]{Corollary}
	\newtheorem{lemma}[thm]{Lemma}

	\theoremstyle{definition}  % Bold headings and Roman body text.
	\newtheorem{example}[thm]{Example}
	\newtheorem{examples}[thm]{Examples}
	\newtheorem{exercise}[thm]{Exercise}
	\newtheorem{note}[thm]{Note}
	\newtheorem{remark}[thm]{Remark}
	\newtheorem{remarks}[thm]{Remarks}
	\newtheorem{discussion}[thm]{Discussion}

	\newcommand{\dfn}{\textbf} % Make defined words bold.
	\newcommand{\mdfn}[1]{\dfn{\mathversion{bold}#1}} % Even make math symbols bold

	% Various commands that are useful.  Please add your own.

	\DeclareMathOperator{\Arg}{Arg}
	\DeclareMathOperator{\re}{Re}
	\DeclareMathOperator{\im}{Im}
	\DeclareMathOperator{\Log}{Log}
	\DeclareMathOperator{\Span}{Span}

	\newcommand{\iso}{\cong}						% isometric/congruent
	\newcommand{\ra}{\rightarrow}                   % right arrow
	\newcommand{\Ra}{\Rightarrow}                   % right implies
	\newcommand{\lra}{\longrightarrow}              % long right arrow
	\newcommand{\la}{\leftarrow}                    % left arrow
	\newcommand{\La}{\Leftarrow}                    % left implies
	\newcommand{\lla}{\longleftarrow}               % long left arrow
	\newcommand{\llra}[1]{\stackrel{#1}{\lra}}      % labeled long right arrow
	\newcommand{\we}{\llra{\sim}}                   % weak equivalence
	\newcommand{\cof}{\rightarrowtail}              % cofibration
	\newcommand{\fib}{\twoheadrightarrow}           % fibration
	\newcommand{\inc}{\hookrightarrow}              % inclusion
	\newcommand{\dbra}{\rightrightarrows}           % double arrow for equalizer diagrams
	\newcommand{\eqra}{\llra{\sim}}                 % equivalence/isomorphism

	% \newcommand{\blank}{\underbar{\ \ }}          % An underscore, as in (__)xV
	\newcommand{\blank}{-}                          % A hyphen, as in (-)xV
	\newcommand{\Id}{Id}                            % The identity functor
	\newcommand{\und}{\underline}
	\newcommand{\norm}[1]{\mid \!\!#1 \!\!\mid}             %\norm{x} gives |x|

	% These commands are for the period and comma in the lower right entry of
	% a diagram.  They put the punctuation 2 pts to the right, but make
	% TeX (and hence the diagram package) unaware of the extra width
	% of that entry.
	\newcommand{\period}    {{\makebox[0pt][l]{\hspace{2pt} .}}}
	\newcommand{\comma}     {{\makebox[0pt][l]{\hspace{2pt} ,}}}
	\newcommand{\semicolon} {{\makebox[0pt][l]{\hspace{2pt} ;}}}

	\newcommand{\Cech}{\v{C}ech}
	\newcommand{\scat}{\Delta}
	\newcommand{\assign}{\ra}
	\newcommand{\copr}{\,\amalg\,}
	\newcommand{\ovcat}{\downarrow}
	\newcommand{\pder}[2]{{\frac{\partial #1}{\partial #2}}}
	\newcommand{\del}{\nabla}
	\newcommand{\vectr}[1]{{\mbox{\boldmath $#1$}}}
	\newcommand{\uvectr}[1]{\vectr{\hat #1}}
	\newcommand{\ihat}{\uvectr \imath}
	\newcommand{\jhat}{\uvectr \jmath}
	\newcommand{\khat}{\uvectr k}
	\newcommand{\rhat}{\uvectr r}
	\newcommand{\thhat}{\uvectr \theta}
	\newcommand{\zhat}{\uvectr z}
	\newcommand{\rhohat}{\uvectr \rho}
	\newcommand{\phihat}{\uvectr \phi}
	\newcommand{\grad}{\vectr{\vec\nabla}}
	% \newcommand{\R}{\mathbb{R}}
	\newcommand{\vv}[1]{\vectr{v_{#1}}}
	\newcommand{\crad}{0.1}
	\newcommand{\lline}[1]{\overleftrightarrow{#1}}
	\DeclareMathOperator{\area}{area}
	\DeclareMathOperator{\vol}{vol}
	\newcommand{\ray}[1]{\overset{\rightarrow}{#1}}
	\newcommand{\sr}[2]{???}
	\newcommand{\iihat}{i}
	\newcommand{\jjhat}{j}
	\newcommand{\kkhat}{k}

		
\begin{document}
	\title{Homework 1  -- Math 441 \\ \today}
	\author{Alex Thies \\ \href{mailto:athies@uoregon.edu}{\lowercase{athies$@$uoregon.edu}}}

	\maketitle

	Assignment: 1.A - 3,8,13; 1.B - 2,6; 1.C - 3,8,10,12,21,23

	\section*{Section 1.A}
		\subsection*{Problem 3}
		Find two distinct square roots of $i$.

		\begin{proof}[Solution]
		We will show that two distinct square roots of $i$ are $\pm (1/\sqrt{2})(1 + i)$.

		Recall that the imaginary unit $i = \sqrt{-1}$ is defined as a number whose square is $-1$.
		From this definition, it follows that $i$ could have several square roots, since that there may exist several numbers that square to $-1$; this is particularly notable given that the previous exercise shows that there are several numbers that cube to $1$.
		
		Customarily, we write elements of $\C$ as ordered pairs, and we think of them as points in the complex plane.
		Given this framework, we can ask things about complex numbers -- points in the plane -- like what is its distance from the origin (denote this $r$), and what angle (denote this $\theta$) does it make with the positive real axis?
		Recall further that with these tools for thinking about complex numbers, we can express elements of $\C$ in the form $re^{i \theta}$ or $r(\cos{\theta} + i\sin{\theta})$ where $r$ and $\theta$ are defined as above.
		Since the complex number $i$ is equivalent to the point $(0,1)$ in the complex plane, we know that its distance from the origin is $r = 1$, and that its angle with the positive real axis is $\theta = \pi/2$.
		Hence, let's write $i = e^{i(\pi/2)}$.
		Taking the square root of $i$ in this form yields $\sqrt{i} = \pm e^{i(\pi/4)}$.
		In order to arrive at our final answer, we will transition from the exponential notation ($re^{i \theta}$) to the trigonometric notation ($r(\cos{\theta} + i\sin{\theta})$) and see that:
			\begin{align*}
			\pm\sqrt{i} &= \pm\sqrt{e^{i(\pi/2)}}, \\
			&= \pm \left( e^{i(\pi/2)} \right)^{1/2}, \\
			&= \pm e^{i(\pi/4)}, \\
			&= \pm \left( \cos{(\pi/4)} + i\sin{(\pi/4)} \right), \\
			&= \pm \left( \frac{1}{\sqrt{2}} + i \frac{1}{\sqrt{2}} \right), \\
			&= \frac{\pm 1}{\sqrt{2}} \left( 1 + i \right), \\
			&= \frac{-1 - i}{\sqrt{2}}, \ \frac{1 + i}{\sqrt{2}}.
			\end{align*}
		Thus, we have two distinct square roots of $i$, $z_{1} = \frac{1}{\sqrt{2}}(-1 - i)$, and $z_{2} = \frac{1}{\sqrt{2}}(1 + i)$.
		It is easy to check that these numbers do indeed square to $i$.
		\end{proof}
		% subsection problem_3 (end)

		\subsection*{Problem 8}
		Show that for every $\alpha \in \C$ with $\alpha \neq 0$, there exists a unique $\beta \in \C$ such that $\alpha \beta = 1$.

		\begin{proof}[Solution]
		Let $\alpha$ be as above, write $\alpha = a + bi$, or $(a,b)$ where $a,b \in \R$ and $i^{2} = -1$.
		We will now compute the real and imaginary components of $\beta$, and then show that $\beta$ must be unique.
			\begin{align*}
			\alpha \beta &= 1, \\
			(a + bi)(c + di) &= 1, \\
			c + di &= \frac{1}{a + bi}, \\
			&= \frac{a - bi}{a^{2} + b^{2}}, \\
			&= \frac{a}{a^{2} + b^{2}} - i\frac{b}{a^{2} + b^{2}}.
			\end{align*}
		Thus, $\beta = (1/a^{2} + b^{2})(a,-b)$ is certainly \textit{a} multiplicative inverse for $\alpha$, but is it unique?
		Suppose that there exists another multiplicative inverse for $\alpha$, call it $\beta'$.
		Since $\beta$ is a multiplicative inverse for $\alpha$, we know that $\alpha \beta = 1$, and for the same reasoning we know $\alpha \beta' = 1$, hence $\alpha \beta = \alpha \beta'$, and because $\alpha \neq 0$, we have that $\beta = \beta'$.
		Thus, $\beta$ is the unique multiplicative inverse for $\alpha$.	
		\end{proof}
		% subsection problem_8 (end)

		\subsection*{Problem 13}
		Show that $(ab)x = a(bx)$ for all $x \in \F^{n}$ and $a,b \in \F$.

		\begin{proof}[Solution]
		Let $a,b,x$ be as above, we compute the following:
		\begin{align*}
		(ab)x &= (ab) \begin{pmatrix} x_{1} \\ x_{2} \\ \vdots \\ x_{n} \end{pmatrix}, \\
		&= \begin{pmatrix} (ab)x_{1} \\ (ab)x_{2} \\ \vdots \\ (ab)x_{n} \end{pmatrix}, \\
		&= \begin{pmatrix} a(bx_{1}) \\ a(bx_{2}) \\ \vdots \\ a(bx_{n}) \end{pmatrix}, \\
		&= \hspace{8mm} \vdots
		\end{align*}
		\pagebreak
		
		\begin{align*}
		(ab)x &= \hspace{6mm} \dots \\
		&= a\begin{pmatrix} bx_{1} \\ bx_{2} \\ \vdots \\ bx_{n} \end{pmatrix}, \\
		&= a (bx).
		\end{align*}
		Hence, $(ab)x = a(bx)$ for all $x \in \F^{n}$ and $a,b \in \F$.
		\end{proof}
		% subsection problem_13 (end)
	% section section_1_a (end)

	\section*{Section 1.B}
		\subsection*{Problem 2}
		Suppose $a \in \F$ $v \in V$, and $av = 0$.
		Prove that $a = 0$ or $v = 0$.

		\begin{proof}
		Let $a,v$ be as above, we proceed by cases.
		
		Case 1: Suppose $a \neq 0$, then we can divide $a$ wherever we see it.
		Then since $av = 0$, we can divide by $a$ and find that $v = 0$; and we're done.

		Case 2: Suppose $v \neq 0$, then it has a unique multiplicative inverse $v^{-1}$.
		Then since $av = 0$, we have $avv^{-1} = 0v^{-1} \iff a = 0$; and we're done again.

		So, since the product $av = 0$, we know that one of the factors from that product must also be 0.
		Behind the scenes, this is because $a$ and the components of $v$ are elements of a field, so there are no zero divisors, otherwise the zero product property does not hold.
		\end{proof}
		% subsection problem_2 (end)

		\subsection*{Problem 6}
		Let $\infty$ and $-\infty$ denote two distinct objects, neither of which is in $\R$.
		Define an addition and scalar multiplication on $\R \cup \{\infty\} \cup \{-\infty\}$ as you could guess from the notation.
		Specifically, the sum and product of two real numbers is as usual, and for $t \in \R$ define \[ t \cdot \infty = \begin{cases} -\infty & \text{if $t < 0$,} \\ 0 & \text{if $t = 0$,} \\ \infty & \text{if $t > 0$,} \end{cases} \hspace{1cm} t \cdot (-\infty) = \begin{cases} \infty & \text{if $t < 0$,} \\ 0 & \text{if $t = 0$,} \\ -\infty & \text{if $t > 0$,} \end{cases} \]
		\[ t + \infty = \infty + t = \infty, \hspace{1.5cm} t + (-\infty) = (-\infty) + t = (-\infty), \]
		\[ \infty + \infty = \infty, \hspace{.75cm} (-\infty) + (-\infty) = -\infty \hspace{.75cm} \infty + (-\infty) = 0. \]
		Is $\R \cup \{\infty\} \cup \{-\infty\}$ a vector space over $\R$?
		Explain.

		\begin{proof}[Solution]
		Let $W = \R \cup \{ \infty \} \cup \{ -\infty \}$, we claim that $W$ is not a vector space over $\R$.
		The definition of a vector space requires that it have a unique additive identity, from the definition above we see that $\infty$ has two additive identities, namely 0 and itself, hence $W$ cannot be a vector space.
		\end{proof}
		% subsection problem_6 (end)
	% section section_1_b (end)

	\section*{Section 1.C}	
		\subsection*{Problem 3}
		Show that the set of differentiable real-valued functions $f$ on the interval $(-􏰈4, 4)$ such that $f'(-1) = 3f(2)$ is a subspace of $\R^{[0,1]}$.

		\begin{proof}[Solution]
		\end{proof}
		% subsection problem_3 (end)

		\subsection*{Problem 8}
		Give an example of a nonempty subset $U$ of $\R^{2}$ such that $U$ is closed under scalar multiplication, but $U$ is not a subspace of $\R^{2}$.

		\begin{proof}[Solution]
		\end{proof}	
		% subsection problem_8 (end)

		\subsection*{Problem 10}
		Suppose $U_{1}$ and $U_{2}$ are subspaces of $V$. 
		Prove that the intersection $U_{1} \cap U_{2}$ is a subspace of $V$.

		\begin{proof}
		\end{proof}
		% subsection problem_10 (end)

		\subsection*{Problem 12}
		Prove that the union of two subspaces of V is a subspace of V if and only if one of the subspaces is contained in the other.

		\begin{proof}[Solution]
		\end{proof}		
		% subsection problem_12 (end)

		\subsection*{Problem 21}
		Suppose $$U = \{ (x,y,x+y,x-y,2x) \in \F^{5} : x, y \in \F \}.$$ 
		Find a subspace $W$ of $\F^{5}$ such that $\F^{5} = U \oplus W$.

		\begin{proof}[Solution]
		\end{proof}
		% subsection problem_21 (end)

		\subsection*{Problem 23}
		Prove or give a counterexample: if $U_{1}$, $U_{2}$, $W$ are subspaces of $V$ such that	$$V = U_{1} \oplus W \hspace{1cm} \text{and} \hspace{1cm} V = U_{2} \oplus W,$$ then $U_{1} = U_{2}$. 

		\begin{proof}[Solution]
		\end{proof}
		% subsection problem_23 (end)
	% section section_1_c (end)
\end{document}