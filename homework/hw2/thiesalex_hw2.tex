%!TEX output_directory = temp
\documentclass[letterpaper, 12pt]{amsart}
	%%%%%%%%%%%%%%%%%%%%%%%%%%%%%%%%%%%%%%%%%%%%%%%%%%%%%%%%%%%%%%%%%%%%%%%%%%%%%%
	%%%%%%%%%%%%%%%%%%%%%%%%%%%% boilerplate packages %%%%%%%%%%%%%%%%%%%%%%%%%%%%
		\usepackage[margin=2in]{geometry}
		\usepackage{amsmath,amssymb,amsthm}
		\usepackage{marvosym}
		\usepackage[mathscr]{euscript}
		\usepackage{enumerate}
		\usepackage{graphicx}
		\usepackage{mathrsfs}
		\usepackage{color}
		\usepackage{hyperref}
		\usepackage{verbatim}
		\usepackage{stmaryrd}

	%%%%%%%%%%%%%%%%%%%%%%%%%%%%%%%%%%%%%%%%%%%%%%%%%%%%%%%%%%%%%%%%%%%%%%%%%%%%%%
	%%%%%%%%%%%%%%%%%%%%%%%%%%%%% rename the abstract %%%%%%%%%%%%%%%%%%%%%%%%%%%%
		\renewcommand{\abstractname}{Assignment}

	%%%%%%%%%%%%%%%%%%%%%%%%%%%%%%%%%%%%%%%%%%%%%%%%%%%%%%%%%%%%%%%%%%%%%%%%%%%%%%
	%%%%%%%%%%%%%%%%%%%%%%%%%%%%%%%%%%%%% sets %%%%%%%%%%%%%%%%%%%%%%%%%%%%%%%%%%%
		\DeclareMathOperator{\N}{\mathbb{N}}				% natural numbers
		\DeclareMathOperator{\Z}{\mathbb{Z}}				% integers
		\DeclareMathOperator{\Zp}{\mathbb{Z}^{+}}			% positive integers
		\DeclareMathOperator{\Q}{\mathbb{Q}}				% rationals
		\DeclareMathOperator{\Qc}{\mathbb{Q}^{c}}			% irrationals
		\DeclareMathOperator{\R}{\mathbb{R}}				% reals
		\DeclareMathOperator{\F}{\mathbb{F}}				% a field
		\DeclareMathOperator{\C}{\mathbb{C}}				% complex numbers
		\DeclareMathOperator{\Cnon}{\mathbb{C}^{\times}}	% nonzero complex numbers
		\DeclareMathOperator{\Pcal}{\mathcal{P}}			% powerset, or set of polynomials
		\DeclareMathOperator{\Ell}{\mathscr{L}}				% set of linear maps, or linear operator

	%%%%%%%%%%%%%%%%%%%%%%%%%%%%%%%%%%%%%%%%%%%%%%%%%%%%%%%%%%%%%%%%%%%%%%%%%%%%%%
	%%%%%%%%%%%%%%%%%%%%%%%%%%%%%% use pretty letters %%%%%%%%%%%%%%%%%%%%%%%%%%%%
		\DeclareMathOperator{\ep}{\varepsilon}				% epsilons
		\DeclareMathOperator{\ph}{\varphi}					% phis

	%%%%%%%%%%%%%%%%%%%%%%%%%%%%%%%%%%%%%%%%%%%%%%%%%%%%%%%%%%%%%%%%%%%%%%%%%%%%%%
	%%%%%%%%%%%%%%%%%%%%%%%%%%%%%%%%%%% algebra %%%%%%%%%%%%%%%%%%%%%%%%%%%%%%%%%%
		\renewcommand{\null}{\text{null }}					% null space
		\DeclareMathOperator{\range}{\text{range }}			% range
		\newcommand{\bmat}[1]{{\mathbf{#1}}}				% bold matrix
		\newcommand{\bvec}[1]{{\vec{\mathbf{#1}}}}			% bold vector
		\DeclareMathOperator{\ind}{\perp\!\!\!\perp}		% perpendicular, orthogonal
		\DeclareMathOperator{\ord}{\text{ord}}				% order of a structure
		\DeclareMathOperator{\Log}{Log}						% logarithm
		\DeclareMathOperator{\Span}{Span}					% span
		\newcommand{\pid}[1]{\langle #1 \rangle}			% bracket notation, used for 
															% ideals or inner products
		\newcommand{\norm}[1]{\mid \!\!#1 \!\!\mid}			%\norm{x} gives |x|

		% fatdot notation
		\makeatletter
			\newcommand*\fatdot{\mathpalette\fatdot@{.5}}
			\newcommand*\fatdot@[2]{\mathbin{\vcenter{\hbox{\scalebox{#2}{$\m@th#1\bullet$}}}}}
		\makeatother

	%%%%%%%%%%%%%%%%%%%%%%%%%%%%%%%%%%%%%%%%%%%%%%%%%%%%%%%%%%%%%%%%%%%%%%%%%%%%%%
	%%%%%%%%%%%%%%%%%%%%%%%%%%% probability & statistics %%%%%%%%%%%%%%%%%%%%%%%%%
		\renewcommand{\Pr}{\mathbb{P}}						% probability
		\DeclareMathOperator{\E}{\mathbb{E}}				% expectation
		\DeclareMathOperator{\var}{\rm Var}					% variance
		\DeclareMathOperator{\sd}{\rm SD}					% standard deviation
		\DeclareMathOperator{\cov}{\rm Cov}					% covariance
		\DeclareMathOperator{\SE}{\rm SE}					% standard error
		\DeclareMathOperator{\ssreg}{{\rm SS}_{{\rm Reg}}}	% sum of squared regression
		\DeclareMathOperator{\ssr}{{\rm SS}_{{\rm Res}}}	% sum of squared residuals
		\DeclareMathOperator{\sst}{{\rm SS}_{{\rm Tot}}}	% total sum of squares

	%%%%%%%%%%%%%%%%%%%%%%%%%%%%%%%%%%%%%%%%%%%%%%%%%%%%%%%%%%%%%%%%%%%%%%%%%%%%%%
	%%%%%%%%%%%%%%%%%%%%%%%%%%%%%%% number theory %%%%%%%%%%%%%%%%%%%%%%%%%%%%%%%%
		\renewcommand{\mod}[1]{\ (\mathrm{mod}\ #1)}		% congruences

	%%%%%%%%%%%%%%%%%%%%%%%%%%%%%%%%%%%%%%%%%%%%%%%%%%%%%%%%%%%%%%%%%%%%%%%%%%%%%%
	%%%%%%%%%%%%%%%%%%%%%%%%%%%% theorem environments %%%%%%%%%%%%%%%%%%%%%%%%%%%%
		% Some theorem-like environments, all numbered together starting at 1
		% in each section.

		\newtheorem{thm}{Theorem}[section]					% The default \theoremstyle is 
		\newtheorem{defn}[thm]{Definition}					% bold headings and italic body text.
		\newtheorem{prop}[thm]{Proposition}
		\newtheorem{claim}[thm]{Claim}
		\newtheorem{cor}[thm]{Corollary}
		\newtheorem{lemma}[thm]{Lemma}

		\theoremstyle{definition}  							% Bold headings and Roman body text.
		\newtheorem{example}[thm]{Example}
		\newtheorem{examples}[thm]{Examples}
		\newtheorem{exercise}[thm]{Exercise}
		\newtheorem{note}[thm]{Note}
		\newtheorem{remark}[thm]{Remark}
		\newtheorem{remarks}[thm]{Remarks}
		\newtheorem{discussion}[thm]{Discussion}

		\newcommand{\dfn}{\textbf} 							% Make defined words bold.
		\newcommand{\mdfn}[1]{\dfn{\mathversion{bold}#1}} 	% Even make math symbols bold

	%%%%%%%%%%%%%%%%%%%%%%%%%%%%%%%%%%%%%%%%%%%%%%%%%%%%%%%%%%%%%%%%%%%%%%%%%%%%%%
	%%%%%%%%%%%%%%%%%%%%%%%%%%%%%%% complex numbers %%%%%%%%%%%%%%%%%%%%%%%%%%%%%%
		\DeclareMathOperator{\Arg}{Arg}						% argument of z \in \C
		\DeclareMathOperator{\re}{Re}						% real component
		\DeclareMathOperator{\im}{Im}						% imaginary component

	%%%%%%%%%%%%%%%%%%%%%%%%%%%%%%%%%%%%%%%%%%%%%%%%%%%%%%%%%%%%%%%%%%%%%%%%%%%%%%
	%%%%%%%%%%%%%%%%%%%%%%%%%%%%%%% various symbols %%%%%%%%%%%%%%%%%%%%%%%%%%%%%%
		\newcommand{\iso}{\cong}						% isometric/congruent
		\newcommand{\ra}{\rightarrow}                   % right arrow
		\newcommand{\Ra}{\Rightarrow}                   % right implies
		\newcommand{\lra}{\longrightarrow}              % long right arrow
		\newcommand{\la}{\leftarrow}                    % left arrow
		\newcommand{\La}{\Leftarrow}                    % left implies
		\newcommand{\lla}{\longleftarrow}               % long left arrow
		\newcommand{\eqra}{\llra{\sim}}                 % equivalence/isomorphism
		\newcommand{\blank}{\underbar{\ \ }}          	% An underscore, as in (__)xV
		% \newcommand{\blank}{-}                          % A hyphen, as in (-)xV
		\newcommand{\Id}{Id}                            % The identity functor
		\newcommand{\und}{\underline}
		\newcommand{\del}{\nabla}						% gradient vector

		\raggedbottom		
\begin{document}
	\title{Homework 2  -- Math 441 \\ A\lowercase{pril 25, 2018}}
	\author{Alex Thies \\ \href{mailto:athies@uoregon.edu}{\lowercase{athies$@$uoregon.edu}}}

	\begin{abstract}
	The following exercises are assigned from \textit{Linear Algebra Done Right}, 3rd Edition, by Sheldon Axler. 
			\begin{tabular}{rl}
				& 1.C - 21, 23; \\
				& 2.A - 5, 6, 9, 10, 12, 14, 15.
			\end{tabular}
	\end{abstract}

	\maketitle

	\section*{Section 1.C}
		\subsection*{Problem 21}
		Suppose $$U = \{(x, y, x+y, x-y, 2x) \in \F^{5} : x,y \in \F\}$$ Find a subspace $W$ of $\F^{5}$ such that $\F^{5} = U \oplus W$.

		\begin{proof}[Solution]
		\end{proof}
		% subsection problem_21 (end)

		\subsection*{Problem 23}
		Prove or give a counterexample: if $U_{1}$, $U_{2}$, $W$ are subspaces of $V$ such that $$V = U_{1} \oplus W \ \ \text{and} \ \ V = U_{2} \oplus W$$ then $U_{1} = U_{2}$.

		\begin{proof}[Solution]
		\end{proof}
		% subsection problem_23 (end)
	% section section_1_c (end)

	\section*{Section 2.A}
		\subsection*{Problem 5}
		\begin{enumerate}[\hspace*{5mm} (a)]
		\item Show that if we think of $\C$ as a vector space over $\R$, then the list $(1 + i, 1 - i)$ is linearly independent.
		\item Show that if we think of $\C$ as a vector space over $\C$, then the list $(1 + i, 1 - i)$ is linearly dependent.
		\end{enumerate}

		\begin{proof}[Solution]
		\end{proof}
		% subsection problem_5 (end)

		\subsection*{Problem 6}	
		Suppose $v_{1}, v_{2}, v_{3}, v_{4}$ is linearly independent in $V$. 
		Prove that the list $$v_{1} - 􏰋v_{2}, v_{2} - 􏰋v_{3}, v_{3} - 􏰋v_{4}, v_{4}$$ is also linearly independent.

		\begin{proof}
		\end{proof}
		% subsection problem_6 (end)

		\subsection*{Problem 9}
		Prove or give a counterexample: If $v_{1}, \dots, v_{m}$ and $w_{1}, \dots, w_{m}$ are linearly independent lists of vectors in $V$, then $v_{1} + w_{1}, \dots, v_{m} + w_{m}$ is linearly independent.

		\begin{proof}
		\end{proof}
		% subsection problem_9 (end)

		\subsection*{Problem 10}
		Suppose $v_{1}, \dots, v_{m}$ is linearly independent in $V$ and $w \in V$. 
		Prove that if $v_{1} + w, \dots, v_{m} + w$ is linearly dependent, then $w \in \Span(v_{1}, \dots, v_{m})$.

		\begin{proof}
		\end{proof}
		% subsection problem_10 (end)

		\subsection*{Problem 12}
		Explain why there does not exist a list of six polynomials that is linearly independent in $\mathcal{P}_{4}(\F)$.

		\begin{proof}
		\end{proof}
		% subsection problem_12 (end)

		\subsection*{Problem 14}
		Prove that $V$ is infinite-dimensional if and only if there is a sequence $v_{1}, v_{2}, \dots$ of vectors in $V$ such that $v_{1}, \dots, v_{m}$ is linearly independent for every positive integer $m$.

		\begin{proof}
		\end{proof}
		% subsection problem_14 (end)

		\subsection*{Problem 15}
		Prove that $\F^{\infty}$ is infinite-dimensional.

		\begin{proof}
		\end{proof}
		% subsection problem_15 (end)
	% section section_2_a (end)
\end{document}