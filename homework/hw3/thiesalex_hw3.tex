%!TEX output_directory = temp
\documentclass[letterpaper, 12pt]{amsart}
	%%%%%%%%%%%%%%%%%%%%%%%%%%%%%%%%%%%%%%%%%%%%%%%%%%%%%%%%%%%%%%%%%%%%%%%%%%%%%%
	%%%%%%%%%%%%%%%%%%%%%%%%%%%% boilerplate packages %%%%%%%%%%%%%%%%%%%%%%%%%%%%
	\usepackage{amsmath,amssymb,amsthm}
	\usepackage[mathscr]{euscript}
	\usepackage{enumerate}
	\usepackage{graphicx}
	\usepackage{mathrsfs}
	\usepackage{color}
	\usepackage{hyperref}
	\usepackage{verbatim}
	\usepackage{stmaryrd}
	\usepackage[margin=1.25in]{geometry}

	%%%%%%%%%%%%%%%%%%%%%%%%%%%%%%%%%%%%%%%%%%%%%%%%%%%%%%%%%%%%%%%%%%%%%%%%%%%%%%
	%%%%%%%%%%%%%%%%%%%%%%%%%%%%% rename the abstract %%%%%%%%%%%%%%%%%%%%%%%%%%%%
	% \renewcommand{\abstractname}{Introduction}

	%%%%%%%%%%%%%%%%%%%%%%%%%%%%%%%%%%%%%%%%%%%%%%%%%%%%%%%%%%%%%%%%%%%%%%%%%%%%%%
	%%%%%%%%%%%%%%%%%%%%%%%%%%%%%%%%%%%%% sets %%%%%%%%%%%%%%%%%%%%%%%%%%%%%%%%%%%
		%% sets 
		\DeclareMathOperator{\N}{\mathbb{N}}
		\DeclareMathOperator{\Z}{\mathbb{Z}}
		\DeclareMathOperator{\Zp}{\mathbb{Z}^{+}}
		\DeclareMathOperator{\Q}{\mathbb{Q}}
		\DeclareMathOperator{\Qp}{\mathbb{Q}^{+}}
		\DeclareMathOperator{\Qc}{\mathbb{Q}^{c}}
		\DeclareMathOperator{\R}{\mathbb{R}}
		\DeclareMathOperator{\F}{\mathbb{F}}
		\DeclareMathOperator{\Rp}{\mathbb{R}^{+}}
		\DeclareMathOperator{\C}{\mathbb{C}}
		\DeclareMathOperator{\Cnon}{\mathbb{C}^{\times}}
		%% powerset of a set
		\DeclareMathOperator{\pset}{\mathcal{P}}
		%% set of continuous functions in a certain variable
		\DeclareMathOperator{\cont}{\mathscr{C}}
		%% set of functions in a certain variable
		\DeclareMathOperator{\func}{\mathscr{F}}
		
	%%%%%%%%%%%%%%%%%%%%%%%%%%%%%%%%%%%%%%%%%%%%%%%%%%%%%%%%%%%%%%%%%%%%%%%%%%%%%%
	%%%%%%%%%%%%%%%%%%%%%%%%%%%%%%%% linear algebra %%%%%%%%%%%%%%%%%%%%%%%%%%%%%%
		%% linear span
		\DeclareMathOperator{\Ell}{\mathscr{L}}
		%% bold vectors with arrows, and bold matrices
		\newcommand{\bmat}[1]{{\mathbf{#1}}}
		\newcommand{\bvec}[1]{{\vec{\mathbf{#1}}}}
		%% independent vectors/matrices
		\DeclareMathOperator{\ind}{\perp\!\!\!\perp}
		%% order
		\DeclareMathOperator{\ord}{\text{ord}}

	%%%%%%%%%%%%%%%%%%%%%%%%%%%%%%%%%%%%%%%%%%%%%%%%%%%%%%%%%%%%%%%%%%%%%%%%%%%%%%
	%%%%%%%%%%%%%%%%%%%%%%%%%%% probability & statistics %%%%%%%%%%%%%%%%%%%%%%%%%
		%% probability, expectation, variance, etc.
		\renewcommand{\Pr}{\mathbb{P}}
		\DeclareMathOperator{\E}{\mathbb{E}}
		\DeclareMathOperator{\var}{\rm Var}
		\DeclareMathOperator{\sd}{\rm SD}
		\DeclareMathOperator{\cov}{\rm Cov}
		\DeclareMathOperator{\SE}{\rm SE}
		\DeclareMathOperator{\ssreg}{{\rm SS}_{{\rm Reg}}}
		\DeclareMathOperator{\ssr}{{\rm SS}_{{\rm Res}}}
		\DeclareMathOperator{\sst}{{\rm SS}_{{\rm Tot}}}

	%%%%%%%%%%%%%%%%%%%%%%%%%%%%%%%%%%%%%%%%%%%%%%%%%%%%%%%%%%%%%%%%%%%%%%%%%%%%%%
	%%%%%%%%%%%%%%%%%%%%%%%%%%%%%%%% congruences %%%%%%%%%%%%%%%%%%%%%%%%%%%%%%%%%
		\renewcommand{\mod}[1]{\ (\mathrm{mod}\ #1)}

	%%%%%%%%%%%%%%%%%%%%%%%%%%%%%%%%%%%%%%%%%%%%%%%%%%%%%%%%%%%%%%%%%%%%%%%%%%%%%%
	%%%%%%%%%%%%%%%%%%%%%%%%%%%%%% bracket notation %%%%%%%%%%%%%%%%%%%%%%%%%%%%%%
		% I first used this for principal ideals, that is why the abbreviation is pid
		\newcommand{\pid}[1]{\langle #1 \rangle}

	%%%%%%%%%%%%%%%%%%%%%%%%%%%%%%%%%%%%%%%%%%%%%%%%%%%%%%%%%%%%%%%%%%%%%%%%%%%%%%
	%%%%%%%%%%%%%%%%%%%%%%%%%%%%%%% fatdot notation %%%%%%%%%%%%%%%%%%%%%%%%%%%%%%
		\makeatletter
			\newcommand*\fatdot{\mathpalette\fatdot@{.5}}
			\newcommand*\fatdot@[2]{\mathbin{\vcenter{\hbox{\scalebox{#2}{$\m@th#1\bullet$}}}}}
		\makeatother

	%%%%%%%%%%%%%%%%%%%%%%%%%%%%%%%%%%%%%%%%%%%%%%%%%%%%%%%%%%%%%%%%%%%%%%%%%%%%%%
	%%%%%%%%%%%%%%%%%%%%%%%%%%%%%% use pretty letters %%%%%%%%%%%%%%%%%%%%%%%%%%%%
		\DeclareMathOperator{\ep}{\varepsilon}
		\DeclareMathOperator{\ph}{\varphi}

	%%%%%%%%%%%%%%%%%%%%%%%%%%%%%%%%%%%%%%%%%%%%%%%%%%%%%%%%%%%%%%%%%%%%%%%%%%%%%%
	%%%%%%%%%%%%%%%%%%%%%%%%%%% stolen from Jeske/Dugger %%%%%%%%%%%%%%%%%%%%%%%%%
	% Some theorem-like environments, all numbered together starting at 1
	% in each section.

	% The default \theoremstyle is bold headings and italic body text.
	\newtheorem{thm}{Theorem}[section]
	\newtheorem{defn}[thm]{Definition}
	\newtheorem{prop}[thm]{Proposition}
	\newtheorem{claim}[thm]{Claim}
	\newtheorem{cor}[thm]{Corollary}
	\newtheorem{lemma}[thm]{Lemma}

	\theoremstyle{definition}  % Bold headings and Roman body text.
	\newtheorem{example}[thm]{Example}
	\newtheorem{examples}[thm]{Examples}
	\newtheorem{exercise}[thm]{Exercise}
	\newtheorem{note}[thm]{Note}
	\newtheorem{remark}[thm]{Remark}
	\newtheorem{remarks}[thm]{Remarks}
	\newtheorem{discussion}[thm]{Discussion}

	\newcommand{\dfn}{\textbf} % Make defined words bold.
	\newcommand{\mdfn}[1]{\dfn{\mathversion{bold}#1}} % Even make math symbols bold

	% Various commands that are useful.  Please add your own.

	\DeclareMathOperator{\Arg}{Arg}
	\DeclareMathOperator{\re}{Re}
	\DeclareMathOperator{\im}{Im}
	\DeclareMathOperator{\Log}{Log}
	\DeclareMathOperator{\Span}{Span}

	\newcommand{\iso}{\cong}						% isometric/congruent
	\newcommand{\ra}{\rightarrow}                   % right arrow
	\newcommand{\Ra}{\Rightarrow}                   % right implies
	\newcommand{\lra}{\longrightarrow}              % long right arrow
	\newcommand{\la}{\leftarrow}                    % left arrow
	\newcommand{\La}{\Leftarrow}                    % left implies
	\newcommand{\lla}{\longleftarrow}               % long left arrow
	\newcommand{\llra}[1]{\stackrel{#1}{\lra}}      % labeled long right arrow
	\newcommand{\we}{\llra{\sim}}                   % weak equivalence
	\newcommand{\cof}{\rightarrowtail}              % cofibration
	\newcommand{\fib}{\twoheadrightarrow}           % fibration
	\newcommand{\inc}{\hookrightarrow}              % inclusion
	\newcommand{\dbra}{\rightrightarrows}           % double arrow for equalizer diagrams
	\newcommand{\eqra}{\llra{\sim}}                 % equivalence/isomorphism

	% \newcommand{\blank}{\underbar{\ \ }}          % An underscore, as in (__)xV
	\newcommand{\blank}{-}                          % A hyphen, as in (-)xV
	\newcommand{\Id}{Id}                            % The identity functor
	\newcommand{\und}{\underline}
	\newcommand{\norm}[1]{\mid \!\!#1 \!\!\mid}             %\norm{x} gives |x|

	% These commands are for the period and comma in the lower right entry of
	% a diagram.  They put the punctuation 2 pts to the right, but make
	% TeX (and hence the diagram package) unaware of the extra width
	% of that entry.
	\newcommand{\period}    {{\makebox[0pt][l]{\hspace{2pt} .}}}
	\newcommand{\comma}     {{\makebox[0pt][l]{\hspace{2pt} ,}}}
	\newcommand{\semicolon} {{\makebox[0pt][l]{\hspace{2pt} ;}}}

	\newcommand{\Cech}{\v{C}ech}
	\newcommand{\scat}{\Delta}
	\newcommand{\assign}{\ra}
	\newcommand{\copr}{\,\amalg\,}
	\newcommand{\ovcat}{\downarrow}
	\newcommand{\pder}[2]{{\frac{\partial #1}{\partial #2}}}
	\newcommand{\del}{\nabla}
	\newcommand{\vectr}[1]{{\mbox{\boldmath $#1$}}}
	\newcommand{\uvectr}[1]{\vectr{\hat #1}}
	\newcommand{\ihat}{\uvectr \imath}
	\newcommand{\jhat}{\uvectr \jmath}
	\newcommand{\khat}{\uvectr k}
	\newcommand{\rhat}{\uvectr r}
	\newcommand{\thhat}{\uvectr \theta}
	\newcommand{\zhat}{\uvectr z}
	\newcommand{\rhohat}{\uvectr \rho}
	\newcommand{\phihat}{\uvectr \phi}
	\newcommand{\grad}{\vectr{\vec\nabla}}
	% \newcommand{\R}{\mathbb{R}}
	\newcommand{\vv}[1]{\vectr{v_{#1}}}
	\newcommand{\crad}{0.1}
	\newcommand{\lline}[1]{\overleftrightarrow{#1}}
	\DeclareMathOperator{\area}{area}
	\DeclareMathOperator{\vol}{vol}
	\newcommand{\ray}[1]{\overset{\rightarrow}{#1}}
	\newcommand{\sr}[2]{???}
	\newcommand{\iihat}{i}
	\newcommand{\jjhat}{j}
	\newcommand{\kkhat}{k}

		
\begin{document}
	\title{Homework 3  -- Math 441 \\ \today}
	\author{Alex Thies \\ \href{mailto:athies@uoregon.edu}{\lowercase{athies$@$uoregon.edu}}}

	\maketitle

	Assignment: 2.B - 5, 7, 8; 2.C - 1, 9, 10, 11, 14, 17.

	\section*{Section 2.B}
		\subsection*{Problem 5}
		Prove or disprove: there exists a basis $p_{0}$, $p_{1}$, $p_{2}$, $p_{3}$ of $\mathcal{P}_{3}(\F)$ such that none of the polynomials $p_{0}$, $p_{1}$, $p_{2}$, $p_{3}$ has degree 2.

		\begin{proof}[Counterexample]
		We provide a counterexample.
		Recall that for a vector space $V$ of dimension $m$, with basis $v_{1}, \dots, v_{m}$, a list of $m$ linear combinations of basis elements $v_{1}, \dots, v_{m}$ is itself a basis for $V$.
		With that in mind we will create some useful linear combinations of the standard basis elements of $\mathcal{P}_{3}(\F)$.
		Let $e_{0} = x^{0}$, $e_{1} = x^{1}$, $e_{2} = x^{2}$, $e_{3} = x^{3}$ be the standard basis of $\mathcal{P}_{3}(\F)$; our $p_{i}$'s will be linear combinations of these basis elements.
		Let $p_{0} = x^{0}$, $p_{1} = x^{0} + x^{1}$, $p_{2} = x^{2} + x^{3}$, $p_{3} = x^{3}$, for reasons explained above, these form a basis of $\mathcal{P}_{3}(\F)$.
		Notice that none of the $p_{i}$'s are degree 2.
		\end{proof}
		% subsection problem_5 (end)

		\subsection*{Problem 7}
		Prove or give a counterexample: If $v_{1}$, $v_{2}$, $v_{3}$, $v_{4}$ is a basis of $V$ and $U$ is a subspace of $V$ such that $v_{1}$, $v_{2} \in U$ and $v_{3} \notin U$ and $v_{4} \notin U$, then $v_{1}$, $v_{2}$ is a basis of $U$.

		\begin{proof}[Counterexample]
		Let $V = \R^{4}$, $U = \{ (x_{1}, x_{2}, 0, x_{3}) : x_{i} \in \R \}$ and $v_{1} = (1,0,0,0)$, $v_{2} = (0,1,0,0)$, $v_{3} = (0,0,1,0)$, and $v_{4} = (0,0,1,1)$.
		Notice that $U$ is a subspace of $V$, that $v_{1}$, $v_{2}$, $v_{3}$, $v_{4}$ is a basis of $V$, and that $v_{1}, v_{2} \in U$, and $v_{3} \notin U$ and $v_{4} \notin U$; so we have satisfied all the conditions above.
		However, notice that $v_{1}, v_{2}$ is clearly not a basis of $U$, because neither vector contains an $x_{3}$ term.
		\end{proof}
		% subsection problem_7 (end)

		\subsection*{Problem 8}
		Suppose $U$ and $W$ are subspaces of $V$ such that $V = U \oplus W$. 
		Suppose also that $u_{1}, \dots, u_{m}$ is a basis of $U$ and $w_{1}, \dots, w_{n}$ is a basis of $W$. 
		Prove that $$ u_{1}, \dots, u_{m}, w_{1}, \dots, w_{n} $$ is a basis of $V$.

		\begin{proof}
		Let $U$, $W$, $V$, $u_{1}, \dots, u_{m}$, and $w_{1}, \dots, w_{n}$ be as above.
		To show that $u_{1}, \dots, u_{m},$ $w_{1}, \dots, w_{n}$ is a basis of $V$, we will show that it is a linearly independent list, and that it spans $V$.

		Since $u_{1}, \dots, u_{m}$ and $w_{1}, \dots, w_{n}$ are bases of $U$ and $W$, respectively, we know that they are linearly independent.
		Therefore we know $0 = a_{1}u_{1} + \cdots + a_{m}u_{m}$ and $0 = b_{1}w_{1} + \cdots + b_{n}w_{n}$ only for $a_{1} = \cdots = a_{m} = 0$ and $b_{1} = \cdots = b_{n} = 0$.
		Thus $0 = 0 + 0 = a_{1}u_{1} + \cdots + a_{m}u_{m} + b_{1}w_{1} + \cdots + b_{n}w_{n}$, and since $V = U \oplus W$, we know that this linear combination is unique, hence $u_{1}, \dots, u_{m},$ $w_{1}, \dots, w_{n}$ is linearly independent; it remains to show that it spans $V$.

		Since $V = U \oplus W$ implies that $v = u + w$ is unique for $u \in U$ and $w \in W$, and since both $u_{1}, \dots, u_{m}$ and $w_{1}, \dots, w_{n}$ are bases of $U$ and $W$, we have that 
			\begin{align*}
				v &= u + w, \\
				&= a_{1}u_{1} + \cdots + a_{m}u_{m} + b_{1}w_{1} + \cdots + b_{n}w_{n}.
			\end{align*}
		So, we can express each element in $V$ as a unique linear combination of elements from the linearly independent list of vectors $u_{1}, \dots, u_{m},$ $w_{1}, \dots, w_{n}$.
		It follows that $u_{1}, \dots, u_{m},$ $w_{1}, \dots, w_{n}$ spans $V$, and is a basis for $V$, as we aimes to prove.
		\end{proof}
		% subsection problem_8 (end)
	% section section_2_b (end)

	\section*{Section 2.C}
		\subsection*{Problem 1}
		Suppose $V$ is finite-dimensional and $U$ is a subspace of $V$ such that $\dim U = \dim V$. 
		Prove that $U = V$.

		\begin{proof}
		Let $U$ and $V$ be as above, we will prove that they are equal.
		Let $n = \dim U = \dim V$, and $u_{1}, \dots, u_{n}$ and $v_{1}, \dots, v_{n}$ be bases for $U$ and $V$, respectively.
		Since $U$ is a subspace of $V$, each of $u_{1}, \dots, u_{n}$ are also elements of $V$.
		So $u_{1}, \dots, u_{n}$ is a linearly independent list of vectors in $V$ with length $\dim V$, and $u_{1}, \dots, u_{n}$ is a basis for both $U$ and $V$.
		Hence, since $U$ and $V$ have identical bases, we have $U = V$.
		\end{proof}
		% subsection problem_1 (end)

		\subsection*{Problem 9}
		Suppose $v_{1}, \dots, v_{m}$ is linearly independent in $V$ and $w \in V$. 
		Prove that $$\dim \Span(v_{1} + w, \dots, v_{m} + w) \geq m-1.$$

		\begin{proof}
		Let $v_{1}, \dots, v_{m}$, $V$, and $w$ be as above.

		...
		\end{proof}
		% subsection problem_9 (end)

		\subsection*{Problem 10}
		Suppose $p_{0}, p_{1}, \dots, p_{m}, \in \mathcal{P}(\F)$ are such that each $p_{j}$ has degree $j$. 
		Prove that $p_{0}, p_{1}, \dots, p_{m}$ is a basis of $\mathcal{P}_{m}(\F)$.

		\begin{proof}
		Let $p_{0}, p_{1}, \dots, p_{m}$ be as above, further, let $p_{i} = x^{i}$ for each $0 \leq i \leq m$.
		We proceed by mathematical induction.
			\begin{enumerate}
				\item[\textbf{Base case}:] Let $n = 0$, then $p_{0} = x^{0} = 1$.
				Since $\mathcal{P}_{0}(\F)$ is just the set of constant functions, its clear that $p_{0} = 1$ forms a basis for $\mathcal{P}_{0}(\F)$.

				\item[\textbf{Hypothesis}:] Assume that for some $n \in \N$ we have that $p_{0}, p_{1}, \dots, p_{n}$ is a basis of $\mathcal{P}_{n}(\F)$.

				\item[\textbf{Induction}:] We will show that for the given proposition $P(n)$, that our induction hypothesis implies $P(n+1)$.
			\end{enumerate}

		I ran out of time to finish this.
		\end{proof}
		% subsection problem_10 (end)

		\subsection*{Problem 11}
		Suppose that $U$ and $W$ are subspaces of $\R^{8}$ such that $\dim U = 3$, $\dim W = 5$, and $U + W = \R^{8}$. 
		Prove that $\R^{8} = U \oplus W$.

		\begin{proof}
		Let $U$ and $W$ be as above, and recall that $V = U \oplus W$ is equivalent to $U \cap W = \{ 0 \}$.
		Further, recall the identity:
			\begin{align*}
				\dim(U + W) &= \dim U + \dim W - \dim(U \cap W), \\
				\dim(U \cap W) &= \dim U + \dim W - \dim(U + W).
			\end{align*}
		Notice that $3 + 5 - 8 = 0$, which implies that $\dim(U \cap W) = 0 \iff U \cap W = \{ 0 \}$.
		Hence, $\R^{8} = U \oplus W$, as we aimed to show.	
		\end{proof}
		% subsection problem_11 (end)

		\subsection*{Problem 14}
		Suppose $U_{1}, \dots, U_{m}$, are finite-dimensional subspaces of $V$.
		Prove that $U_{1} + \cdots + U_{m}$ is finite-dimensional and $$\dim(U_{1} + \cdots + U_{m}) \leq \dim U_{1} + \cdots + \dim U_{m}.$$

		\begin{proof}
		Let $U_{1}, \dots, U_{m}$ and $V$ be as above.
		Let $u_{1_{1}}, \dots, u_{1_{m}}$ be a basis for $U_{1}$, and generally, let $u_{i_{1}}, \dots, u_{i_{j}}, \dots u_{i_{m}}$ be a basis for $U_{i}$.
		From the definition of sums of vector spaces, each $v \in U_{1} + \cdots + U_{m}$ can be written as a linear combination of each of the $u_{i_{j}}$'s.
		It follows then that $u_{1_{1}}, \dots, u_{1_{m}}, \dots, u_{m_{1}}, \dots, u_{m_{m}}$ spans $U_{1} + \cdots + U_{m}$.
		Thus by Theorem 2.31, since $u_{1_{1}}, \dots, u_{1_{m}}, \dots, u_{m_{1}}, \dots, u_{m_{m}}$ spans $U_{1} + \cdots + U_{m}$, it contains a basis for $U_{1} + \cdots + U_{m}$, i.e., it could be made shorter.
		Thus $\dim(U_{1} + \cdots + U_{m}) \leq \dim U_{1} + \cdots + \dim U_{m}$; it remains to show that $U_{1} + \cdots + U_{m}$ is finite-dimensional.
		Notice that each of $\dim U_{i} \in \Z^{+}$, and that $\dim U_{1} + \cdots + \dim U_{m}$ is a finite sum.
		Hence $\dim(U_{1} + \cdots + U_{m})$ is less than or equal to a finite sum of integers, which implies that $\dim(U_{1} + \cdots + U_{m}) \leq t$ for some $t \in \Z^{+}$, thus $U_{1} + \cdots + U_{m}$ is finite-dimensional.
		\end{proof}
		% subsection problem_14 (end)

		\subsection*{Problem 17}
		You might guess, by analogy with the formula for the number of elements in the union of three subsets of a finite set, that if $U_{1}, U_{2}, U_{3}$ are subspaces of a finite-dimensional vector space, then 
		\begin{align*}
			\dim(U_{1} + U_{2} + U_{3}) &= \dim U_{1} + \dim U_{2} + \dim U_{3} \\
			&\hspace{5mm} - \dim(U_{1} \cap U_{2}) - \dim(U_{1} \cap U_{3}) - \dim(U_{2} \cap U_{3}) \\
			&\hspace{5mm} + \dim(U_{1} \cap U_{2} \cap U_{3}).
		\end{align*}
		Prove this or give a counterexample.

		\begin{proof}[Counterexample]
		Given the usefullness of this identity in set theory when applied to probability, I really wanted it to be true, but I ended up finding a counterexample.
		Consider $V = \F^{2}$ and $U_{1} = \{ (x,x) : x \in \F \}$, $U_{2} = \{ (y,0) : x \in \F \}$, and $U_{1} = \{ (0,z) : y \in \F \}$.
		It's clear that $\dim U_{i} = 1$ for $1 \leq i \leq 3$, but each of the pairwise, and also the triple intersection are simply $\{ 0 \}$, so their dimensions are $0$, i.e., $\dim U_{1}U_{2} = \dim U_{1}U_{3} = \dim U_{2}U_{3} = \dim U_{1}U_{2}U_{3} = 0$.
		According to the above identity, this would imply that $\dim(\F^{2}) = \dim U_{1} + \cdots - \dim U_{1}U_{2}U_{3} \iff  2 = 3$, which is obviously false.
		\end{proof}
		% subsection problem_17 (end)
	% section section_2_c (end)
\end{document}