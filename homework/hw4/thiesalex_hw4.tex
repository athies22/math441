%!TEX output_directory = temp
\documentclass[letterpaper, 12pt]{amsart}
	%%%%%%%%%%%%%%%%%%%%%%%%%%%%%%%%%%%%%%%%%%%%%%%%%%%%%%%%%%%%%%%%%%%%%%%%%%%%%%
	%%%%%%%%%%%%%%%%%%%%%%%%%%%% boilerplate packages %%%%%%%%%%%%%%%%%%%%%%%%%%%%
		\usepackage[margin=2in]{geometry}
		\usepackage{amsmath,amssymb,amsthm}
		\usepackage{marvosym}
		\usepackage[mathscr]{euscript}
		\usepackage{enumerate}
		\usepackage{graphicx}
		\usepackage{mathrsfs}
		\usepackage{color}
		\usepackage{hyperref}
		\usepackage{verbatim}
		\usepackage{stmaryrd}

	%%%%%%%%%%%%%%%%%%%%%%%%%%%%%%%%%%%%%%%%%%%%%%%%%%%%%%%%%%%%%%%%%%%%%%%%%%%%%%
	%%%%%%%%%%%%%%%%%%%%%%%%%%%%% rename the abstract %%%%%%%%%%%%%%%%%%%%%%%%%%%%
		\renewcommand{\abstractname}{Assignment}

	%%%%%%%%%%%%%%%%%%%%%%%%%%%%%%%%%%%%%%%%%%%%%%%%%%%%%%%%%%%%%%%%%%%%%%%%%%%%%%
	%%%%%%%%%%%%%%%%%%%%%%%%%%%%%%%%%%%%% sets %%%%%%%%%%%%%%%%%%%%%%%%%%%%%%%%%%%
		\DeclareMathOperator{\N}{\mathbb{N}}				% natural numbers
		\DeclareMathOperator{\Z}{\mathbb{Z}}				% integers
		\DeclareMathOperator{\Zp}{\mathbb{Z}^{+}}			% positive integers
		\DeclareMathOperator{\Q}{\mathbb{Q}}				% rationals
		\DeclareMathOperator{\Qc}{\mathbb{Q}^{c}}			% irrationals
		\DeclareMathOperator{\R}{\mathbb{R}}				% reals
		\DeclareMathOperator{\F}{\mathbb{F}}				% a field
		\DeclareMathOperator{\C}{\mathbb{C}}				% complex numbers
		\DeclareMathOperator{\Cnon}{\mathbb{C}^{\times}}	% nonzero complex numbers
		\DeclareMathOperator{\Pcal}{\mathcal{P}}			% powerset, or set of polynomials
		\DeclareMathOperator{\Ell}{\mathscr{L}}				% set of linear maps, or linear operator

	%%%%%%%%%%%%%%%%%%%%%%%%%%%%%%%%%%%%%%%%%%%%%%%%%%%%%%%%%%%%%%%%%%%%%%%%%%%%%%
	%%%%%%%%%%%%%%%%%%%%%%%%%%%%%% use pretty letters %%%%%%%%%%%%%%%%%%%%%%%%%%%%
		\DeclareMathOperator{\ep}{\varepsilon}				% epsilons
		\DeclareMathOperator{\ph}{\varphi}					% phis

	%%%%%%%%%%%%%%%%%%%%%%%%%%%%%%%%%%%%%%%%%%%%%%%%%%%%%%%%%%%%%%%%%%%%%%%%%%%%%%
	%%%%%%%%%%%%%%%%%%%%%%%%%%%%%%%%%%% algebra %%%%%%%%%%%%%%%%%%%%%%%%%%%%%%%%%%
		\renewcommand{\null}{\text{null }}					% null space
		\DeclareMathOperator{\range}{\text{range }}			% range
		\newcommand{\bmat}[1]{{\mathbf{#1}}}				% bold matrix
		\newcommand{\bvec}[1]{{\vec{\mathbf{#1}}}}			% bold vector
		\DeclareMathOperator{\ind}{\perp\!\!\!\perp}		% perpendicular, orthogonal
		\DeclareMathOperator{\ord}{\text{ord}}				% order of a structure
		\DeclareMathOperator{\Log}{Log}						% logarithm
		\DeclareMathOperator{\Span}{Span}					% span
		\newcommand{\pid}[1]{\langle #1 \rangle}			% bracket notation, used for 
															% ideals or inner products
		\newcommand{\norm}[1]{\mid \!\!#1 \!\!\mid}			%\norm{x} gives |x|

		% fatdot notation
		\makeatletter
			\newcommand*\fatdot{\mathpalette\fatdot@{.5}}
			\newcommand*\fatdot@[2]{\mathbin{\vcenter{\hbox{\scalebox{#2}{$\m@th#1\bullet$}}}}}
		\makeatother

	%%%%%%%%%%%%%%%%%%%%%%%%%%%%%%%%%%%%%%%%%%%%%%%%%%%%%%%%%%%%%%%%%%%%%%%%%%%%%%
	%%%%%%%%%%%%%%%%%%%%%%%%%%% probability & statistics %%%%%%%%%%%%%%%%%%%%%%%%%
		\renewcommand{\Pr}{\mathbb{P}}						% probability
		\DeclareMathOperator{\E}{\mathbb{E}}				% expectation
		\DeclareMathOperator{\var}{\rm Var}					% variance
		\DeclareMathOperator{\sd}{\rm SD}					% standard deviation
		\DeclareMathOperator{\cov}{\rm Cov}					% covariance
		\DeclareMathOperator{\SE}{\rm SE}					% standard error
		\DeclareMathOperator{\ssreg}{{\rm SS}_{{\rm Reg}}}	% sum of squared regression
		\DeclareMathOperator{\ssr}{{\rm SS}_{{\rm Res}}}	% sum of squared residuals
		\DeclareMathOperator{\sst}{{\rm SS}_{{\rm Tot}}}	% total sum of squares

	%%%%%%%%%%%%%%%%%%%%%%%%%%%%%%%%%%%%%%%%%%%%%%%%%%%%%%%%%%%%%%%%%%%%%%%%%%%%%%
	%%%%%%%%%%%%%%%%%%%%%%%%%%%%%%% number theory %%%%%%%%%%%%%%%%%%%%%%%%%%%%%%%%
		\renewcommand{\mod}[1]{\ (\mathrm{mod}\ #1)}		% congruences

	%%%%%%%%%%%%%%%%%%%%%%%%%%%%%%%%%%%%%%%%%%%%%%%%%%%%%%%%%%%%%%%%%%%%%%%%%%%%%%
	%%%%%%%%%%%%%%%%%%%%%%%%%%%% theorem environments %%%%%%%%%%%%%%%%%%%%%%%%%%%%
		% Some theorem-like environments, all numbered together starting at 1
		% in each section.

		\newtheorem{thm}{Theorem}[section]					% The default \theoremstyle is 
		\newtheorem{defn}[thm]{Definition}					% bold headings and italic body text.
		\newtheorem{prop}[thm]{Proposition}
		\newtheorem{claim}[thm]{Claim}
		\newtheorem{cor}[thm]{Corollary}
		\newtheorem{lemma}[thm]{Lemma}

		\theoremstyle{definition}  							% Bold headings and Roman body text.
		\newtheorem{example}[thm]{Example}
		\newtheorem{examples}[thm]{Examples}
		\newtheorem{exercise}[thm]{Exercise}
		\newtheorem{note}[thm]{Note}
		\newtheorem{remark}[thm]{Remark}
		\newtheorem{remarks}[thm]{Remarks}
		\newtheorem{discussion}[thm]{Discussion}

		\newcommand{\dfn}{\textbf} 							% Make defined words bold.
		\newcommand{\mdfn}[1]{\dfn{\mathversion{bold}#1}} 	% Even make math symbols bold

	%%%%%%%%%%%%%%%%%%%%%%%%%%%%%%%%%%%%%%%%%%%%%%%%%%%%%%%%%%%%%%%%%%%%%%%%%%%%%%
	%%%%%%%%%%%%%%%%%%%%%%%%%%%%%%% complex numbers %%%%%%%%%%%%%%%%%%%%%%%%%%%%%%
		\DeclareMathOperator{\Arg}{Arg}						% argument of z \in \C
		\DeclareMathOperator{\re}{Re}						% real component
		\DeclareMathOperator{\im}{Im}						% imaginary component

	%%%%%%%%%%%%%%%%%%%%%%%%%%%%%%%%%%%%%%%%%%%%%%%%%%%%%%%%%%%%%%%%%%%%%%%%%%%%%%
	%%%%%%%%%%%%%%%%%%%%%%%%%%%%%%% various symbols %%%%%%%%%%%%%%%%%%%%%%%%%%%%%%
		\newcommand{\iso}{\cong}						% isometric/congruent
		\newcommand{\ra}{\rightarrow}                   % right arrow
		\newcommand{\Ra}{\Rightarrow}                   % right implies
		\newcommand{\lra}{\longrightarrow}              % long right arrow
		\newcommand{\la}{\leftarrow}                    % left arrow
		\newcommand{\La}{\Leftarrow}                    % left implies
		\newcommand{\lla}{\longleftarrow}               % long left arrow
		\newcommand{\eqra}{\llra{\sim}}                 % equivalence/isomorphism
		\newcommand{\blank}{\underbar{\ \ }}          	% An underscore, as in (__)xV
		% \newcommand{\blank}{-}                          % A hyphen, as in (-)xV
		\newcommand{\Id}{Id}                            % The identity functor
		\newcommand{\und}{\underline}
		\newcommand{\del}{\nabla}						% gradient vector

		\raggedbottom		
\begin{document}
	\title{Homework 4  -- Math 441 \\ M\lowercase{ay 9, 2018}}
	\author{Alex Thies \\ \href{mailto:athies@uoregon.edu}{\lowercase{athies$@$uoregon.edu}}}

	\begin{abstract}
	The following exercises are assigned from \textit{Linear Algebra Done Right}, 3rd Edition, by Sheldon Axler. 
			\begin{tabular}{rl}
				& 3.A - 3, 4, 7, 8, 10; \\
				& 3.B - 4, 9, 31.
			\end{tabular}
	\end{abstract}
	
	\maketitle

	\section*{Section 3.A}
		\subsection*{Problem 3}
		Suppose $T \in \Ell (\F^{n}, \F^{m})$.
		Show that there exist scalars $A_{j,k} \in \F$ for $j = 1, \dots, m$ and $k = 1, \dots, n$ such that $$T(x_{1}, \dots, x_{n}) = (A_{1,1} x_{1} + \cdots + A_{1,n}x_{n}, \cdots, A_{m,1}x_{1} + \cdots + A_{m,n}x_{n})$$ for every $(x_{1}, \dots, x_{n}) \in \F^{n}$.

		\begin{proof}[Solution]
		Let $T$ be as above, and let $e_{1}, \dots, e_{n}$ be the standard basis for $\F^{n}$, i.e., let $e_{i} \in \F^{n}$ for $1 \leq i \leq n$ be a vector of all zeros, with the exception that the $i^{th}$ entry is a 1.
		This allows us to greatly simplify the notation we're working with.
		Consider the basis element $(1,0, \dots, 0) \in \F^{n}$; if we apply $T$ to this basis element, then  we get $A_{1,1} (1) + A_{1,2} (0) + \cdots + A_{1,n}(0), \ A_{2,1} (1) + A_{2,2} (0) + \cdots + A_{2,n}(0), \ \dots, \ A_{m,1} (1) + A_{m,2} (0) + \cdots + A_{m,n}(0)$, because the only terms that don't get multiplied by zero are precisely the terms being multiplied by $x_{1} = 1$.
		Thus, we get $Te_{1} = (A_{1,1}, A_{2,1}, \dots, A_{m,1})$.
		Continue this process and we obtain:
			\begin{align*}
				Te_{2} &= (A_{1,2}, A_{2,2}, \dots, A_{m,2}), \\
				Te_{3} &= (A_{1,3}, A_{2,3}, \dots, A_{m,3}), \\
				& \hspace{2mm} \vdots \\
				Te_{n} &= (A_{m,1}, A_{m,2}, \dots, A_{m,n}).
			\end{align*}
		Since $e_{1}, \dots, e_{n}$ is a basis for $\F^{n}$ and $(x_{1}, \dots, x_{n}) \in \F^{n}$ we have $(x_{1}, \dots, x_{n}) \in \Span(e_{1}, \dots, e_{n})$, so $(x_{1}, \dots, x_{n})$ can be written as linear combinations of the basis elements $e_{1}, \dots, e_{n}$.
		So, with $n$-many $x_{i}$'s for components, instead of the 1 and $(n-1)$-many zeros that we used for the standard basis elements, we get $T(x_{1}, \dots, x_{n}) = (A_{1,1} x_{1} + \cdots + A_{1,n}x_{n}, \cdots, A_{m,1}x_{1} + \cdots + A_{m,n}x_{n})$, as we aimed to show.
		\end{proof}
		% subsection problem_3 (end)

		\subsection*{Problem 4}
		Suppose $T \in \Ell(V,W)$ and $v_{1}, \dots, v_{m}$ is a list of vectors in $V$ such that $Tv_{1}, \dots, Tv_{n}$ is a linearly independent list in $W$.
		Prove that $v_{1}, \dots, v_{m}$ is linearly independent.

		\begin{proof}
		Let $V,W,T$, $v_{1}, \dots, v_{m}$, and $\, Tv_{1}, \dots, Tv_{n}$ be as above.
		Then, since $Tv_{1}, \dots, Tv_{n}$ is linearly independent in $W$, we know that $a_{1}Tv_{1} + \cdots + a_{n}Tv_{n} = 0$ if and only if each $a_{i} = 0$ with $1 \leq i \leq n$.
		From the additivity of linear maps we can write $T(a_{1}v_{1} + \cdots + a_{n}v_{n}) = 0$.
		To show that $v_{1}, \dots, v_{n}$ is linearly independent, it remains to show that no other choice of $a_{i}$'s yields $a_{1}v_{1} + \cdots + a_{n}v_{n} = 0$.
		Suppose that there exist $c_{i} \in V$ with $1 \leq i \leq n$ such that $c_{1}v_{1} + \cdots c_{n}v_{n} = 0$.
		Since $T$ maps $0_{V}$ to $0_{W}$, we have that $T(c_{1}v_{1} + \cdots c_{n}v_{n}) = 0$, again by additivity we can write $c_{1}Tv_{1} + \cdots + c_{n}Tv_{n} = 0$.
		By the linear independence of $Tv_{1}, \dots, Tv_{n}$, and given our previous statement, it follows that $a_{i} = c_{i} = 0$ for each $1 \leq i \leq n$.
		Thus, our choice of $a_{i}$'s was unique, which satisfies the condition for $v_{1}, \dots, v_{n}$ to be linearly independent, as we aimed to show.
		\end{proof}
		% subsection problem_4 (end)

		\subsection*{Problem 7}
		Show that every linear map from a 1-dimensional vector space to itself is multiplication by some scalar. 
		More precisely, prove that if $\dim V = 1$ and $T \in \Ell(V,V)$, then there exists 􏰀$\lambda \in \F$ such that $Tv = \lambda v$ for all $v \in V$.

		\begin{proof}
		Let $V$ and $T$ be as above.
		Since $\dim V = 1$, take $u_{1}$ as a basis for $V$, additionally, with $T \in \Ell(V,V)$, it follows that $Tu_{1} \in V$ as well.
		Since $u_{1}$ is a basis for $V$, each element of $V$ is a scalar multiple of $u_{1}$; linear combinations of one element are pretty boring.
		Recall that vector spaces are closed under their operations, therefore \textit{all} scalar multiples $\lambda u_{1}$ (with scalars from $\F$) are elements of $V$, and since $T$ is mapping elements of $V$ to other elements of $V$, it follows that $T$ is doing this mapping by scalar multiplication.
		More formally, $Tu_{1} = \lambda u_{1}$ for some $\lambda \in \F$.
		Let $v \in V$ and $a \in \F$ such that $v = a u_{1}$, we compute the following:
			\begin{align*}
				Tv &= T(au_{1}), \\
				&= a T(u_{1}), \\
				&= a \lambda u_{1}, \\
				&= \lambda a u_{1}, \\
				&= \lambda v.
			\end{align*}
		Thus, since $\dim V = 1$ implies that any basis of $V$ consists of only one basis element, for $T \in \Ell(V,V)$, there exists 􏰀$\lambda \in \F$ such that $Tv = \lambda v$ for all $v \in V$
		\end{proof}
		% subsection problem_7 (end)

		\subsection*{Problem 8}
		Give an example of a function $\varphi : \R^{2} \to \R$ such that $\varphi(av) = a\varphi(v)$ for all $a \in \R$ and all $v \in \R^{2}$ but $\varphi$ is not linear.

		\begin{proof}[Solution]
		Let $\varphi : \R^{2} \to \R$ by $(x,y) \mapsto (x + y)^{1/2}$; consider $(1,0)$ and $(0,1)$, the standard basis elements of $\R^{2}$.
		Notice that $\varphi(1,0) = 1$, \ $\varphi(0,1) = 1$, and $\varphi(1,0) + \varphi(0,1) = 2$.
		However, $\varphi((1,0) + (0,1)) = \varphi(1,1) = \sqrt{2}$.
		Since $2 \neq \sqrt{2}$, we have shown $\varphi$ is not additive; it remains to show that it is still linear with respect to scalar multiplication.
		Let $a \in \R$.

		I forgot that I had yet to complete this and did not complete it on time.
		\end{proof}
		% subsection problem_8 (end)

		\subsection*{Problem 10}
		Suppose $U$ is a subspace of $V$ with $U \neq V$.
		Suppose $S \in \Ell(U,W)$ and $S \neq 0$.
		Define $T : V \to W$ by $$Tv = \begin{cases} Sv & \text{if $v \in U$}, \\ 0 & \text{if $v \in V$ and $v \notin U$}. \end{cases}$$
		Prove that $T$ is not a linear map on $V$.

		\begin{proof}
		Let $U,V,S,T$ be as above.
		Further, let $u \in U$ and $v \in V$ such that $v \notin U$, notice that $u+v \in V$ but $u+v \notin U$.
		Then, $T(u) = Sv \neq 0$ and $T(v) = 0$, so $T(u) + T(v) = Sv$.
		But $T(u+v) = 0$, and $Sv \neq 0$, so $T$ is not additive, hence it is not linear on $V$.
		\end{proof}
		% subsection problem_10 (end)
	% section section_3_a (end)

	\section*{Section 3.B}
		\subsection*{Problem 4}
		Show that $$\{ T \in \Ell(\R^{5}, \R^{4}) : \dim \null T > 2 \}$$ is not a subspace of $\Ell(\R^{5}, \R^{4})$.

		\begin{proof}[Solution]
		Let $W = \{ T \in \Ell(\R^{5}, \R^{4}) : \dim \null T > 2 \}$, we will show that $W \subsetneq \Ell(\R^{5}, \R^{4})$ is not a subspace by showing that it is not closed under addition.
		Let $e_{1}, \dots, e_{5}$ be a basis for $\R^{5}$, and let $\ep_{1}, \dots, \ep_{4}$ be a basis for $\R^{4}$.
		Since the dimension of the null space for elements of $W$ must be either 3, 4, or 5, we will construct two linear maps $S,T$ with $\dim \null S = \dim \null T = 3$, and show that distributing the standard basis elements above across $S+T$ will yield linear maps with a dimension two null space.

		Let $S,T$ be linear maps such that $Se_{i} = \ep_{i}$ for $i = 1,2$; $Se_{i} = 0$ for $i = 3,4,5$; $Te_{i} = \ep_{i}$ for $i = 2,3$; and $Te_{i} = 0$ for $i = 1,4,5$.
		Notice that $\dim \null S = \dim \null T = 3$, so $S,T \in W$.
		Consider the linear map $(S+T)$; we compute the following:
			\begin{align*}
				(S + T)e_{1} &= Se_{1} + Te_{1}, \\
				&= \ep_{1}; \\
				(S + T)e_{2} &= Se_{2} + Te_{2}, \\
				&= 2\ep_{2}; \\
				(S + T)e_{3} &= Se_{3} + Te_{3}, \\
				&= \ep_{3}; \\
				(S + T)e_{4} &= Se_{4} + Te_{4}, \\
				&= 0; \\
				(S + T)e_{5} &= Se_{5} + Te_{5}, \\
				&= 0.
			\end{align*}
		Thus that $\dim \range (S+T) = 3$ and $\dim \null (S+T) = 2$, so $(S+T) \notin W$, and $W$ is not closed under addition.
		Hence, $W \subsetneq \Ell(\R^{5},\R^{4})$ is not a subspace.

		Notice that we could replicate this `partial-overlapping' strategy with linear maps $S,T$ such that $\dim \null S = \dim \null T = 4$ and create $(S+T)$ such that $\dim \null (S+T) = 1$, but not with linear maps $S'$ such that $\dim \null S' = 5$, because these linear maps are just the zero map, and so they have no issues with additivity.
		\end{proof}
		% subsection problem_4 (end)

		\subsection*{Problem 9}
		Suppose $T \in \Ell(V,W)$ is injective and $v_{1}, \dots, v_{n}$ is linearly independent in $V$.
		Prove that $Tv_{1}, \dots, Tv_{n}$ is linearly independent in $W$.

		\begin{proof}
		Let $T, V, W; v_{1}, \dots, v_{n}$ be as above.
		Since $v_{1}, \dots, v_{n}$ is linearly independent in $V$, we know that $0 = a_{1}v_{1} + \cdots + a_{n}v_{n}$ if and only if each $a_{i} = 0$ for $1 \leq i \leq n$.
		Recall that since $T$ is injective, we have $\null T = \{ 0 \}$, thus, $T(0) = 0$ is a unique mapping.
		Apply $T$ to $0 = a_{1}v_{1} + \cdots + a_{n}v_{n}$:
			\begin{align*}
				0_{V} &= a_{1}v_{1} + \cdots + a_{n}v_{n}, \\
				T(0_{V}) &= T(a_{1}v_{1} + \cdots + a_{n}v_{n}), \\
				0_{W} &= a_{1}T(v_{1}) + \cdots + a_{n}T(v_{n}).
			\end{align*}
		Recall that each $a_{i} = 0$ for $1 \leq i \leq n$, and since the mapping $T(0)=0$ is unique, these $a_{i}$'s are the only way to write $Tv_{1}, \dots, Tv_{n}$ as a homogeneous linear combination, hence $Tv_{1}, \dots, Tv_{n}$ is linearly independent, and since $Tv_{1}, \dots, Tv_{n} \in \range T$ and $\range T \subset W$, we have that $Tv_{1}, \dots, Tv_{n}$ is linearly independent in $W$.
		\end{proof}
		% subsection problem_9 (end)

		\subsection*{Problem 31}
		Give an example of two linear maps $T_{1}$ and $T_{2}$ from $\R^{5}$ to $\R^{2}$ that have the same null space but are such that $T_{1}$ is not a scalar multiple of $T_{2}$.

		\begin{proof}[Solution]
		Let $e_{1}, \dots, e_{5}$ be a basis for $\R^{5}$ and $\ep_{1}, \ep_{2}$ be a basis for $\R^{2}$; further, let $T_{1}, T_{2} \in \Ell(\R^{5}, \R^{2})$ such that:
			\begin{figure}[h]
				\begin{tabular}{ll}
					$T_{1}e_{1} = \ep_{1}$, & $T_{2}e_{1} = \ep_{2}$, \\
					$T_{1}e_{2} = \ep_{2}$, & $T_{2}e_{2} = \ep_{1}$, \\
					$T_{1}e_{3} = \ep_{1} + \ep_{2}$, & $T_{2}e_{3} = \ep_{2} - \ep_{1}$, \\
					$T_{1}e_{4} = \ep_{1} - \ep_{2}$, & $T_{2}e_{4} = \ep_{2} + \ep_{1}$, \\
					$T_{1}e_{5} = 0$. & $T_{2}e_{5} = 0$.
				\end{tabular}
			\end{figure}

		Consider $2T_{1}e_{1} = 2\ep_{1}$ and $2T_{2}e_{1} = 2\ep_{2}$, clearly these are not scalar multiples of another; there are many other examples such as these.
		\end{proof}
		% subsection problem_31 (end)
	% section section_3_b (end)
\end{document}