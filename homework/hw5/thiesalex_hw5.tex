%!TEX output_directory = temp
\documentclass[letterpaper, 12pt]{amsart}
	%%%%%%%%%%%%%%%%%%%%%%%%%%%%%%%%%%%%%%%%%%%%%%%%%%%%%%%%%%%%%%%%%%%%%%%%%%%%%%
	%%%%%%%%%%%%%%%%%%%%%%%%%%%% boilerplate packages %%%%%%%%%%%%%%%%%%%%%%%%%%%%
	\usepackage{amsmath,amssymb,amsthm}
	\usepackage{marvosym}
	\usepackage[mathscr]{euscript}
	\usepackage{enumerate}
	\usepackage{graphicx}
	\usepackage{mathrsfs}
	\usepackage{color}
	\usepackage{hyperref}
	\usepackage{verbatim}
	\usepackage{stmaryrd}
	\usepackage[margin=1.25in]{geometry}

	\raggedbottom

	%%%%%%%%%%%%%%%%%%%%%%%%%%%%%%%%%%%%%%%%%%%%%%%%%%%%%%%%%%%%%%%%%%%%%%%%%%%%%%
	%%%%%%%%%%%%%%%%%%%%%%%%%%%%% rename the abstract %%%%%%%%%%%%%%%%%%%%%%%%%%%%
	% \renewcommand{\abstractname}{Introduction}

	%%%%%%%%%%%%%%%%%%%%%%%%%%%%%%%%%%%%%%%%%%%%%%%%%%%%%%%%%%%%%%%%%%%%%%%%%%%%%%
	%%%%%%%%%%%%%%%%%%%%%%%%%%%%%%%%%%%%% sets %%%%%%%%%%%%%%%%%%%%%%%%%%%%%%%%%%%
		%% sets 
		\DeclareMathOperator{\N}{\mathbb{N}}
		\DeclareMathOperator{\Z}{\mathbb{Z}}
		\DeclareMathOperator{\Zp}{\mathbb{Z}^{+}}
		\DeclareMathOperator{\Q}{\mathbb{Q}}
		\DeclareMathOperator{\Qp}{\mathbb{Q}^{+}}
		\DeclareMathOperator{\Qc}{\mathbb{Q}^{c}}
		\DeclareMathOperator{\R}{\mathbb{R}}
		\DeclareMathOperator{\F}{\mathbb{F}}
		\DeclareMathOperator{\Rp}{\mathbb{R}^{+}}
		\DeclareMathOperator{\C}{\mathbb{C}}
		\DeclareMathOperator{\Cnon}{\mathbb{C}^{\times}}
		%% powerset of a set
		\DeclareMathOperator{\pset}{\mathcal{P}}
		%% set of continuous functions in a certain variable
		\DeclareMathOperator{\cont}{\mathscr{C}}
		%% set of functions in a certain variable
		\DeclareMathOperator{\func}{\mathscr{F}}

		\renewcommand{\null}{\text{null }}
		\DeclareMathOperator{\range}{\text{range }}
		
	%%%%%%%%%%%%%%%%%%%%%%%%%%%%%%%%%%%%%%%%%%%%%%%%%%%%%%%%%%%%%%%%%%%%%%%%%%%%%%
	%%%%%%%%%%%%%%%%%%%%%%%%%%%%%%%% linear algebra %%%%%%%%%%%%%%%%%%%%%%%%%%%%%%
		%% linear span
		\DeclareMathOperator{\Ell}{\mathscr{L}}
		%% bold vectors with arrows, and bold matrices
		\newcommand{\bmat}[1]{{\mathbf{#1}}}
		\newcommand{\bvec}[1]{{\vec{\mathbf{#1}}}}
		%% independent vectors/matrices
		\DeclareMathOperator{\ind}{\perp\!\!\!\perp}
		%% order
		\DeclareMathOperator{\ord}{\text{ord}}

	%%%%%%%%%%%%%%%%%%%%%%%%%%%%%%%%%%%%%%%%%%%%%%%%%%%%%%%%%%%%%%%%%%%%%%%%%%%%%%
	%%%%%%%%%%%%%%%%%%%%%%%%%%% probability & statistics %%%%%%%%%%%%%%%%%%%%%%%%%
		%% probability, expectation, variance, etc.
		\renewcommand{\Pr}{\mathbb{P}}
		\DeclareMathOperator{\E}{\mathbb{E}}
		\DeclareMathOperator{\var}{\rm Var}
		\DeclareMathOperator{\sd}{\rm SD}
		\DeclareMathOperator{\cov}{\rm Cov}
		\DeclareMathOperator{\SE}{\rm SE}
		\DeclareMathOperator{\ssreg}{{\rm SS}_{{\rm Reg}}}
		\DeclareMathOperator{\ssr}{{\rm SS}_{{\rm Res}}}
		\DeclareMathOperator{\sst}{{\rm SS}_{{\rm Tot}}}

	%%%%%%%%%%%%%%%%%%%%%%%%%%%%%%%%%%%%%%%%%%%%%%%%%%%%%%%%%%%%%%%%%%%%%%%%%%%%%%
	%%%%%%%%%%%%%%%%%%%%%%%%%%%%%%%% congruences %%%%%%%%%%%%%%%%%%%%%%%%%%%%%%%%%
		\renewcommand{\mod}[1]{\ (\mathrm{mod}\ #1)}

	%%%%%%%%%%%%%%%%%%%%%%%%%%%%%%%%%%%%%%%%%%%%%%%%%%%%%%%%%%%%%%%%%%%%%%%%%%%%%%
	%%%%%%%%%%%%%%%%%%%%%%%%%%%%%% bracket notation %%%%%%%%%%%%%%%%%%%%%%%%%%%%%%
		% I first used this for principal ideals, that is why the abbreviation is pid
		\newcommand{\pid}[1]{\langle #1 \rangle}

	%%%%%%%%%%%%%%%%%%%%%%%%%%%%%%%%%%%%%%%%%%%%%%%%%%%%%%%%%%%%%%%%%%%%%%%%%%%%%%
	%%%%%%%%%%%%%%%%%%%%%%%%%%%%%%% fatdot notation %%%%%%%%%%%%%%%%%%%%%%%%%%%%%%
		\makeatletter
			\newcommand*\fatdot{\mathpalette\fatdot@{.5}}
			\newcommand*\fatdot@[2]{\mathbin{\vcenter{\hbox{\scalebox{#2}{$\m@th#1\bullet$}}}}}
		\makeatother

	%%%%%%%%%%%%%%%%%%%%%%%%%%%%%%%%%%%%%%%%%%%%%%%%%%%%%%%%%%%%%%%%%%%%%%%%%%%%%%
	%%%%%%%%%%%%%%%%%%%%%%%%%%%%%% use pretty letters %%%%%%%%%%%%%%%%%%%%%%%%%%%%
		\DeclareMathOperator{\ep}{\varepsilon}
		\DeclareMathOperator{\ph}{\varphi}

	%%%%%%%%%%%%%%%%%%%%%%%%%%%%%%%%%%%%%%%%%%%%%%%%%%%%%%%%%%%%%%%%%%%%%%%%%%%%%%
	%%%%%%%%%%%%%%%%%%%%%%%%%%% stolen from Jeske/Dugger %%%%%%%%%%%%%%%%%%%%%%%%%
	% Some theorem-like environments, all numbered together starting at 1
	% in each section.

	% The default \theoremstyle is bold headings and italic body text.
	\newtheorem{thm}{Theorem}[section]
	\newtheorem{defn}[thm]{Definition}
	\newtheorem{prop}[thm]{Proposition}
	\newtheorem{claim}[thm]{Claim}
	\newtheorem{cor}[thm]{Corollary}
	\newtheorem{lemma}[thm]{Lemma}

	\theoremstyle{definition}  % Bold headings and Roman body text.
	\newtheorem{example}[thm]{Example}
	\newtheorem{examples}[thm]{Examples}
	\newtheorem{exercise}[thm]{Exercise}
	\newtheorem{note}[thm]{Note}
	\newtheorem{remark}[thm]{Remark}
	\newtheorem{remarks}[thm]{Remarks}
	\newtheorem{discussion}[thm]{Discussion}

	\newcommand{\dfn}{\textbf} % Make defined words bold.
	\newcommand{\mdfn}[1]{\dfn{\mathversion{bold}#1}} % Even make math symbols bold

	% Various commands that are useful.  Please add your own.

	\DeclareMathOperator{\Arg}{Arg}
	\DeclareMathOperator{\re}{Re}
	\DeclareMathOperator{\im}{Im}
	\DeclareMathOperator{\Log}{Log}
	\DeclareMathOperator{\Span}{Span}

	\newcommand{\iso}{\cong}						% isometric/congruent
	\newcommand{\ra}{\rightarrow}                   % right arrow
	\newcommand{\Ra}{\Rightarrow}                   % right implies
	\newcommand{\lra}{\longrightarrow}              % long right arrow
	\newcommand{\la}{\leftarrow}                    % left arrow
	\newcommand{\La}{\Leftarrow}                    % left implies
	\newcommand{\lla}{\longleftarrow}               % long left arrow
	\newcommand{\llra}[1]{\stackrel{#1}{\lra}}      % labeled long right arrow
	\newcommand{\we}{\llra{\sim}}                   % weak equivalence
	\newcommand{\cof}{\rightarrowtail}              % cofibration
	\newcommand{\fib}{\twoheadrightarrow}           % fibration
	\newcommand{\inc}{\hookrightarrow}              % inclusion
	\newcommand{\dbra}{\rightrightarrows}           % double arrow for equalizer diagrams
	\newcommand{\eqra}{\llra{\sim}}                 % equivalence/isomorphism

	% \newcommand{\blank}{\underbar{\ \ }}          % An underscore, as in (__)xV
	\newcommand{\blank}{-}                          % A hyphen, as in (-)xV
	\newcommand{\Id}{Id}                            % The identity functor
	\newcommand{\und}{\underline}
	\newcommand{\norm}[1]{\mid \!\!#1 \!\!\mid}             %\norm{x} gives |x|

	% These commands are for the period and comma in the lower right entry of
	% a diagram.  They put the punctuation 2 pts to the right, but make
	% TeX (and hence the diagram package) unaware of the extra width
	% of that entry.
	\newcommand{\period}    {{\makebox[0pt][l]{\hspace{2pt} .}}}
	\newcommand{\comma}     {{\makebox[0pt][l]{\hspace{2pt} ,}}}
	\newcommand{\semicolon} {{\makebox[0pt][l]{\hspace{2pt} ;}}}

	\newcommand{\Cech}{\v{C}ech}
	\newcommand{\scat}{\Delta}
	\newcommand{\assign}{\ra}
	\newcommand{\copr}{\,\amalg\,}
	\newcommand{\ovcat}{\downarrow}
	\newcommand{\pder}[2]{{\frac{\partial #1}{\partial #2}}}
	\newcommand{\del}{\nabla}
	\newcommand{\vectr}[1]{{\mbox{\boldmath $#1$}}}
	\newcommand{\uvectr}[1]{\vectr{\hat #1}}
	\newcommand{\ihat}{\uvectr \imath}
	\newcommand{\jhat}{\uvectr \jmath}
	\newcommand{\khat}{\uvectr k}
	\newcommand{\rhat}{\uvectr r}
	\newcommand{\thhat}{\uvectr \theta}
	\newcommand{\zhat}{\uvectr z}
	\newcommand{\rhohat}{\uvectr \rho}
	\newcommand{\phihat}{\uvectr \phi}
	\newcommand{\grad}{\vectr{\vec\nabla}}
	% \newcommand{\R}{\mathbb{R}}
	\newcommand{\vv}[1]{\vectr{v_{#1}}}
	\newcommand{\crad}{0.1}
	\newcommand{\lline}[1]{\overleftrightarrow{#1}}
	\DeclareMathOperator{\area}{area}
	\DeclareMathOperator{\vol}{vol}
	\newcommand{\ray}[1]{\overset{\rightarrow}{#1}}
	\newcommand{\sr}[2]{???}
	\newcommand{\iihat}{i}
	\newcommand{\jjhat}{j}
	\newcommand{\kkhat}{k}

		
\begin{document}
	\title{Homework 5  -- Math 441 \\ \today}
	\author{Alex Thies \\ \href{mailto:athies@uoregon.edu}{\lowercase{athies$@$uoregon.edu}}}

	\maketitle

	Assignment: 3.B - 6, 16, 21; 3.C - 4, 10, 14; 3.D - TBA;

	\section*{Section 3.B}
		\subsection*{Problem 6}
		Prove that there does not exist a linear map $T : \R^{5} \to \R^{5}$ such that $$\range T = \null T.$$

		\begin{proof}
		Let $T \in \Ell(\R^{5}, \R^{5})$, i.e., $T : \R^{5} \to \R^{5}$.
		By the Fundamental Theoreom for Linear Maps (FTLM) we have $\dim \R^{5} = \dim \null T + \, \dim \range T$.
		Suppose by way of contradiction that $\range T = \null T$, then $\dim \null T = \dim \range T$, so the FTLM states $5 = 2 \dim \null T$.
		Since $\dim \null T \in \Z^{+}$, the FTLM is asserting that $5 = 2n$, i.e., $5$ is even$\, \lightning$		
		Hence, there does not exist a linear map $T : \R^{5} \to \R^{5}$ such that $\range T = \null T$, as we aimed to show.
		\end{proof}
		% subsection problem_6 (end)

		\subsection*{Problem 16}
		Suppose there exists a linear map on $V$ whose null space and range are both finite-dimensional. 
		Prove that $V$ is finite-dimensional.

		\begin{proof}
		Let $T \in \Ell(\mathbf{V},\mathbf{W})$ such that $\dim \null T = m$ and $\dim \range T = n$ for $m,n \in \Z^{+}$.
		Let $u_{1}, \dots, u_{m}$ be a basis of $\null T$ and $w_{1}, \dots, w_{n}$ be a basis of $\range T$.
		Since $T$ is surjective, $\range T = \mathbf{W}$, so $w_{1}, \dots, w_{n}$ is also a basis of $\mathbf{W}$.
		Moreover, since $T$ is surjective, for each $w \in \mathbf{W}$, there exists $v \in \mathbf{V}$ such that $Tv = w$, hence we can write our basis of $W$ as $Tv_{1}, \dots, Tv_{n}$ for $v_{j} \in V$.
		Then $Tv = a_{1}w_{1} + \cdots + a_{n}w_{n} = a_{1}Tv_{1} + \cdots + a_{n}Tv_{n}$.
		So, with additivity and homogeneity we can compute the following:
			\begin{align*}
				Tv &= a_{1}Tv_{1} + \cdots + a_{n}Tv_{n}, \\
				0 &= T(a_{1}v_{1} + \cdots a_{n}v_{n}) - Tv, \\
				&= T(a_{1}v_{1} + \cdots a_{n}v_{n} - v), \\
				&= T(v - (a_{1}v_{1} + \cdots a_{n}v_{n}) ), \\
			\end{align*}
		So, we have that $v - a_{1}v_{1} - \cdots a_{n}v_{n} \in \null T$, thus we can write it as a linear combination of the basis elements $u_{i}$. 
		So we have:
		\begin{align*}
			v - (a_{1}v_{1} + \cdots a_{n}v_{n}) &= b_{1}u_{1} + \cdots b_{n}u_{m}, \\
			v &= b_{1}u_{1} + \cdots b_{n}u_{m} + a_{1}v_{1} + \cdots a_{n}v_{n}
		\end{align*}
		Notice that since $\null T \subsetneq \mathbf{V}$, each of the $u_{i}$'s are elements of $\mathbf{V}$.
		Thus, an arbitrarily chosen element of $\mathbf{V}$ can be expressed as a linear combination of finitely-many basis elements, which implies that $\mathbf{V}$ is finite-dimensional, as we aimed to show.
		\end{proof}
		% subsection problem_16 (end)

		\subsection*{Problem 21}
		Suppose $V$ is finite-dimensional and $T \in \Ell(V,W)$. 
		Prove that $T$ is surjective if and only if there exists $S \in \Ell(W,V)$ such that $TS$ is the identity map on $W$.

		\begin{proof}
		Let $\mathbf{V}$ be a finite-dimensional vector space over $\F$ and let $T \in \Ell(\mathbf{V},\mathbf{W})$ for some vector space $W$; let $\dim \mathbf{V} = n$ for $n \in \Z^{+}$.

		$\Ra)$ Assume $T$ is surjecitive.
		Then $\range T = \mathbf{W}$, but more importantly, we have that for each $w \in \mathbf{W}$, there exists $v \in \mathbf{V}$ such that $Tv = w$.
		Define $S : \mathbf{W} \to \mathbf{V}$ mapped by $w \mapsto v$ where $v$ is such that $Tv = w$.
		Since $T$ is surjective, $S$ is well-defined.
		Then we have $TSw = Tv = w$, which shows that $TS$ acts as the identity element from $\mathbf{W}$, as we aimed to show; it remains to prove the converse.

		$\La)$ Assume there exists $S \in \Ell(\mathbf{W},\mathbf{V})$ such that $TS$ is the identity map on $\mathbf{W}$, we will show that $T$ is surjective, using the definition of surjectivity\footnote{I tried to show that $\range T = \textbf{W}$ for awhile, and that was hard; Shida suggested that we just use the definition.}
		Let $w \in \mathbf{W}$, and note that $S$ in this part of the proof is \textbf{not} defined as it is previously.\footnote{This is because the previous definition of $S$ utilized the fact that $T$ is surjective.}
		Notice that $Sw \in V$, hence for some $v \in \mathbf{V}$ we know $Sw = v$.
		By assumption $TSw = w$, but from the previous line we know that $Sw \in \mathbf{V}$ is the element that $T$ maps to $w$, and since $w$ is arbitrary, we have that $Tv = w$ for each $w \in \mathbf{W}$, which means that $T$ satisfies the definition of being a surjective mapping, as we aimed show.
		\end{proof}
		% subsection problem_21 (end)
	% section section_3_b (end)

	\section*{Section 3.C}
		\subsection*{Problem 4}
		Suppose $v_{1}, \dots, v_{m}$ is a basis of $V$ and $W$ is finite-dimensional. 
		Suppose $T \in \Ell(V,W)$. 
		Prove that there exists a basis $w_{1}, \dots, w_{m}$ of $W$ such that all the entries in the first column of $\mathcal{M}(T)$ (with respect to the bases $v_{1}, \dots, v_{m}$ and $w_{1}, \dots, w_{m}$) are 0 except for possibly a 1 in the first row, first column.

		\begin{proof}
		Let $v_{1}, \dots, v_{m}$, $V$, $W$, and $T$ be as above.
		I think the idea behind this problem is to show that matrices can be row-reduced, so we need to make a basis $w_{1}, \dots, w_{m}$ where $Tv_{1} = w_{1}$.
		So, we have two cases: (1) $Tv_{1} = 0$; and (2) $Tv_{1} \neq 0$.
		If $Tv_{1} = 0$, then any $w_{1}, \dots, w_{m}$ will work fine.
		If $Tv_{1} \neq 0$, choose any $w_{1}, \dots, w_{m}$ so that $Tv_{1} = w_{1}$, as we alluded to above.
		\end{proof}
		% subsection problem_4 (end)

		\subsection*{Problem 10}
		Suppose $A$ is an $m$-by-$n$ matrix and $C$ is an $n$-by-$p$ matrix. 
		Prove that $$(AC)_{j,\cdot}$$
		In other words, show that row $j$ of $AC$ equals (row $j$ of $A$) times $C$.

		\begin{proof}
		The notation for this problem was very cumbersome, so I wasn't able to come up with a good, clean solution.
		\end{proof}
		% subsection problem_10 (end)

		\subsection*{Problem 14}
		Prove that matrix multiplication is associative. 
		In other words, suppose $A$, $B$, and $C$ are matrices whose sizes are such that $(AB)C$ makes sense. Prove that $A(BC)$ makes sense and that $(AB)C = A(BC)$.

		\begin{proof}
		As a consequence of my previous abstract algebra coursework, I hate proving that something is associative directly from the definition, its tedious and error-prone, so I tried to avoid that here.
		Recall that multiplication of linear maps is associative, and that linear maps can always be expressed as a matrix, we just need to ensure that the maps we're choosing have the appropriate dimensions.
		Let $T \in \Ell(\F^n,\F^m)$, $S \in \Ell(\F^m,\F^p)$, and $R \in \Ell(\F^p,\F^q)$.
		Next, let $\mathcal{M}(T) = A$, $\mathcal{M}(S) = B$, and $\mathcal{M}(R) = C$.
		Recall again that by Theorem 3.43 $\mathcal{M}(ST) = \mathcal{M}(S)\mathcal{M}(T)$, using this we compute the following:
			\begin{align*}
			(AB)C &= (\mathcal{M}(T)\mathcal{M}(S))\mathcal{M}(R), \\
			&= \mathcal{M}(TS)\mathcal{M}(R), \\
			&= \mathcal{M}((TS)R), \\
			&= \mathcal{M}(TSR), \\
			&= \mathcal{M}(T(SR)), \\
			&= \mathcal{M}(T)\mathcal{M}(SR), \\
			&= \mathcal{M}(T)(\mathcal{M}(S)\mathcal{M}(R)), \\
			&= A(BC).
			\end{align*}
		Thus, $(AB)C = A(BC)$ as we aimed to show.
			
		\end{proof}
		% subsection problem_14 (end)
	% section section_3_c (end)
\end{document}