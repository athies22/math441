%!TEX output_directory = temp
\documentclass[letterpaper, 12pt]{amsart}
	%%%%%%%%%%%%%%%%%%%%%%%%%%%%%%%%%%%%%%%%%%%%%%%%%%%%%%%%%%%%%%%%%%%%%%%%%%%%%%
	%%%%%%%%%%%%%%%%%%%%%%%%%%%% boilerplate packages %%%%%%%%%%%%%%%%%%%%%%%%%%%%
		\usepackage[margin=2in]{geometry}
		\usepackage{amsmath,amssymb,amsthm}
		\usepackage{marvosym}
		\usepackage[mathscr]{euscript}
		\usepackage{enumerate}
		\usepackage{graphicx}
		\usepackage{mathrsfs}
		\usepackage{color}
		\usepackage{hyperref}
		\usepackage{verbatim}
		\usepackage{stmaryrd}

	%%%%%%%%%%%%%%%%%%%%%%%%%%%%%%%%%%%%%%%%%%%%%%%%%%%%%%%%%%%%%%%%%%%%%%%%%%%%%%
	%%%%%%%%%%%%%%%%%%%%%%%%%%%%% rename the abstract %%%%%%%%%%%%%%%%%%%%%%%%%%%%
		\renewcommand{\abstractname}{Assignment}

	%%%%%%%%%%%%%%%%%%%%%%%%%%%%%%%%%%%%%%%%%%%%%%%%%%%%%%%%%%%%%%%%%%%%%%%%%%%%%%
	%%%%%%%%%%%%%%%%%%%%%%%%%%%%%%%%%%%%% sets %%%%%%%%%%%%%%%%%%%%%%%%%%%%%%%%%%%
		\DeclareMathOperator{\N}{\mathbb{N}}				% natural numbers
		\DeclareMathOperator{\Z}{\mathbb{Z}}				% integers
		\DeclareMathOperator{\Zp}{\mathbb{Z}^{+}}			% positive integers
		\DeclareMathOperator{\Q}{\mathbb{Q}}				% rationals
		\DeclareMathOperator{\Qc}{\mathbb{Q}^{c}}			% irrationals
		\DeclareMathOperator{\R}{\mathbb{R}}				% reals
		\DeclareMathOperator{\F}{\mathbb{F}}				% a field
		\DeclareMathOperator{\C}{\mathbb{C}}				% complex numbers
		\DeclareMathOperator{\Cnon}{\mathbb{C}^{\times}}	% nonzero complex numbers
		\DeclareMathOperator{\Pcal}{\mathcal{P}}			% powerset, or set of polynomials
		\DeclareMathOperator{\Ell}{\mathscr{L}}				% set of linear maps, or linear operator

	%%%%%%%%%%%%%%%%%%%%%%%%%%%%%%%%%%%%%%%%%%%%%%%%%%%%%%%%%%%%%%%%%%%%%%%%%%%%%%
	%%%%%%%%%%%%%%%%%%%%%%%%%%%%%% use pretty letters %%%%%%%%%%%%%%%%%%%%%%%%%%%%
		\DeclareMathOperator{\ep}{\varepsilon}				% epsilons
		\DeclareMathOperator{\ph}{\varphi}					% phis

	%%%%%%%%%%%%%%%%%%%%%%%%%%%%%%%%%%%%%%%%%%%%%%%%%%%%%%%%%%%%%%%%%%%%%%%%%%%%%%
	%%%%%%%%%%%%%%%%%%%%%%%%%%%%%%%%%%% algebra %%%%%%%%%%%%%%%%%%%%%%%%%%%%%%%%%%
		\renewcommand{\null}{\text{null }}					% null space
		\DeclareMathOperator{\range}{\text{range }}			% range
		\newcommand{\bmat}[1]{{\mathbf{#1}}}				% bold matrix
		\newcommand{\bvec}[1]{{\vec{\mathbf{#1}}}}			% bold vector
		\DeclareMathOperator{\ind}{\perp\!\!\!\perp}		% perpendicular, orthogonal
		\DeclareMathOperator{\ord}{\text{ord}}				% order of a structure
		\DeclareMathOperator{\Log}{Log}						% logarithm
		\DeclareMathOperator{\Span}{Span}					% span
		\newcommand{\pid}[1]{\langle #1 \rangle}			% bracket notation, used for 
															% ideals or inner products
		\newcommand{\norm}[1]{\mid \!\!#1 \!\!\mid}			%\norm{x} gives |x|

		% fatdot notation
		\makeatletter
			\newcommand*\fatdot{\mathpalette\fatdot@{.5}}
			\newcommand*\fatdot@[2]{\mathbin{\vcenter{\hbox{\scalebox{#2}{$\m@th#1\bullet$}}}}}
		\makeatother

	%%%%%%%%%%%%%%%%%%%%%%%%%%%%%%%%%%%%%%%%%%%%%%%%%%%%%%%%%%%%%%%%%%%%%%%%%%%%%%
	%%%%%%%%%%%%%%%%%%%%%%%%%%% probability & statistics %%%%%%%%%%%%%%%%%%%%%%%%%
		\renewcommand{\Pr}{\mathbb{P}}						% probability
		\DeclareMathOperator{\E}{\mathbb{E}}				% expectation
		\DeclareMathOperator{\var}{\rm Var}					% variance
		\DeclareMathOperator{\sd}{\rm SD}					% standard deviation
		\DeclareMathOperator{\cov}{\rm Cov}					% covariance
		\DeclareMathOperator{\SE}{\rm SE}					% standard error
		\DeclareMathOperator{\ssreg}{{\rm SS}_{{\rm Reg}}}	% sum of squared regression
		\DeclareMathOperator{\ssr}{{\rm SS}_{{\rm Res}}}	% sum of squared residuals
		\DeclareMathOperator{\sst}{{\rm SS}_{{\rm Tot}}}	% total sum of squares

	%%%%%%%%%%%%%%%%%%%%%%%%%%%%%%%%%%%%%%%%%%%%%%%%%%%%%%%%%%%%%%%%%%%%%%%%%%%%%%
	%%%%%%%%%%%%%%%%%%%%%%%%%%%%%%% number theory %%%%%%%%%%%%%%%%%%%%%%%%%%%%%%%%
		\renewcommand{\mod}[1]{\ (\mathrm{mod}\ #1)}		% congruences

	%%%%%%%%%%%%%%%%%%%%%%%%%%%%%%%%%%%%%%%%%%%%%%%%%%%%%%%%%%%%%%%%%%%%%%%%%%%%%%
	%%%%%%%%%%%%%%%%%%%%%%%%%%%% theorem environments %%%%%%%%%%%%%%%%%%%%%%%%%%%%
		% Some theorem-like environments, all numbered together starting at 1
		% in each section.

		\newtheorem{thm}{Theorem}[section]					% The default \theoremstyle is 
		\newtheorem{defn}[thm]{Definition}					% bold headings and italic body text.
		\newtheorem{prop}[thm]{Proposition}
		\newtheorem{claim}[thm]{Claim}
		\newtheorem{cor}[thm]{Corollary}
		\newtheorem{lemma}[thm]{Lemma}

		\theoremstyle{definition}  							% Bold headings and Roman body text.
		\newtheorem{example}[thm]{Example}
		\newtheorem{examples}[thm]{Examples}
		\newtheorem{exercise}[thm]{Exercise}
		\newtheorem{note}[thm]{Note}
		\newtheorem{remark}[thm]{Remark}
		\newtheorem{remarks}[thm]{Remarks}
		\newtheorem{discussion}[thm]{Discussion}

		\newcommand{\dfn}{\textbf} 							% Make defined words bold.
		\newcommand{\mdfn}[1]{\dfn{\mathversion{bold}#1}} 	% Even make math symbols bold

	%%%%%%%%%%%%%%%%%%%%%%%%%%%%%%%%%%%%%%%%%%%%%%%%%%%%%%%%%%%%%%%%%%%%%%%%%%%%%%
	%%%%%%%%%%%%%%%%%%%%%%%%%%%%%%% complex numbers %%%%%%%%%%%%%%%%%%%%%%%%%%%%%%
		\DeclareMathOperator{\Arg}{Arg}						% argument of z \in \C
		\DeclareMathOperator{\re}{Re}						% real component
		\DeclareMathOperator{\im}{Im}						% imaginary component

	%%%%%%%%%%%%%%%%%%%%%%%%%%%%%%%%%%%%%%%%%%%%%%%%%%%%%%%%%%%%%%%%%%%%%%%%%%%%%%
	%%%%%%%%%%%%%%%%%%%%%%%%%%%%%%% various symbols %%%%%%%%%%%%%%%%%%%%%%%%%%%%%%
		\newcommand{\iso}{\cong}						% isometric/congruent
		\newcommand{\ra}{\rightarrow}                   % right arrow
		\newcommand{\Ra}{\Rightarrow}                   % right implies
		\newcommand{\lra}{\longrightarrow}              % long right arrow
		\newcommand{\la}{\leftarrow}                    % left arrow
		\newcommand{\La}{\Leftarrow}                    % left implies
		\newcommand{\lla}{\longleftarrow}               % long left arrow
		\newcommand{\eqra}{\llra{\sim}}                 % equivalence/isomorphism
		\newcommand{\blank}{\underbar{\ \ }}          	% An underscore, as in (__)xV
		% \newcommand{\blank}{-}                          % A hyphen, as in (-)xV
		\newcommand{\Id}{Id}                            % The identity functor
		\newcommand{\und}{\underline}
		\newcommand{\del}{\nabla}						% gradient vector

		\raggedbottom		
\begin{document}
	\title{Homework 5  -- Math 441 \\ M\lowercase{ay 9, 2018}}
	\author{Alex Thies \\ \href{mailto:athies@uoregon.edu}{\lowercase{athies$@$uoregon.edu}}}

	\begin{abstract}
	The following exercises are assigned from \textit{Linear Algebra Done Right}, 3rd Edition, by Sheldon Axler. 
			\begin{tabular}{rl}
				& 3.B - 6, 16, 21; \\
				& 3.C - 4, 10, 14.
			\end{tabular}
	\end{abstract}

	\maketitle

	\section*{Section 3.B}
		\subsection*{Problem 6}
		Prove that there does not exist a linear map $T : \R^{5} \to \R^{5}$ such that $$\range T = \null T.$$

		\begin{proof}
		Let $T \in \Ell(\R^{5}, \R^{5})$, i.e., $T : \R^{5} \to \R^{5}$.
		By the Fundamental Theoreom for Linear Maps (FTLM) we have $\dim \R^{5} = \dim \null T + \, \dim \range T$.
		Suppose by way of contradiction that $\range T = \null T$, then $\dim \null T = \dim \range T$, so the FTLM states $5 = 2 \dim \null T$.
		Since $\dim \null T \in \Z^{+}$, the FTLM is asserting that $5 = 2n$, i.e., $5$ is even$\, \lightning$		
		Hence, there does not exist a linear map $T : \R^{5} \to \R^{5}$ such that $\range T = \null T$, as we aimed to show.
		\end{proof}
		% subsection problem_6 (end)

		\subsection*{Problem 16}
		Suppose there exists a linear map on $V$ whose null space and range are both finite-dimensional. 
		Prove that $V$ is finite-dimensional.

		\begin{proof}
		Let $T \in \Ell(\mathbf{V},\mathbf{W})$ such that $\dim \null T = m$ and $\dim \range T = n$ for $m,n \in \Z^{+}$.
		Let $u_{1}, \dots, u_{m}$ be a basis of $\null T$ and $w_{1}, \dots, w_{n}$ be a basis of $\range T$.
		Since $T$ is surjective, $\range T = \mathbf{W}$, so $w_{1}, \dots, w_{n}$ is also a basis of $\mathbf{W}$.
		Moreover, since $T$ is surjective, for each $w \in \mathbf{W}$, there exists $v \in \mathbf{V}$ such that $Tv = w$, hence we can write our basis of $W$ as $Tv_{1}, \dots, Tv_{n}$ for $v_{j} \in V$.
		Then $Tv = a_{1}w_{1} + \cdots + a_{n}w_{n} = a_{1}Tv_{1} + \cdots + a_{n}Tv_{n}$.
		So, with additivity and homogeneity we can compute the following:
			\begin{align*}
				Tv &= a_{1}Tv_{1} + \cdots + a_{n}Tv_{n}, \\
				0 &= T(a_{1}v_{1} + \cdots a_{n}v_{n}) - Tv, \\
				&= T(a_{1}v_{1} + \cdots a_{n}v_{n} - v), \\
				&= T(v - (a_{1}v_{1} + \cdots a_{n}v_{n}) ), \\
			\end{align*}
		So, we have that $v - a_{1}v_{1} - \cdots a_{n}v_{n} \in \null T$, thus we can write it as a linear combination of the basis elements $u_{i}$. 
		So we have:
		\begin{align*}
			v - (a_{1}v_{1} + \cdots a_{n}v_{n}) &= b_{1}u_{1} + \cdots b_{n}u_{m}, \\
			v &= b_{1}u_{1} + \cdots b_{n}u_{m} + a_{1}v_{1} + \cdots a_{n}v_{n}
		\end{align*}
		Notice that since $\null T \subsetneq \mathbf{V}$, each of the $u_{i}$'s are elements of $\mathbf{V}$.
		Thus, an arbitrarily chosen element of $\mathbf{V}$ can be expressed as a linear combination of finitely-many basis elements, which implies that $\mathbf{V}$ is finite-dimensional, as we aimed to show.
		\end{proof}
		% subsection problem_16 (end)

		\subsection*{Problem 21}
		Suppose $V$ is finite-dimensional and $T \in \Ell(V,W)$. 
		Prove that $T$ is surjective if and only if there exists $S \in \Ell(W,V)$ such that $TS$ is the identity map on $W$.

		\begin{proof}
		Let $\mathbf{V}$ be a finite-dimensional vector space over $\F$ and let $T \in \Ell(\mathbf{V},\mathbf{W})$ for some vector space $W$; let $\dim \mathbf{V} = n$ for $n \in \Z^{+}$.

		$\Ra)$ Assume $T$ is surjecitive.
		Then $\range T = \mathbf{W}$, but more importantly, we have that for each $w \in \mathbf{W}$, there exists $v \in \mathbf{V}$ such that $Tv = w$.
		Define $S : \mathbf{W} \to \mathbf{V}$ mapped by $w \mapsto v$ where $v$ is such that $Tv = w$.
		Since $T$ is surjective, $S$ is well-defined.
		Then we have $TSw = Tv = w$, which shows that $TS$ acts as the identity element from $\mathbf{W}$, as we aimed to show; it remains to prove the converse.

		$\La)$ Assume there exists $S \in \Ell(\mathbf{W},\mathbf{V})$ such that $TS$ is the identity map on $\mathbf{W}$, we will show that $T$ is surjective, using the definition of surjectivity\footnote{I tried to show that $\range T = \textbf{W}$ for awhile, and that was hard; Shida suggested that we just use the definition.}
		Let $w \in \mathbf{W}$, and note that $S$ in this part of the proof is \textbf{not} defined as it is previously.\footnote{This is because the previous definition of $S$ utilized the fact that $T$ is surjective.}
		Notice that $Sw \in V$, hence for some $v \in \mathbf{V}$ we know $Sw = v$.
		By assumption $TSw = w$, but from the previous line we know that $Sw \in \mathbf{V}$ is the element that $T$ maps to $w$, and since $w$ is arbitrary, we have that $Tv = w$ for each $w \in \mathbf{W}$, which means that $T$ satisfies the definition of being a surjective mapping, as we aimed show.
		\end{proof}
		% subsection problem_21 (end)
	% section section_3_b (end)

	\section*{Section 3.C}
		\subsection*{Problem 4}
		Suppose $v_{1}, \dots, v_{m}$ is a basis of $V$ and $W$ is finite-dimensional. 
		Suppose $T \in \Ell(V,W)$. 
		Prove that there exists a basis $w_{1}, \dots, w_{m}$ of $W$ such that all the entries in the first column of $\mathcal{M}(T)$ (with respect to the bases $v_{1}, \dots, v_{m}$ and $w_{1}, \dots, w_{m}$) are 0 except for possibly a 1 in the first row, first column.

		\begin{proof}
		Let $v_{1}, \dots, v_{m}$, $V$, $W$, and $T$ be as above.
		I think the idea behind this problem is to show that matrices can be row-reduced, so we need to make a basis $w_{1}, \dots, w_{m}$ where $Tv_{1} = w_{1}$.
		So, we have two cases: (1) $Tv_{1} = 0$; and (2) $Tv_{1} \neq 0$.
		If $Tv_{1} = 0$, then any $w_{1}, \dots, w_{m}$ will work fine.
		If $Tv_{1} \neq 0$, choose any $w_{1}, \dots, w_{m}$ so that $Tv_{1} = w_{1}$, as we alluded to above.
		\end{proof}
		% subsection problem_4 (end)

		\subsection*{Problem 10}
		Suppose $A$ is an $m$-by-$n$ matrix and $C$ is an $n$-by-$p$ matrix. 
		Prove that $$(AC)_{j,\cdot}$$
		In other words, show that row $j$ of $AC$ equals (row $j$ of $A$) times $C$.

		\begin{proof}
		The notation for this problem was very cumbersome, so I wasn't able to come up with a good, clean solution.
		\end{proof}
		% subsection problem_10 (end)

		\subsection*{Problem 14}
		Prove that matrix multiplication is associative. 
		In other words, suppose $A$, $B$, and $C$ are matrices whose sizes are such that $(AB)C$ makes sense. Prove that $A(BC)$ makes sense and that $(AB)C = A(BC)$.

		\begin{proof}
		As a consequence of my previous abstract algebra coursework, I hate proving that something is associative directly from the definition, its tedious and error-prone, so I tried to avoid that here.
		Recall that multiplication of linear maps is associative, and that linear maps can always be expressed as a matrix, we just need to ensure that the maps we're choosing have the appropriate dimensions.
		Let $T \in \Ell(\F^n,\F^m)$, $S \in \Ell(\F^m,\F^p)$, and $R \in \Ell(\F^p,\F^q)$.
		Next, let $\mathcal{M}(T) = A$, $\mathcal{M}(S) = B$, and $\mathcal{M}(R) = C$.
		Recall again that by Theorem 3.43 $\mathcal{M}(ST) = \mathcal{M}(S)\mathcal{M}(T)$, using this we compute the following:
			\begin{align*}
			(AB)C &= (\mathcal{M}(T)\mathcal{M}(S))\mathcal{M}(R), \\
			&= \mathcal{M}(TS)\mathcal{M}(R), \\
			&= \mathcal{M}((TS)R), \\
			&= \mathcal{M}(TSR), \\
			&= \mathcal{M}(T(SR)), \\
			&= \mathcal{M}(T)\mathcal{M}(SR), \\
			&= \mathcal{M}(T)(\mathcal{M}(S)\mathcal{M}(R)), \\
			&= A(BC).
			\end{align*}
		Thus, $(AB)C = A(BC)$ as we aimed to show.
			
		\end{proof}
		% subsection problem_14 (end)
	% section section_3_c (end)
\end{document}