%!TEX output_directory = temp
\documentclass[letterpaper, 12pt]{amsart}
	%%%%%%%%%%%%%%%%%%%%%%%%%%%%%%%%%%%%%%%%%%%%%%%%%%%%%%%%%%%%%%%%%%%%%%%%%%%%%%
	%%%%%%%%%%%%%%%%%%%%%%%%%%%% boilerplate packages %%%%%%%%%%%%%%%%%%%%%%%%%%%%
		\usepackage[margin=2in]{geometry}
		\usepackage{amsmath,amssymb,amsthm}
		\usepackage{marvosym}
		\usepackage[mathscr]{euscript}
		\usepackage{enumerate}
		\usepackage{graphicx}
		\usepackage{mathrsfs}
		\usepackage{color}
		\usepackage{hyperref}
		\usepackage{verbatim}
		\usepackage{stmaryrd}

	%%%%%%%%%%%%%%%%%%%%%%%%%%%%%%%%%%%%%%%%%%%%%%%%%%%%%%%%%%%%%%%%%%%%%%%%%%%%%%
	%%%%%%%%%%%%%%%%%%%%%%%%%%%%% rename the abstract %%%%%%%%%%%%%%%%%%%%%%%%%%%%
		\renewcommand{\abstractname}{Assignment}

	%%%%%%%%%%%%%%%%%%%%%%%%%%%%%%%%%%%%%%%%%%%%%%%%%%%%%%%%%%%%%%%%%%%%%%%%%%%%%%
	%%%%%%%%%%%%%%%%%%%%%%%%%%%%%%%%%%%%% sets %%%%%%%%%%%%%%%%%%%%%%%%%%%%%%%%%%%
		\DeclareMathOperator{\N}{\mathbb{N}}				% natural numbers
		\DeclareMathOperator{\Z}{\mathbb{Z}}				% integers
		\DeclareMathOperator{\Zp}{\mathbb{Z}^{+}}			% positive integers
		\DeclareMathOperator{\Q}{\mathbb{Q}}				% rationals
		\DeclareMathOperator{\Qc}{\mathbb{Q}^{c}}			% irrationals
		\DeclareMathOperator{\R}{\mathbb{R}}				% reals
		\DeclareMathOperator{\F}{\mathbb{F}}				% a field
		\DeclareMathOperator{\C}{\mathbb{C}}				% complex numbers
		\DeclareMathOperator{\Cnon}{\mathbb{C}^{\times}}	% nonzero complex numbers
		\DeclareMathOperator{\Pcal}{\mathcal{P}}			% powerset, or set of polynomials
		\DeclareMathOperator{\Ell}{\mathscr{L}}				% set of linear maps, or linear operator

	%%%%%%%%%%%%%%%%%%%%%%%%%%%%%%%%%%%%%%%%%%%%%%%%%%%%%%%%%%%%%%%%%%%%%%%%%%%%%%
	%%%%%%%%%%%%%%%%%%%%%%%%%%%%%% use pretty letters %%%%%%%%%%%%%%%%%%%%%%%%%%%%
		\DeclareMathOperator{\ep}{\varepsilon}				% epsilons
		\DeclareMathOperator{\ph}{\varphi}					% phis

	%%%%%%%%%%%%%%%%%%%%%%%%%%%%%%%%%%%%%%%%%%%%%%%%%%%%%%%%%%%%%%%%%%%%%%%%%%%%%%
	%%%%%%%%%%%%%%%%%%%%%%%%%%%%%%%%%%% algebra %%%%%%%%%%%%%%%%%%%%%%%%%%%%%%%%%%
		\renewcommand{\null}{\text{null }}					% null space
		\DeclareMathOperator{\range}{\text{range }}			% range
		\newcommand{\bmat}[1]{{\mathbf{#1}}}				% bold matrix
		\newcommand{\bvec}[1]{{\vec{\mathbf{#1}}}}			% bold vector
		\DeclareMathOperator{\ind}{\perp\!\!\!\perp}		% perpendicular, orthogonal
		\DeclareMathOperator{\ord}{\text{ord}}				% order of a structure
		\DeclareMathOperator{\Log}{Log}						% logarithm
		\DeclareMathOperator{\Span}{Span}					% span
		\newcommand{\pid}[1]{\langle #1 \rangle}			% bracket notation, used for 
															% ideals or inner products
		\newcommand{\norm}[1]{\mid \!\!#1 \!\!\mid}			%\norm{x} gives |x|

		% fatdot notation
		\makeatletter
			\newcommand*\fatdot{\mathpalette\fatdot@{.5}}
			\newcommand*\fatdot@[2]{\mathbin{\vcenter{\hbox{\scalebox{#2}{$\m@th#1\bullet$}}}}}
		\makeatother

	%%%%%%%%%%%%%%%%%%%%%%%%%%%%%%%%%%%%%%%%%%%%%%%%%%%%%%%%%%%%%%%%%%%%%%%%%%%%%%
	%%%%%%%%%%%%%%%%%%%%%%%%%%% probability & statistics %%%%%%%%%%%%%%%%%%%%%%%%%
		\renewcommand{\Pr}{\mathbb{P}}						% probability
		\DeclareMathOperator{\E}{\mathbb{E}}				% expectation
		\DeclareMathOperator{\var}{\rm Var}					% variance
		\DeclareMathOperator{\sd}{\rm SD}					% standard deviation
		\DeclareMathOperator{\cov}{\rm Cov}					% covariance
		\DeclareMathOperator{\SE}{\rm SE}					% standard error
		\DeclareMathOperator{\ssreg}{{\rm SS}_{{\rm Reg}}}	% sum of squared regression
		\DeclareMathOperator{\ssr}{{\rm SS}_{{\rm Res}}}	% sum of squared residuals
		\DeclareMathOperator{\sst}{{\rm SS}_{{\rm Tot}}}	% total sum of squares

	%%%%%%%%%%%%%%%%%%%%%%%%%%%%%%%%%%%%%%%%%%%%%%%%%%%%%%%%%%%%%%%%%%%%%%%%%%%%%%
	%%%%%%%%%%%%%%%%%%%%%%%%%%%%%%% number theory %%%%%%%%%%%%%%%%%%%%%%%%%%%%%%%%
		\renewcommand{\mod}[1]{\ (\mathrm{mod}\ #1)}		% congruences

	%%%%%%%%%%%%%%%%%%%%%%%%%%%%%%%%%%%%%%%%%%%%%%%%%%%%%%%%%%%%%%%%%%%%%%%%%%%%%%
	%%%%%%%%%%%%%%%%%%%%%%%%%%%% theorem environments %%%%%%%%%%%%%%%%%%%%%%%%%%%%
		% Some theorem-like environments, all numbered together starting at 1
		% in each section.

		\newtheorem{thm}{Theorem}[section]					% The default \theoremstyle is 
		\newtheorem{defn}[thm]{Definition}					% bold headings and italic body text.
		\newtheorem{prop}[thm]{Proposition}
		\newtheorem{claim}[thm]{Claim}
		\newtheorem{cor}[thm]{Corollary}
		\newtheorem{lemma}[thm]{Lemma}

		\theoremstyle{definition}  							% Bold headings and Roman body text.
		\newtheorem{example}[thm]{Example}
		\newtheorem{examples}[thm]{Examples}
		\newtheorem{exercise}[thm]{Exercise}
		\newtheorem{note}[thm]{Note}
		\newtheorem{remark}[thm]{Remark}
		\newtheorem{remarks}[thm]{Remarks}
		\newtheorem{discussion}[thm]{Discussion}

		\newcommand{\dfn}{\textbf} 							% Make defined words bold.
		\newcommand{\mdfn}[1]{\dfn{\mathversion{bold}#1}} 	% Even make math symbols bold

	%%%%%%%%%%%%%%%%%%%%%%%%%%%%%%%%%%%%%%%%%%%%%%%%%%%%%%%%%%%%%%%%%%%%%%%%%%%%%%
	%%%%%%%%%%%%%%%%%%%%%%%%%%%%%%% complex numbers %%%%%%%%%%%%%%%%%%%%%%%%%%%%%%
		\DeclareMathOperator{\Arg}{Arg}						% argument of z \in \C
		\DeclareMathOperator{\re}{Re}						% real component
		\DeclareMathOperator{\im}{Im}						% imaginary component

	%%%%%%%%%%%%%%%%%%%%%%%%%%%%%%%%%%%%%%%%%%%%%%%%%%%%%%%%%%%%%%%%%%%%%%%%%%%%%%
	%%%%%%%%%%%%%%%%%%%%%%%%%%%%%%% various symbols %%%%%%%%%%%%%%%%%%%%%%%%%%%%%%
		\newcommand{\iso}{\cong}						% isometric/congruent
		\newcommand{\ra}{\rightarrow}                   % right arrow
		\newcommand{\Ra}{\Rightarrow}                   % right implies
		\newcommand{\lra}{\longrightarrow}              % long right arrow
		\newcommand{\la}{\leftarrow}                    % left arrow
		\newcommand{\La}{\Leftarrow}                    % left implies
		\newcommand{\lla}{\longleftarrow}               % long left arrow
		\newcommand{\eqra}{\llra{\sim}}                 % equivalence/isomorphism
		\newcommand{\blank}{\underbar{\ \ }}          	% An underscore, as in (__)xV
		% \newcommand{\blank}{-}                          % A hyphen, as in (-)xV
		\newcommand{\Id}{Id}                            % The identity functor
		\newcommand{\und}{\underline}
		\newcommand{\del}{\nabla}						% gradient vector

		\raggedbottom		
\begin{document}
	\title{Homework 7  -- Math 441 \\ M\lowercase{ay 23, 2018}}
	\author{Alex Thies \\ \href{mailto:athies@uoregon.edu}{\lowercase{athies$@$uoregon.edu}}}

	\begin{abstract}
	The following exercises are assigned from \textit{Linear Algebra Done Right}, 3rd Edition, by Sheldon Axler. 
			\begin{tabular}{rl}
				& 3.E - 7, 13, 16; \\
				& 4 - 4, 5; \\
				& 5.A - 3, 6, 12, 17, 21;
			\end{tabular}
	\end{abstract}

	\maketitle

	\section*{Section 3.E}
		\subsection*{Exercise 7}
		Suppose $v, x$ are vectors in $V$ and $U, W$ are subspaces of $V$ such that $v + U = x + W$. 
		Prove that $U = W$.

		\begin{proof}
		Let $v, x, V, U, W$ be as above.
		Then $v = x + w_{1}$ for some $w_{1} \in W$, and by Theorem 3.85 we have $v - x \in W$.
		Thus, for some $u \in U$ it follows that $v + u = x + w_{2}$ for some $w_{2} \in W$.
		Therefore, we have that $u = (x-v) + w_{2} = -(v-x) + w_{2} = -w_{1}+w_{2} \in W$.
		Since $u$ was chosen arbitrarily, we have shown that $U \subseteq W$, \textit{mutatis mutandis} to show that $W \subseteq U$.
		Hence, by double-inclusion we have shown that $U = W$, as we aimed to do.
		\end{proof}
		% subsection exercise_7 (end)

		\subsection*{Exercise 13}
		Suppose $U$ is a subspace of $V$ and $v_{1} + U, \dots, v_{m} + U$ is a basis of $V/U$ and $u_{1}, \dots, u_{n}$ is a basis of $U$. 
		Prove that $v_{1}, \dots, v_{m},u_{1}, \dots, u_{n}$ is a basis of $V$.

		\begin{proof}
		Let $U, V, v_{1} + U, \dots, v_{m} + U$, and $u_{1}, \dots, u_{n}$ be as above.
		Let $w \in V$, to show that $v_{1}, \dots, v_{m},u_{1}, \dots, u_{n}$ is a basis of $V$, we will show that $w \in \Span(v_{1}, \dots, v_{m},u_{1}, \dots, u_{n})$.
		Since $v_{1} + U, \dots, v_{m} + U$ is a basis of $V/U$, we have $$w + U = a_{1}(v_{1} + U) + \cdots + a_{m}(v_{m}+U).$$ By Theorem 3.85 we have that $w - (a_{1}v_{1} + \cdots + a_{m}v_{m}) \in U$.
		Therefore, we can express this difference in terms of a basis of $U$, i.e., $$w - (a_{1}v_{1} + \cdots + a_{m}v_{m}) = b_{1}u_{1} + \cdots + b_{n}u_{n}.$$ 
		This allows us to write $$w = a_{1}v_{1} + \cdots + a_{m}v_{m} + b_{1}u_{1} + \cdots + b_{n}u_{n}.$$ 
		Therefore, $$w \in \Span(v_{1}, \dots, v_{m},u_{1}, \dots,u_{n}),$$ and $v_{1}, \dots, v_{m},u_{1}, \dots, u_{n}$ is a basis of $V$, as we aimed to show.
		\end{proof}
		% subsection exercise_13 (end)

		\subsection*{Exercise 16}
		Suppose $U$ is a subspace of $V$ such that $\dim V/U = 1$. 
		Prove that there exists $\ph \in \Ell(V, \F)$ such that $\null \ph = U$.

		\begin{proof}
		Let $U,V$ be as above.
		The best way forward will be to define a map from $V/U \to \F$, and then build $\ph$ as the composition of our map with the quotient map $\pi : V \to V/U$.
		Since linear maps are closed under composition, this new composition of maps $\ph$ will be in the vector space $\Ell(V, \F)$, like we want.
		We will have to pick a map from $V/U \to \F$ so that we get the desired property that $\null \ph = U$.

		The fact that $\dim V/U = 1$ tells us that there exists a $v \in V$ where $v \notin U$ and such that $v + U$ is a basis of $V/U$.
		We can send scalar multiples of these $v$ to the scalar in $\F$, and since $v \notin U$, every $u \in U$ will be sent to $0$.
		So, define the linear map $\psi : V/U \to \F$ by the mapping $\lambda v + U \mapsto \lambda$.
		Next, let $\ph = \psi \circ \pi$, notice that $\ph : V \to V/U \to \F$, so $\ph \in \Ell(V,\F)$, like we want.
		We will proceed by double-inclusion to show that $\null \ph = U$.

		\textbf{Step 1} ($\null \ph \subset U$).
		Let $w \in V$ such that $\ph(w) = 0$.
		It follows by the way we created $\ph$, that $w + U = 0v + U$, hence $w \in U$ and $\null \ph \subset U$.

		\textbf{Step 2} ($U \subset \null \ph$).
		Let $u \in U$.
		Then $\pi(u) = u + U = 0 + U = 0v + U$, hence $\psi$ sends $u$ to $0$, i.e. $\ph(u) = 0$, and $u \in \null \ph$.
		Thus, $U \subset \null \ph$, and by the double-inclusion that we have just shown, we have $U = \null \ph$, as desired.
		\end{proof}
		% subsection exercise_16 (end)
	% section section_3_e (end)

	\section*{Section 4}
		\subsection*{Exercise 4}
		Suppose $m$ and $n$ are positive integers with $m \leq n$, and suppose $\lambda_{1}, \dots, \lambda_{m} \in \F$􏰃. 
		Prove that there exists a polynomial $p \in \mathcal{P}(\F)$ with $\deg p = n$ such that $0 = p(\lambda_{1}) = \dots = p(\lambda_{m})$ and such that $p$ has no other zeros.

		\begin{proof}
		Let $m,n$ be as above.
		Define $p: \mathcal{P}(\F) \to \F$ by $p(z) \mapsto (z - \lambda_{1})\cdots(z - \lambda_{m})$.
		Then $p$ has $\lambda_{1}, \dots, \lambda_{m}$ as roots, but we can see that $\deg p = m$, which is too small.
		Let $q = n-m+1$, then modify $p$ so that $\tilde{p}(z) \mapsto (z - \lambda_{1})^{q} \cdots (z - \lambda_{m})$.
		Then we still have exactly $\lambda_{1}, \dots, \lambda_{m}$ as roots, and now with $m-1$ linear factors each with multiplicity 1, and one linear factor with multiplicity $q = n-m+1$, it follows that $\deg \tilde{p} = n$.
		Hence, for $m \leq n$, we have shown that here exists a polynomial $\tilde{p} \in \mathcal{P}_{n}(\F)$ that has exactly $\lambda_{1}, \dots, \lambda_{m}$ as its roots, as we aimed to do.
		\end{proof}
		% subsection exercise_4 (end)

		\subsection*{Exercise 5}
		Suppose $m$ is a nonnegative integer, $z_{1}, \dots, z_{m+1}$ are distinct elements of $\F$, and $w_{1}, \dots, w_{m+1} \in \F$.
		Prove that there exists a unique polynomial $p \in \mathcal{P}_{m}(\F)$ such that $$p(z_{j}) = w_{j}$$ for $j = 1, \dots, m+1$.

		\begin{proof}
		Let $m, z_{1}, \dots, z_{m+1}$, and $w_{1}, \dots, w_{m+1}$ be as above.
		Define $T : \mathcal{P}_{m}(\F) \to \F^{m+1}$ by the mapping $$Tp \mapsto (p(z_{1}), \dots, p(z_{m+1})) = (w_{1}, \dots, w_{m+1}).$$
		We will prove existence and uniqueness by showing that $T$ is a bijection, but first we have to show that $T$ is a linear map.
		To show that $T$ is a linear map we will show that it is closed under addition and scalar multiplication.
		Let $q,r \in \mathcal{P}_{m}(\F)$.
		Then, $T(q+r) = ((q+r)(z_{1}), \dots, (q+r)(z_{m+1}))$, since $(q+r) \in \mathcal{P}_{m}(\F)$, it follows that $T$ is closed under addition.
		Let $\lambda \in \F$.
		Then, $T(\lambda q) = (\lambda q(z_{1}), \dots, \lambda q(z_{m+1})) = \lambda (q(z_{1}), \dots, q(z_{m+1}))$, so we have that $T$ is closed under scalar multiplication, hence $T \in \Ell(\mathcal{P}_{m}(\F), \F^{m+1})$; it remains to show that $T$ is a bijection.

		\textbf{Injective}.
		Because no $p \in \mathcal{P}_{m}(\F)$ has more than $m$ roots, it is impossible for any nonzero $p \in \mathcal{P}_{m}(\F)$ to be mapped to the $(m+1)$tuple of zeros, hence $\null T = \{ 0 \}$.
		This implies that $T$ is injective, and that these polynomials $p$ are unique.

		\textbf{Surjective}.
		By the FTLM we have,
		\begin{align*}
			\dim \range T &= \dim \mathcal{P}_{m}(\F) - \dim \null T, \\
			&= m + 1 - 0, \\
			&= \dim \F^{m+1}.
		\end{align*}
		This implies that $T$ is surjective, which guarantees that each list $w_{1},\dots,w_{m+1} \in \F^{m+1}$ will get hit by $T$.
		Hence, $T$ is both injective, and surjective as we aimed to show.
		As a result of this map $T$, it follows that there exists a unique polynomial $p \in \mathcal{P}_{m}(\F)$ such that $p(z_{j}) = w_{j}$ for $j = 1, \dots, m+1$.	
		\end{proof}
		% subsection exercise_5 (end)
	% section section_4 (end)

	\section*{Section 5.A}
		\subsection*{Exercise 3}
		Suppose $S,T \in \Ell(V)$ are such that $ST = TS$. 
		Prove that $\range S$ is invariant under $T$.

		\begin{proof}
		Let $S,T$ be as above, and let $u \in \range S$.
		Then, there exists $v \in V$ such that $u = Sv$.
		To show that $\range S$ is invariant under $T$, we will show that $Tu \in \range S$.
		Consider the following,
			\begin{align*}
				Tu &= T(Sv), \\
				&= (TS)v, \\
				&= (ST)v, \\
				&= S(Tv) \in \range S.
			\end{align*}
		Since $u$ was arbitrarily chosen, we have that $Tu \in \range S$, hence $\range S$ is invariant under $T$, as we aimed to prove.
			
		\end{proof}
		% subsection exercise_3 (end)

		\subsection*{Exercise 6}
		Prove or give a counterexample: if $V$ is finite-dimensional and $U$ is a subspace of $V$ that is invariant under every operator on $V$, then $U = \{ 0 \}$ or $U = V$.

		\begin{proof}
		Let $V$ be finite-dimensional. 
		We will proceed by contrapositive and show that $U \neq \{0\}$ and $U \neq V$ together imply that there exists an operator $T$ on $V$ such that $U$ is not invariant under $T$.
		Let $U$ be a subspace of $V$ such that $U \neq \{0\}$ and $U \neq V$.
		Then, let $u \in U - \{0\}$.
		Since $u \neq 0$, it is linearly independent as a list, extend it to a basis of $V$, e.g., $u, v_{1}, \dots, v_{m}$ is a basis of $V$.
		Define a linear operator $T : V \to V$ by $a_{1}u + a_{2}u_{1} + \dots + u_{m+1}u_{m} \mapsto b_{1}w$ where $w \in V/U$.
		Notice that since $w \in V/U$, we know that $w \notin U$; moreover, we have that $Tu = w$ which implies that $U$ is not invariant under $T$, as we aimed to show.
		\end{proof}
		% subsection exercise_6 (end)

		\subsection*{Exercise 12}
		Define $T \in \Ell\left( \mathcal{P}_{4}(\R) \right)$ by $$(Tp)(x) = xp'(x)$$ for all $x = \R$. 
		Find all eigenvalues and eigenvectors of $T$.

		\begin{proof}
		Let $q \in \mathcal{P}_{4}(\F)$, and let $1,x,x^2,x^3,x^4$ be a basis of $\mathcal{P}_{4}(\F)$, then $$q(x) = a_{0} + a_{1}x + a_{2}x^{2} + a_{3}x^{3} + a_{4}x^{4}.$$ We can see that applying $T$ to $q$ yields $Tq = a_{1}x + 2a_{2}x^{2} + 3a_{3}x^{3} + 4a_{4}x^{4}$.
		Recall that linear maps are additive, therefore we have,
			\begin{align*}
				(Tq)(x) &= T(a_{0} + a_{1}x + a_{2}x^{2} + a_{3}x^{3} + a_{4}x^{4}), \\
				&= T(a_{0}) + T(a_{1}x) + T(a_{2}x^{2}) + T(a_{3}x^{3}) + T(a_{4}x^{4}), \\
				&= a_{1}x + 2a_{2}x^{2} + 3a_{3}x^{3} + 4a_{4}x^{4}, \\
				&= \lambda_{0}\left( 0 \right) + \lambda_{1}\left( a_{1}x \right) + \lambda_{2}\left( a_{2}x^{2} \right) + \lambda_{3}\left( a_{3}x^{3} \right) + \lambda_{4}\left( a_{4}x^{4} \right).
			\end{align*}
		Hence, we can see that $\lambda_{0} = 0, \lambda_{1} = 1, \lambda_{2} = 2, \lambda_{3} = 3, \lambda_{4} = 4$ with corresponding eigenvectors that can be seen above.
		To be clear, we have eigenvectors $v_{0} = 1, $
		\end{proof}
		% subsection exercise_12 (end)

		\subsection*{Exercise 17}
		Give an example of an operator $T \in \Ell \left( \R^{4} \right)$ such that $T$ has no (real) eigenvalues.

		\begin{proof}
		My initial idea for this proof was to modify Example 5.8(a) from the text, because it presents a simple case to think about, but I couldn't think about what a linear operator looks like that does counterclockwise rotation in $\R^{4}$.
		Fortunately, Example 5.8(a) presents an easily copy-able pattern, which we use here.

		Consider $T \in \Ell(\R^{4})$ such that $$(x_{1},x_{2},x_{3},x_{4}) \mapsto (-x_{2},x_{1},-x_{4},x_{3}).$$
		Then suppose $T$ has a real eigenvalues, i.e., suppose there exists $\lambda \in \R$ such that $T(x_{1},x_{2},x_{3},x_{4}) = \lambda(x_{1},x_{2},x_{3},x_{4})$ and one of $x_{i}$ is not zero.
		This implies that $(-x_{2},x_{1},-x_{4},x_{3}) = \lambda(x_{1},x_{2},x_{3},x_{4})$, hence we have the following system of equations,
			\begin{align*}
				\lambda x_{1} + x_{2} &= 0, \\
				\lambda x_{2} - x_{1} &= 0, \\
				\lambda x_{3} + x_{4} &= 0, \\
				\lambda x_{4} - x_{3} &= 0.
			\end{align*}
		Multiplying the corresponding binomials yields,
			\begin{align*}
			(\lambda x_{1} + x_{2})(\lambda x_{2} - x_{1}) &= 0, \\
			\lambda^{2}x_{1}x_{2} - \lambda x_{1}x_{1} + \lambda x_{2}x_{2} - x_{1}x_{2} &= 0, \\
			\lambda^{2}x_{1}x_{2} - (\lambda x_{1})x_{1} + (\lambda x_{2}) x_{2} - x_{1}x_{2} &= 0, \\
			\lambda^{2}x_{1}x_{2} + x_{2}x_{1} + x_{1}x_{2} - x_{1}x_{2} &= 0, \\
			\lambda^{2}x_{1}x_{2} + x_{1}x_{2} &=0, \\
			\lambda^{2}x_{1}x_{2} &= - x_{1}x_{2}.
			\end{align*}
		\textit{Mutatis mutandis} to show that we also have $\lambda^{2}x_{3}x_{4} = - x_{3}x_{4}$.
		Since at least one of $x_{i}$ is not zero, we have that $\lambda^{2} = -1$ which is true if and only if $\lambda = i \ \lightning$
		This contradicts our assumption that $\lambda \in \R$, hence $T$ has no real eigenvalues, as we aimed to prove.
		\end{proof}
		% subsection exercise_17 (end)

		\subsection*{Exercise 21}
		Suppose $T \in \Ell(V)$ is invertible.
		\begin{enumerate}[\hspace{5mm} (a)]
			\item Suppose $\lambda \in \F$ with $\lambda \neq 0$.
			Prove that $\lambda$ is an eigenvalue of $T$ if and only if $\tfrac{1}{\lambda}$ is an eigenvalue of $T^{-1}$.

			\item Prove that $T$ and $T^{-1}$ have the same eigenvectors.
		\end{enumerate}
		
		\begin{proof}
		Let $T \in \Ell(V)$ be invertible; recall that this is equivalent to $T$ is a bijection.

		\textbf{Part (a)}.
		Let $\lambda$ be as above.
		Recall, since $\F$ is a field and $\lambda \neq 0$, there exists $\lambda^{-1} = 1/\lambda \in \F$ such that $\lambda\lambda^{-1}=1$.

		$\Ra)$ Assume $\lambda$ is an eigenvalue of $T$.
		Then for some nonzero $v \in V$ we have $$Tv = \lambda v.$$
		Apply $T^{-1}$ to each side and we see that,
		\begin{align*}
			Tv &= \lambda v, \\
			T^{-1}Tv &= T^{-1}(\lambda v), \\
			v &= \lambda T^{-1}v, \\
			\lambda^{-1}v &= \lambda^{-1}\lambda T^{-1}v, \\
			\lambda^{-1}v &= T^{-1}v.
		\end{align*}
		Notice that by definition, $\lambda^{-1}$ is an eigenvalue of $T^{-1}$, as we aimed to show.

		$\La)$ Assume $\lambda^{-1}$ is an eigenvalue of $T^{-1}$.
		Then for some nonzero $v \in V$ we have $$T^{-1}v = \lambda^{-1} v.$$
		Apply $T$ to each side and we see that,
		\begin{align*}
			T^{-1}v &= \lambda^{-1} v, \\
			TT^{-1}v &= T(\lambda^{-1} v), \\
			v &= \lambda^{-1} Tv, \\
			\lambda v &= \lambda\lambda^{-1} Tv, \\
			\lambda v &= Tv.
		\end{align*}
		Notice that by definition, $\lambda$ is an eigenvalue of $T$, thus $\lambda$ is an eigenvalue of $T$ if and only if $\lambda^{-1}$ is an eigenvalue of $T^{-1}$, as we aimed to show.

		\textbf{Part (b)}.
		A linear operator $T \in \Ell(V)$ either has eigenvalues, or it doesn't.
		If $T$ has no eigenvalues, then it's trivial for the purposes of this question, so assume $T$ has eigenvalues $\lambda_{i}$ for $i = 0, 1, 2, \dots$; recall that if $\dim V = n$, then $i \leq n$, however if $V$ is infinite-dimensional, then $i$ has no upper bound.\footnote{A good example for the infinite-dimensional case would be to alter Exercise 5.A.12 so that instead of $\Ell(\mathcal{P}_{4}(\R)$ the maps were drawn from $\Ell(\mathcal{P}(\R))$. It is easily seen that in this case the eigenvalues would be $\lambda_{i} = i$ for $i = 0,1,2,\dots$.}
		We continue to disregard the case where $\lambda_{i} = 0$, because by Theorem 5.6 $(T - 0 \cdot I) = T$ is not invertible, so there doesn't even exist a $T^{-1}$ to have identical eigenvectors.
		Then, for some nonzero $v_{i} \in V$ we have $$Tv_{i} = \lambda_{i}v_{i}.$$
		Consider one of sets of equations from the proof of part (a) but with the addition of our subscripts,
		\begin{align*}
			Tv_{i} &= \lambda_{i}v_{i}, \\
			T^{-1}Tv_{i} &= T^{-1}\lambda_{i} v_{i}, \\
			v_{i} &= \lambda_{i} T^{-1}v_{i}, \\
			\lambda_{i}^{-1}v_{i} &= T^{-1}v_{i}.
		\end{align*}
		Notice that for each pair $\lambda_{i}, \lambda_{i}^{-1}$ guaranteed by part (a), they have the same corresponding eigenvector $v_{i}$.
		It follows that for $\lambda_{i} \neq 0$, the maps $T$ and $T^{-1}$ have the same eigenvectors.
		\end{proof}
		% subsection exercise_21 (end)
	% section section_5_a (end)
\end{document}