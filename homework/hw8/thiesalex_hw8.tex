%!TEX output_directory = temp
\documentclass[letterpaper, 12pt]{amsart}
	%%%%%%%%%%%%%%%%%%%%%%%%%%%%%%%%%%%%%%%%%%%%%%%%%%%%%%%%%%%%%%%%%%%%%%%%%%%%%%
	%%%%%%%%%%%%%%%%%%%%%%%%%%%% boilerplate packages %%%%%%%%%%%%%%%%%%%%%%%%%%%%
		\usepackage[margin=2in]{geometry}
		\usepackage{amsmath,amssymb,amsthm}
		\usepackage{marvosym}
		\usepackage[mathscr]{euscript}
		\usepackage{enumerate}
		\usepackage{graphicx}
		\usepackage{mathrsfs}
		\usepackage{color}
		\usepackage{hyperref}
		\usepackage{verbatim}
		\usepackage{stmaryrd}

	%%%%%%%%%%%%%%%%%%%%%%%%%%%%%%%%%%%%%%%%%%%%%%%%%%%%%%%%%%%%%%%%%%%%%%%%%%%%%%
	%%%%%%%%%%%%%%%%%%%%%%%%%%%%% rename the abstract %%%%%%%%%%%%%%%%%%%%%%%%%%%%
		\renewcommand{\abstractname}{Assignment}

	%%%%%%%%%%%%%%%%%%%%%%%%%%%%%%%%%%%%%%%%%%%%%%%%%%%%%%%%%%%%%%%%%%%%%%%%%%%%%%
	%%%%%%%%%%%%%%%%%%%%%%%%%%%%%%%%%%%%% sets %%%%%%%%%%%%%%%%%%%%%%%%%%%%%%%%%%%
		\DeclareMathOperator{\N}{\mathbb{N}}				% natural numbers
		\DeclareMathOperator{\Z}{\mathbb{Z}}				% integers
		\DeclareMathOperator{\Zp}{\mathbb{Z}^{+}}			% positive integers
		\DeclareMathOperator{\Q}{\mathbb{Q}}				% rationals
		\DeclareMathOperator{\Qc}{\mathbb{Q}^{c}}			% irrationals
		\DeclareMathOperator{\R}{\mathbb{R}}				% reals
		\DeclareMathOperator{\F}{\mathbb{F}}				% a field
		\DeclareMathOperator{\C}{\mathbb{C}}				% complex numbers
		\DeclareMathOperator{\Cnon}{\mathbb{C}^{\times}}	% nonzero complex numbers
		\DeclareMathOperator{\Pcal}{\mathcal{P}}			% powerset, or set of polynomials
		\DeclareMathOperator{\Ell}{\mathscr{L}}				% set of linear maps, or linear operator

	%%%%%%%%%%%%%%%%%%%%%%%%%%%%%%%%%%%%%%%%%%%%%%%%%%%%%%%%%%%%%%%%%%%%%%%%%%%%%%
	%%%%%%%%%%%%%%%%%%%%%%%%%%%%%% use pretty letters %%%%%%%%%%%%%%%%%%%%%%%%%%%%
		\DeclareMathOperator{\ep}{\varepsilon}				% epsilons
		\DeclareMathOperator{\ph}{\varphi}					% phis

	%%%%%%%%%%%%%%%%%%%%%%%%%%%%%%%%%%%%%%%%%%%%%%%%%%%%%%%%%%%%%%%%%%%%%%%%%%%%%%
	%%%%%%%%%%%%%%%%%%%%%%%%%%%%%%%%%%% algebra %%%%%%%%%%%%%%%%%%%%%%%%%%%%%%%%%%
		\renewcommand{\null}{\text{null }}					% null space
		\DeclareMathOperator{\range}{\text{range }}			% range
		\newcommand{\bmat}[1]{{\mathbf{#1}}}				% bold matrix
		\newcommand{\bvec}[1]{{\vec{\mathbf{#1}}}}			% bold vector
		\DeclareMathOperator{\ind}{\perp\!\!\!\perp}		% perpendicular, orthogonal
		\DeclareMathOperator{\ord}{\text{ord}}				% order of a structure
		\DeclareMathOperator{\Log}{Log}						% logarithm
		\DeclareMathOperator{\Span}{Span}					% span
		\newcommand{\pid}[1]{\langle #1 \rangle}			% bracket notation, used for 
															% ideals or inner products
		\newcommand{\norm}[1]{\mid \!\!#1 \!\!\mid}			%\norm{x} gives |x|

		% fatdot notation
		\makeatletter
			\newcommand*\fatdot{\mathpalette\fatdot@{.5}}
			\newcommand*\fatdot@[2]{\mathbin{\vcenter{\hbox{\scalebox{#2}{$\m@th#1\bullet$}}}}}
		\makeatother

	%%%%%%%%%%%%%%%%%%%%%%%%%%%%%%%%%%%%%%%%%%%%%%%%%%%%%%%%%%%%%%%%%%%%%%%%%%%%%%
	%%%%%%%%%%%%%%%%%%%%%%%%%%% probability & statistics %%%%%%%%%%%%%%%%%%%%%%%%%
		\renewcommand{\Pr}{\mathbb{P}}						% probability
		\DeclareMathOperator{\E}{\mathbb{E}}				% expectation
		\DeclareMathOperator{\var}{\rm Var}					% variance
		\DeclareMathOperator{\sd}{\rm SD}					% standard deviation
		\DeclareMathOperator{\cov}{\rm Cov}					% covariance
		\DeclareMathOperator{\SE}{\rm SE}					% standard error
		\DeclareMathOperator{\ssreg}{{\rm SS}_{{\rm Reg}}}	% sum of squared regression
		\DeclareMathOperator{\ssr}{{\rm SS}_{{\rm Res}}}	% sum of squared residuals
		\DeclareMathOperator{\sst}{{\rm SS}_{{\rm Tot}}}	% total sum of squares

	%%%%%%%%%%%%%%%%%%%%%%%%%%%%%%%%%%%%%%%%%%%%%%%%%%%%%%%%%%%%%%%%%%%%%%%%%%%%%%
	%%%%%%%%%%%%%%%%%%%%%%%%%%%%%%% number theory %%%%%%%%%%%%%%%%%%%%%%%%%%%%%%%%
		\renewcommand{\mod}[1]{\ (\mathrm{mod}\ #1)}		% congruences

	%%%%%%%%%%%%%%%%%%%%%%%%%%%%%%%%%%%%%%%%%%%%%%%%%%%%%%%%%%%%%%%%%%%%%%%%%%%%%%
	%%%%%%%%%%%%%%%%%%%%%%%%%%%% theorem environments %%%%%%%%%%%%%%%%%%%%%%%%%%%%
		% Some theorem-like environments, all numbered together starting at 1
		% in each section.

		\newtheorem{thm}{Theorem}[section]					% The default \theoremstyle is 
		\newtheorem{defn}[thm]{Definition}					% bold headings and italic body text.
		\newtheorem{prop}[thm]{Proposition}
		\newtheorem{claim}[thm]{Claim}
		\newtheorem{cor}[thm]{Corollary}
		\newtheorem{lemma}[thm]{Lemma}

		\theoremstyle{definition}  							% Bold headings and Roman body text.
		\newtheorem{example}[thm]{Example}
		\newtheorem{examples}[thm]{Examples}
		\newtheorem{exercise}[thm]{Exercise}
		\newtheorem{note}[thm]{Note}
		\newtheorem{remark}[thm]{Remark}
		\newtheorem{remarks}[thm]{Remarks}
		\newtheorem{discussion}[thm]{Discussion}

		\newcommand{\dfn}{\textbf} 							% Make defined words bold.
		\newcommand{\mdfn}[1]{\dfn{\mathversion{bold}#1}} 	% Even make math symbols bold

	%%%%%%%%%%%%%%%%%%%%%%%%%%%%%%%%%%%%%%%%%%%%%%%%%%%%%%%%%%%%%%%%%%%%%%%%%%%%%%
	%%%%%%%%%%%%%%%%%%%%%%%%%%%%%%% complex numbers %%%%%%%%%%%%%%%%%%%%%%%%%%%%%%
		\DeclareMathOperator{\Arg}{Arg}						% argument of z \in \C
		\DeclareMathOperator{\re}{Re}						% real component
		\DeclareMathOperator{\im}{Im}						% imaginary component

	%%%%%%%%%%%%%%%%%%%%%%%%%%%%%%%%%%%%%%%%%%%%%%%%%%%%%%%%%%%%%%%%%%%%%%%%%%%%%%
	%%%%%%%%%%%%%%%%%%%%%%%%%%%%%%% various symbols %%%%%%%%%%%%%%%%%%%%%%%%%%%%%%
		\newcommand{\iso}{\cong}						% isometric/congruent
		\newcommand{\ra}{\rightarrow}                   % right arrow
		\newcommand{\Ra}{\Rightarrow}                   % right implies
		\newcommand{\lra}{\longrightarrow}              % long right arrow
		\newcommand{\la}{\leftarrow}                    % left arrow
		\newcommand{\La}{\Leftarrow}                    % left implies
		\newcommand{\lla}{\longleftarrow}               % long left arrow
		\newcommand{\eqra}{\llra{\sim}}                 % equivalence/isomorphism
		\newcommand{\blank}{\underbar{\ \ }}          	% An underscore, as in (__)xV
		% \newcommand{\blank}{-}                          % A hyphen, as in (-)xV
		\newcommand{\Id}{Id}                            % The identity functor
		\newcommand{\und}{\underline}
		\newcommand{\del}{\nabla}						% gradient vector

		\raggedbottom		
\begin{document}
	\title{Homework 8  -- Math 441 \\ M\lowercase{ay 30, 2018}}
	\author{Alex Thies \\ \href{mailto:athies@uoregon.edu}{\lowercase{athies$@$uoregon.edu}}}

	\begin{abstract}
	The following exercises are assigned from \textit{Linear Algebra Done Right}, 3rd Edition, by Sheldon Axler. 
			\begin{tabular}{rl}
				& 5.A - 25, 30, 33; \\
				& 5.B - 2, 6, 7, 13;
			\end{tabular}
	\end{abstract}
	
	\maketitle

	\section*{Section 5.A}
		\subsection*{Exercise 25}
		Suppose $T \in \Ell(V)$ and $u,v$ are eigenvectors of $T$ such that $u + v$ is also an eigenvector of $T$. 
		Prove that $u$ and $v$ are eigenvectors of $T$ corresponding to the same eigenvalue.

		\begin{proof}
		Let $T,u,v$ be as above.
		Let $\lambda_{1},\lambda_{2},\lambda_{3} \in \F$ be the eigenvalues corresponding to $u,v,u+v$.
		Then we have $$Tu = \lambda_{1}u, \hspace{2.5mm} \text{and } \hspace{2.5mm} Tv = \lambda_{2}v.$$
		Since $u+v$ is also an eigenvector, we can see that
			\begin{align*}
				T(u+v) &= \lambda_{3}(u + v), \\
				Tu + Tv &= \lambda_{3}u + \lambda_{3}v, \\
				\lambda_{1}u + \lambda_{2}v &= \lambda_{3}u + \lambda_{3}v, \\
				u(\lambda_{1} - \lambda_{3}) + v(\lambda_{2} - \lambda_{3}) &= 0.
			\end{align*}
		Since $u$ and $v$ are eigenvectors we know that $u \neq 0$ and $v \neq 0$. 
		Therefore, since $\lambda_{i} \in \F$ where $\F$ is a field, the above implies that $\lambda_{1} = \lambda_{3}$ and $\lambda_{2} = \lambda_{3}$, so by transitivity they are each equal to one another.
		It follows that $u$ and $v$ have the same corresponding eigenvalue, as we aimed to prove.
		\end{proof}
		% subsection exercise_25 (end)
		\pagebreak

		\subsection*{Exercise 30}
		Suppose $T \in \Ell(\R^{3})$ and 􏰋$-4$, $5$, and $\sqrt{7}$ are eigenvalues of $T$.
		Prove that there exists $x \in \R^{3}$ such that $Tx - 9x = (-4, 5, \sqrt{7})$.

		\begin{proof}
		Let $T$ be as above.
		Since $\dim \F^{3} = 3$ we know by Theorem 5.13 that $\lambda_{1} = -4$, $\lambda_{2} = 5$, and $\lambda_{3} = \sqrt{7}$ are all of $T$'s eigenvalues, and therefore $9$ is not an eigenvalue of $T$.
		Thus, by Theorem 5.6, it follows that $T - 9I$ is surjective.
		It follows that for $(-4, 5, \sqrt{7}) \in \F^3$ there exists $x \in \F^3$ such that $(T-9I)x = Tx - 9x = (-4, 5, \sqrt{7})$, as desired.
		\end{proof}
		% subsection exercise_30 (end)

		\subsection*{Exercise 33}
		Suppose $T \in \Ell(V)$.
		Prove that $T/(\range T) = 0$.

		\begin{proof}
		Let $T$ be as above.
		Recall that $\range T \subseteq V$ is invariant under $T$.
		Therefore, $T/(\range T)$ is a quotient operator.
		Let $v + \range T \in V/(\range T)$, then by the definition of quotient operators we have $$T/(\range T)(v + \range T) = Tv + \range T.$$
		Notice that $Tv \in \range T$, so $Tv + \range T = 0$, therefore $T/(\range T)(v + \range T) = 0$.
		Since $v + \range T$ is arbitrary, it follows that $T/(\range T) = 0$, as we aimed to prove.
		\end{proof}
		% subsection exercise_33 (end)
	% section section_5_a (end)

	\section*{Section 5.B}
		\subsection*{Exercise 2}
		Suppose $T \in \Ell(V)$ and $(T - 2I)(T - 3I)(T - 4I) = 0$.
		Suppose􏰀 $\lambda$ is an eigenvalue of $T$. 
		Prove that $\lambda = 2$, or $\lambda = 3$, or $\lambda = 4$.

		\begin{proof}
		Let $T \in \Ell(V)$ and $\lambda$ be an eigenvalue of $T$ with corresponding eigenvector $v \in V$.
		Notice that $(T - 2I)(T - 3I)(T - 4I) = 0$ is a third-degree polynomial operator that has been factored.
		We'll show that we can write this polynomial in terms of the eigenvalue $\lambda$ in place of the operator $T$.
		
		Observe the following pattern,
			\begin{align*}
				T^{2}v &= T(Tv), \\
				&= T(\lambda v), \\
				&= \lambda^2 v; \\
				\\
				T^3v &= T(T^2v), \\
				&= T(\lambda^2v), \\
				&= \lambda^{3}; \\
				&\vdots \\
				T^nv &= T(T^{n-1}v), \\
				&= T(\lambda^{n-1}v), \\
				&= \lambda^{n}v.
			\end{align*}
		This allows us to write $$(T - 2I)(T - 3I)(T - 4I) = (\lambda - 2I)(\lambda - 3I)(\lambda - 4I)$$ for the eigenvalue $\lambda$.
		Consider $((\lambda - 2I)(\lambda - 3I)(\lambda - 4I))v = 0$ for the corresponding eigenvector $v$.
		Since $v \neq 0$, it follows that $(\lambda - 2I)(\lambda - 3I)(\lambda - 4I) = 0$.
		This occurs when $\lambda = 2$, or $\lambda = 3$, or $\lambda = 4$, as desired.
		\end{proof}
		% subsection exercise_2 (end)

		\subsection*{Exercise 6}
		Suppose $T \in \Ell(V)$ and $U$ is a subspace of $V$ invariant under $T$. 
		Prove that $U$ is invariant under $p(T)$ for every polynomial $p \in \mathcal{P}(\F)$.

		\vspace{2mm}
		\noindent{Before we begin, it would help to have a little lemma.}

		\setcounter{section}{5}
		\begin{lemma}[$U$ is invariant under $T^{n}$]\label{lemma}
		Suppose $T \in \Ell(V)$ and $U$ is a subspace of $V$ invariant under $T$, then $U$ is invariant under $T^{n}$ for each $n \in \N$.
		\end{lemma}
		\begin{proof}
		We proceed by induction over the degree of $T$.

		\textbf{Base case}. Let $n = 1$, then we have $Tu \in U$ by assumption.

		\textbf{Hypothesis}. Assume for some $n \in \N$ that $T^{n}u \in U$, i.e., $T^{n}u = u'$ for some $u' \in U$.

		\textbf{Induction}. Consider $T^{n+1}u$.
		Compute that $$T^{n+1}u = T(T^{n}u) = Tu' \in U.$$
		By the principle of mathematical induction, we have shown that $T^{n}u \in U$.
		Notice that since $U$ is a vector space, $\lambda T^{n}u \in U$ by homogeneity.
		\end{proof}

		Now we can get on with the exercise.
		\begin{proof}
		Let $T \in \Ell(V)$ and $U$ be as above, let $p \in \mathcal{P}(\F)$.
		We will show that $U$ is invariant under $p(T)$ for every polynomial $p \in \mathcal{P}(\F)$.
		We use induction again, this time over the degree of $p$.

		\textbf{Base case}. Let $n=1$, then $(p(T))u = \lambda_{0}Iu + \lambda_{1}Tu = \lambda_{0}u + \lambda_{1}u' \in U$.
		Hence for $\deg p = 1$, $(p(T))u \in U$.

		\textbf{Hypothesis}. Assume for some $n-1$ that $U$ is invariant under any $p \in \mathcal{P}(\F)$ such that $\deg p \leq n-1$.

		\textbf{Induction}. Write $p(T) = \left( \sum_{i=1}^{n} \lambda_{i}T^{i} \right)u$.
		We compute the following, $$(p(T))u = \left( \sum_{i=1}^{n} \lambda_{i}T^{i} \right)u = \left( \sum_{i=1}^{n-1} \lambda_{i}T^{i} \right)u + \lambda_{n}T^{n}u.$$
		By the induction hypothesis, we have $\left( \sum_{i=1}^{n-1} \lambda_{i}T^{i} \right)u \in U$, and by Lemma \ref{lemma} we have $\lambda_{n}T^{n}u \in U$.
		It follows that $$\left( \sum_{i=1}^{n-1} \lambda_{i}T^{i} \right)u + \lambda_{n}T^{n}u = \left( \sum_{i=1}^{n} \lambda_{i}T^{i} \right)u \in U.$$
		Thus, by the principle of mathematical induction, we have shown that $U$ is invariant under $p(T)$ for every polynomial $p \in \mathcal{P}(\F)$, as desired.
		\end{proof}
		% subsection exercise_6 (end)

		\subsection*{Exercise 7}
		Suppose $T \in \Ell(V)$. 
		Prove that $9$ is an eigenvalue of $T^2$ if and only if $3$ or $-􏰋3$ is an eigenvalue of $T$.

		\begin{proof}
		Let $T \in \Ell(V)$.

		$\Ra)$ Assume $\lambda = 9$ is an eigenvalue of $T^2$ with corresponding eigenvector $v$.
		Then by Theorem 5.10 we know $T^2-9I = (T+3I)(T-3I)$ is not injective.
		In Exercise 3.B.11 we showed that the product of injective linear maps is injective.
		If we apply the contrapositive in this case, that the product of linear maps not being injective implies that the factors are each not injective, then we see that each of $(T+3I),(T-3I)$ are not injective.
		Theorem 5.10 can be applied again to conclude that $\pm3$ are eigenvalues for $T$, as desired.
		It remains to prove the converse.

		$\La)$ Assume $\lambda = \pm3$ is an eigenvalue of $T$ with corresponding eigenvector $v$.
		In Exercise 2 we showed that $T^nv = \lambda^nv$, so we can apply that here and compute that 
			\begin{align*}
				T^{2}v &= \lambda^2v, \\
				&= 3^2v = (-3)^2v, \\
				&= 9v.
			\end{align*}
		It follows that $9$ is an eigenvalue of $T^2$, as we aimed to show.
		\end{proof}
		% subsection exercise_7 (end)

		\subsection*{Exercise 13}
		Suppose $W$ is a complex vector space and $T \in \Ell(W)$ has no eigenvalues. 
		Prove that every subspace of $W$ invariant under $T$ is either $\{ 0 \}$ or infinite- dimensional.

		\begin{proof}
		Let $W$ be a vector space over $\C$ and $T \in \Ell(W)$ is such that $T$ has no eigenvalues.
		Suppose by way of contradiction that $U$ is a finite-dimensional subspace of $W$ that is invariant under $T$, and $U \neq \{0\}$.
		Since $U \neq \{0\}$, we know by Theorem 5.21 that $T|_{U}$ has an eigenvalue $\lambda$ with corresponding eigenvector $v$ and $v \neq 0 \, \lightning$
		This contradicts our assumption that $T$ has no eigenvalues, hence $U$ must either be $\{ 0 \}$, or infinite-dimensional.
		\end{proof}
		% subsection exercise_13 (end)
	% section section_5_b (end)
\end{document}